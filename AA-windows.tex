% The contents of this file is 
% Copyright (c) 2009-  Charles R. Severance, All Righs Reserved

\chapter{윈도우 상에서 파이썬 프로그래밍}

이번 부록에서 일련의 단계를 거쳐서 윈도우상에서 파이썬을 실행한다.

파이썬 프로그램을 편집하고 실행하는데취할 수 있는 서로 다른 많은 접근방법이 있다. 이것은 간단한 접근 방법중의 하나다.

먼저, 프로그래머 편집기를 설치할 필요가 있다.
Notepad나 윈도우 워드를 가지고 파이썬 프로그램을 편집할 필요는 없다.
프로그램은 ''일반 텍스트(flat-text)'' 파일이여서 텍스트 파일을 편집하는데 좋은 에디터만 필요하다.

윈도우 시스템에 추천하고 싶은 편집기는 다음에서 다운받아 설치할 수 있는 Notepad++다.

\url{http://sourceforge.net/projects/notepad-plus/files/}

\url{www.python.org} 웹사이트에서 파이썬 2를 다운로드한다.

\url{http://www.python.org/download/releases/2.7.5/}

파이썬을 설치하면, 컴퓨터에 {\tt C:{\textbackslash}Python27} 같은 새로운 폴더가 생긴다.

파이썬 프로그램을 생성하기 위해서 시작 메뉴에서 NotePad++를 실행하고 확장자가 ''.py''인 파일로 저장한다.
연습으로 {\tt py4inf} 이름의 폴더를 바탕화면에 생성한다.
폴더명을 짧게 하고 폴더명과 파일명에 어떤 공백도 넣지 않는 것이 좋다.

첫 파이썬 프로그램을 다음과 같이 작성한다.

\beforeverb
\begin{verbatim}
print 'Hello Chuck'
\end{verbatim}
\afterverb
%

여러분의 이름으로 바꾸는 것을 제외하고, 파일을 {\tt 바탕화면{\textbackslash}py4inf{\textbackslash}prog1.py}에 저장한다.

명령-줄(command line) 실행은 윈도우 버젼마다 다른다.

\begin{itemize}
\item 윈도우 비스타, 윈도우 7: {\bf 시작(Start)}을 누루고, 명령어 실행 윈도우에서 단어 {\tt command}를 입력하고 엔터를 친다.

\item 윈도우-XP: {\bf 시작(Start)}을 누루고,{\bf 실행(Run)}을 누루고, 대화창에서 단어 {\tt cmd}를 입력하고 {\bf 확인(OK)}을 누른다.
\end{itemize}

지금 어느 폴더에 있는지를 말해주는 프롬프트 텍스트 윈도우에서 현재 위치를 확인할 수 있다.

Windows Vista and Windows-7: {\tt C:{\textbackslash}Users{\textbackslash}csev}\\
Windows XP: {\tt C:{\textbackslash}Documents and Settings{\textbackslash}csev}

이것이 ''홈 디렉토리''다. 이제 다음 명령어를 사용해서 작성한 파이썬 프로그램을 저장한 폴더로 이동한다.

\beforeverb
\begin{verbatim}
C:\Users\csev\> cd Desktop
C:\Users\csev\Desktop> cd py4inf
\end{verbatim}
\afterverb
%

그리고 다음을 타이핑한다.

\beforeverb
\begin{verbatim}
C:\Users\csev\Desktop\py4inf> dir 
\end{verbatim}
\afterverb
%

작성한 파일 목록을 보기 위해서 {\tt dir} 명령어를 타이핑 할때, {\tt prog1.py}이 보여야 한다.

프로그램을 실행하기 위해서, 단순히 명령 프롬프트에서 파일 이름을 타이핑하고 엔터를 친다.

\beforeverb
\begin{verbatim}
C:\Users\csev\Desktop\py4inf> prog1.py
Hello Chuck
C:\Users\csev\Desktop\py4inf> 
\end{verbatim}
\afterverb
%

NotePad++에서 파일을 편집하고, 저장하고, 명령줄로 돌아온다.
다시 명령줄 프롬프트에서 파일명을 타이핑해서 프로그램을 실행한다.

만약 명령줄 윈도우에서 혼동이 생기면, 단순하게 닫고 새로 시작한다.

힌트: 스크롤 백해서 이전에 입력한 명령을 다시 실행하기 위해서 ''위쪽 화살표''를 명령줄에서 누른다.

NotePad++에서 환경설정 선호(preference)를 살펴보고 탭 문자가 공백 4개로 되도록 설정한다.
이 단순한 설정이 들여쓰기 오류를 찾는 수고를 많이 경감시켜 준다.

\url{www.py4inf.com}에서 파이썬 프로그램을 편집하고 실행하는 좀더 많은 정보를 얻을 수 있다.

