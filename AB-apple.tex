% The contents of this file is 
% Copyright (c) 2009-  Charles R. Severance, All Righs Reserved

\chapter{매킨토쉬 상에서 파이썬 프로그래밍}

이번 부록에서 일련의 단계를 거쳐서 매킨토쉬상에서 파이썬을 실행한다.
파이썬이 이미 매킨토쉬 운영시스템에 포함되어 있어서,
어떻게 파이썬 파일을 편집하는지와 터미널 윈도우에서 파이썬 프로그램을 실행하는 것을 학습할 필요가 있다.

파이썬 프로그램을 편집하고 실행하는데 취할 수 있는 서로 다른 많은 접근방법이 있다. 이것은 간단한 접근 방법중의 하나다.

먼저, 프로그래머 편집기를 설치할 필요가 있다.
Notepad나 윈도우 워드를 가지고 파이썬 프로그램을 편집할 필요는 없다.
프로그램은 ''일반 텍스트(flat-text)'' 파일이여서 텍스트 파일을 편집하는데 좋은 에디터만 필요하다.

매킨토쉬 시스템에 추천하고 싶은 편집기는 다음에서 다운받아 설치할 수 있는 TextWrangler다.

\url{http://www.barebones.com/products/TextWrangler/}

파이썬 프로그램을 생성하기 위해서 {\bf Applications} 폴더 {\bf TextWrangler}를 실행한다.

첫 파이썬 프로그램을 다음과 같이 작성한다.

\beforeverb
\begin{verbatim}
print 'Hello Chuck'
\end{verbatim}
\afterverb
%


여러분의 이름으로 바꾸는 것을 제외하고, 파일을 


{\tt 바탕화면{\textbackslash}py4inf{\textbackslash}prog1.py}에 저장한다.

명령-줄(command line) 실행은 윈도우 버젼마다 다른다. {\tt py4inf}로 데스크탑(Desktop) 폴더에 파일을 저장한다.
폴더명을 짧게 하고 폴더명과 파일명에 어떤 공백도 넣지 않는 것이 좋다.
폴더를 만들고 파일을 {\tt Desktop{\textbackslash}py4inf{\textbackslash}prog1.py} 으로 저장한다.

{\bf 터미널(Terminal)} 프로그램을 실행. 가장 쉬운 방법은 화면 우측 상단의 Spotlight 아이콘(돋보기)를 누르고,
''terminal'' 엔터치고 응용프로그램을 실행한다.

''홈 디렉토리''에서 시작한다. 터미널 윈도우에서 {\tt pwd} 명령어를 타이핑해서 현재 디렉토리를 확인한다.

\beforeverb
\begin{verbatim}
67-194-80-15:~ csev$ pwd
/Users/csev
67-194-80-15:~ csev$ 
\end{verbatim}
\afterverb
%

프로그램을 실행하기 위해서 파이썬 프로그램을 담고 있는 폴더에 있어야 한다.
{\tt cd} 명령어를 사용해서 새 폴더로 이동하고 {\tt ls} 명령어로 폴더의 파일 목록을 화면에 출력한다.

\beforeverb
\begin{verbatim}
67-194-80-15:~ csev$ cd Desktop
67-194-80-15:Desktop csev$ cd py4inf
67-194-80-15:py4inf csev$ ls
prog1.py
67-194-80-15:py4inf csev$ 
\end{verbatim}
\afterverb
%

프로그램을 실행하기 위해서, 단순히 명령 프롬프트에서 {\tt python} 명령어와 파일 이름을 타이핑하고 엔터를 친다.

\beforeverb
\begin{verbatim}
67-194-80-15:py4inf csev$ python prog1.py
Hello Chuck
67-194-80-15:py4inf csev$ 
\end{verbatim}
\afterverb
%

TextWrangler에서 파일을 편집하고, 저장하고, 명령줄로 돌아온다.
다시 명령줄 프롬프트에서 파일명을 타이핑해서 프로그램을 실행한다.

만약 명령줄 윈도우에서 혼동이 생기면, 단순하게 닫고 새로 시작한다.

힌트: 스크롤 백해서 이전에 입력한 명령을 다시 실행하기 위해서 ''위쪽 화살표''를 명령줄에서 누른다.

TextWrangler에서 환경설정 선호(preference)를 살펴보고 탭 문자가 공백 4개로 되도록 설정한다.
이 단순한 설정이 들여쓰기 오류를 찾는 수고를 많이 경감시켜 준다.

\url{www.py4inf.com}에서 파이썬 프로그램을 편집하고 실행하는 좀더 많은 정보를 얻을 수 있다.
