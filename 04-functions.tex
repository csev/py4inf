% LaTeX source for ``Python for Informatics: Exploring Information''
% Copyright (c)  2010-  Charles R. Severance, All Rights Reserved

\chapter{함수}
\label{funcchap}

\section{함수 호출}
\label{functionchap}
\index{function call}

프로그래밍 문맥에서, {\bf 함수(function)}는 연산을 수행하는 일련의 명명된 명령문이다. 
함수를 정의할 때, 이름과 일련의 명령문을 명시한다. 후에, 함수를 이름으로 ''호출(call)''할 수 있다.
이미 {\bf 함수 호출(function call)}의 예제를 살펴 보았다.  

\beforeverb
\begin{verbatim}
>>> type(32)
<type 'int'>
\end{verbatim}
\afterverb
%
함수명은 {\tt type}이다. 괄호안의 표현식을 함수의 {\bf 인수(argument)}라고 한다. 
인수는 함수의 입력값으로 함수 내부로 전달되는 값이나 변수이다. 앞의 {\tt type} 함수의 결과값은 인수의 형(type)이다.

\index{parentheses!argument in}

통상 함수가 인수를 받아 결과를 돌려준다고 한다. 결과를 {\bf 결과값(return value)}이라고 부른다.

\index{argument}
\index{return value}

\section{내장(Built-in) 함수}

파이썬에는 함수를 정의할 필요없이 사용할 수 있는 많은 중요 내장함수가 있다.
파이썬을 처음 만든 사람이 공통의 문제를 해결할 수 있는 함수를 작성해서 여러분이 사용할 수 있도록 파이썬에 포함을 했습니다.

{\tt max}와 {\tt min} 함수는 리스트의 최소값과 최대값을 각기 계산해서 여러분에게 보여줍니다.

\beforeverb
\begin{verbatim}
>>> max('Hello world')
'w'
>>> min('Hello world')
' '
>>>
\end{verbatim}
\afterverb
%
{\tt max} 함수는 문자열의 "가장 큰 문자", 상기 예제에서는 ''w'', 
{\tt min}함수는 최소 문자를, 상기 예제에서는 공백, 출력합니다.

또 다른 매우 자주 사용되는 내장 함수는 얼마나 많은 항목이 있는지 출력하는 {\tt len}함수가 있습니다.
만약 {\tt len} 함수의 인수가 문자열이면 문자열의 문자 갯수를 반환합니다.

\beforeverb
\begin{verbatim}
>>> len('Hello world')
11
>>>
\end{verbatim}
\afterverb
%

이들 함수는 문자열에만 한정된 것이 아니라, 뒷장에서 보듯이 다양하게 다양한 자료형에 사용될 수 있습니다.

내장함수의 이름은 사전에 점유된 예약어로 다뤄야 하고, 예를 들어 ''max''를 변수명으로 사용을 피해야 합니다.


\section{형 변환 함수}
\index{conversion!type}
\index{type conversion}

% from Elkner:
% comment on whether these things are _really_ functions?
% use max as an example of a built-in?

% my reply:
% they are on the list of ``built-in functions'' so I am
% willing to call them functions.

파이썬은 A형(type)에서 B형(type)으로 값을 변환하는 내장 함수를 제공합니다.
{\tt int}함수는 임의의 값을 입력 받아 변환이 가능하면 정수형으로 변환하고, 그렇지 않으면 불평을 합니다.

\index{int function}
\index{function!int}

\beforeverb
\begin{verbatim}
>>> int('32')
32
>>> int('Hello')
ValueError: invalid literal for int(): Hello
\end{verbatim}
\afterverb
%

{\tt int}는 부동 소수점 값을 정수로 변환할 수 있지만 소수점 이하를 절사합니다.

\beforeverb
\begin{verbatim}
>>> int(3.99999)
3
>>> int(-2.3)
-2
\end{verbatim}
\afterverb
%

{\tt float}는 정수와 문자열을 부동 소수점으로 변환합니다.

\index{float function}
\index{function!float}

\beforeverb
\begin{verbatim}
>>> float(32)
32.0
>>> float('3.14159')
3.14159
\end{verbatim}
\afterverb
%

마지막으로, {\tt str}은 인수를 문자열로 변환합니다.

\index{str function}
\index{function!str}

\beforeverb
\begin{verbatim}
>>> str(32)
'32'
>>> str(3.14159)
'3.14159'
\end{verbatim}
\afterverb
%

\section{난수(Random numbers)}

\index{random number}
\index{number, random}
\index{deterministic}
\index{pseudorandom}

동일한 입력을 받을 때, 대부분의 컴퓨터는 매번 동일한 출력값을 생성하기 때문에 {\bf 결정적(deterministic)}이라고 합니다.
결정론이 대체로 좋은데 왜냐하면, 동일한 계산은 동일한 결과를 기대할 수 있기 때문입니다. 하지만, 어떤 어플리케이션에서는 컴퓨터가 예측불가능하길 바랍니다. 게임이 좋은 예이고, 더 많은 예를 찾을 수 있습니다. 
프로그램을 온전하게 비결정론적으로 만드는 것은 쉽지 않은 것으로 밝혀졌지만, 비결정론적인 것처럼 보이게 하는 방법은 있습니다. {\bf 의사 난수(pseudorandom numbers)}를 생성하는 알고리즘을 사용하는 방법이 그 중 하나입니다. 의사 난수는 이미 결정된 연산에 의해서 생성된다는 점에서 
진정한 의미의 난수는 아니지만,이렇게 생성된 숫자를 진정한 난수와 구별하기는 불가능에 가깝다.

\index{random module}
\index{module!random}

{\tt random} 모듈은 의사 난수를 생성하는 함수를 제공한다. (이하 의사 난수 대신 ''난수(random)''로 간략히 부르기로 한다.)

\index{random function}
\index{function!random}

{\tt random} 함수는 0.0과 1.0 사이의 부동 소수점 난수를 반환한다. {\tt random} 함수는 0.0은 생성하지만 1.0은 생성하지 않는다.
매번 {\tt random} 함수를 호출할 때 마다 이미 생성된 난수열에서 하나씩 하나씩 뽑아 쓰게 됩니다. 샘플로 다음 반복문을 실행해 봅시다.

\beforeverb
\begin{verbatim}
import random

for i in range(10):
    x = random.random()
    print x
\end{verbatim}
\afterverb
%

프로그램이 0.0과 1.0을 포함하지 않는 최대 1.0에서 10개의 난수 리스트를 생성합니다.

\beforeverb
\begin{verbatim}
0.301927091705
0.513787075867
0.319470430881
0.285145917252
0.839069045123
0.322027080731
0.550722110248
0.366591677812
0.396981483964
0.838116437404
\end{verbatim}
\afterverb
%

\begin{ex}
여러분의 컴퓨터에 프로그램을 실행해서, 어떤 난수가 생성되는지 살펴보세요.
한번 이상 프로그램을 실행하여 보고, 어떤 난수가 생성되는지 다시 살펴보세요.
\end{ex}

{\tt random}함수는 난수를 다루는 단지 많은 함수 중의 하나입니다.
{\tt randint} 함수는 {\tt 최저(low)}와 {\tt 최고(high)} 매개 변수를 입력받아
{\tt 최저값(low)}와 {\tt 최고값(high)}을 포함하는 두 사이의 정수를 반환합니다.

\index{randint function}
\index{function!randint}

\beforeverb
\begin{verbatim}
>>> random.randint(5, 10)
5
>>> random.randint(5, 10)
9
\end{verbatim}
\afterverb
%

무작위로 배열로부터 하나의 숫자를 뽑아내기 위해서, {\tt choice}를 사용합니다.

\index{choice function}
\index{function!choice}

\beforeverb
\begin{verbatim}
>>> t = [1, 2, 3]
>>> random.choice(t)
2
>>> random.choice(t)
3
\end{verbatim}
\afterverb
%

{\tt random} 모듈은 정규분포, 지수분포, 감마분포 및 몇가지 추가된 연속형 분포에서
난수를 생성할 수 있는 함수도 제공합니다.

\section{수학 함수}
\index{math function}
\index{function, math}
\index{module}
\index{module object}

파이썬은 가장 친숙한 수학 함수를 제공하는 수학 모듈이 있습니다.
수학 모듈을 사용하기 전에, 수학 모듈 가져오기를 실행합니다.

\beforeverb
\begin{verbatim}
>>> import math
\end{verbatim}
\afterverb
%

이 명령문은 math 모듈 개체를 생성한다. 모듈 개체를 출력하면, 모듈 개체에 대한 정보를 얻을 수 있다.

\beforeverb
\begin{verbatim}
>>> print math
<module 'math' from '/usr/lib/python2.5/lib-dynload/math.so'>
\end{verbatim}
\afterverb
%

모듈 개체는 모듈에 정의된 함수와 변수를 담고 있다. 함수 중에 하나에 접근하기 위해서, 점으로 구분되는 모듈의 이름과 함수의 이름을 명시해야 한다.
이런 형식을 {\bf 점 표기법(dot notation)}이라고 부른다.

\index{dot notation}

\beforeverb
\begin{verbatim}
>>> ratio = signal_power / noise_power
>>> decibels = 10 * math.log10(ratio)

>>> radians = 0.7
>>> height = math.sin(radians)
\end{verbatim}
\afterverb
%

첫 예제는 로그 지수 10으로 신호 대비 소음 비율을  계산한다.
수학 모듈은 자연 로그를 {\tt log}함수를 호출해서 사용할 수 있도록 제공한다.

\index{log function}
\index{function!log}
\index{sine function}
\index{radian}
\index{trigonometric function}
\index{function, trigonometric}

두 번째 예제는 라디안의 사인값을 찾는 것이다. 변수의 이름이 힌트로 {\tt sin}과 다른 삼각 함수({\tt cos}, {\tt tan} 등)는 라디안을 인수로 받는다.
도수에서 라디안으로 변환하기 위해서 360으로 나누고 $2\pi$를 곱한다.

\beforeverb
\begin{verbatim}
>>> degrees = 45
>>> radians = degrees / 360.0 * 2 * math.pi
>>> math.sin(radians)
0.707106781187
\end{verbatim}
\afterverb
%

{\tt math.pi} 표현문은 수학 모듈에서 {\tt pi} 변수를 얻고, $\pi$와 근사적으로 동일하고 15 자리수까지 정확하다. 

\index{pi}

삼각함수를 배웠다면, 앞의 연산 결과를 2를 루트를 씌우고 2로 나누어서 비교한다.

\index{sqrt function}
\index{function!sqrt}

\beforeverb
\begin{verbatim}
>>> math.sqrt(2) / 2.0
0.707106781187
\end{verbatim}
\afterverb
%


\section{신규 함수 추가}

지금까지 파이썬에 딸려 있는 함수를 사용했지만 새로운 함수를 추가하는 것도 가능하다.
{\bf 함수 정의(function definition)}는 신규 함수명과 함수가 호출될 때 실행할 일련의 명령문을 명세한다.
함수를 신규로 정의하면, 프로그램 내내 반복해서 함수를 재사용할 수 있다. 

\index{function}
\index{function definition}
\index{definition!function}

여기 예제가 있다.

\beforeverb
\begin{verbatim}
def print_lyrics():
    print "I'm a lumberjack, and I'm okay."
    print 'I sleep all night and I work all day.'
\end{verbatim}
\afterverb
%
{\tt def}는 이것이 함수 정의를 나타내는 키워드입니다. 함수명은 \verb"print_lyrics"입니다.
함수명을 명명 규칙은 변수명과 동일합니다. 문자, 숫자, 그리고 몇몇 문장 부호는 사용할 수 있지만,
첫 문자가 숫자는 될 수 없다. 함수명으로 예약어 키워드를 사용할 수도 없고, 동일한 변수명과 함수명은 피하는 것이 좋다.

\index{def keyword}
\index{keyword!def}
\index{argument}

함수명 뒤에 빈 괄호는 이 함수가 어떠한 인수도 갖지 않는다는 것을 나타낸다.
나중에, 입력값으로 인수를 가지는 함수를 작성해 볼 것이다.

\index{parentheses!empty}
\index{header}
\index{body}
\index{indentation}
\index{colon}

함수 정의 첫번째 줄을 {\bf 머리 부문(헤더, header)}, 나머지 부문을 {\bf 몸통 부문(바디, body)}라고 부른다.
머리 부문은 콜론(:)으로 끝나고, 몸통 부문은 들여쓰기를 해야 한다.
파이썬 관례로 들여쓰기는 항상 4칸의 공백이다. 몸통 부문은 제약 없이 명령문을 작성할 수 있다.

print문의 문자열은 이중 인용부호로 감싸진다. 단일 인용부호나, 이중 인용부호나 차이가 없다.
대부분의 경우 단일 인용부호를 사용하고, 단일 인용부호가 문자열에 나타나는 경우, 이중 인용부호를 사용하여 단일 인용부호가 출력되게 감싼다.

\index{ellipses}

함수 정의를 인터렉티브 모드에서 타이핑을 한다면, 함수 정의가 끝나지 않았다는 것을 알 수 있도록 생략부호(...)를 출력한다.

\beforeverb
\begin{verbatim}
>>> def print_lyrics():
...     print "I'm a lumberjack, and I'm okay."
...     print 'I sleep all night and I work all day.'
...
\end{verbatim}
\afterverb
%

함수 정의를 끝내기 위해서는 엔터(Enter)키를 눌러 빈 줄을 삽입한다. (스크립트에서는 반듯이 필요한 것은 아니다.) 

함수를 정의하게 되면 동일한 이름의 변수도 생성된다.

\beforeverb
\begin{verbatim}
>>> print print_lyrics
<function print_lyrics at 0xb7e99e9c>
>>> print type(print_lyrics)
<type 'function'>
\end{verbatim}
\afterverb
%

\verb"print_lyrics"의 값은 \verb"'function'" 형을 가지는 {\bf 함수 개체(function object)}이다. 

\index{function object}
\index{object!function}
신규 함수를 호출하는 구문은 내장 함수의 경우와 동일합니다.

\beforeverb
\begin{verbatim}
>>> print_lyrics()
I'm a lumberjack, and I'm okay.
I sleep all night and I work all day.
\end{verbatim}
\afterverb
%

함수를 정의하면, 또 다른 함수 내부에서 사용이 가능합니다.
예를 들어, 이전의 후렴구를 반복하기 위해 \verb"repeat_lyrics" 함수를 작성할 수 있습니다.

\beforeverb
\begin{verbatim}
def repeat_lyrics():
    print_lyrics()
    print_lyrics()
\end{verbatim}
\afterverb
%

그리고 나서, \verb"repeat_lyrics" 함수를 호출합니다.

\beforeverb
\begin{verbatim}
>>> repeat_lyrics()
I'm a lumberjack, and I'm okay.
I sleep all night and I work all day.
I'm a lumberjack, and I'm okay.
I sleep all night and I work all day.
\end{verbatim}
\afterverb
%

하지만, 이 방법이 실제 노래가 불려지는 방법은 아닙니다.

\section{함수 정의와 사용법}
\index{function definition}

앞 절의 코드 조각을 모아서 작성한 전체 프로그램은 다음과 같다.

\beforeverb
\begin{verbatim}
def print_lyrics():
    print "I'm a lumberjack, and I'm okay."
    print 'I sleep all night and I work all day.'

def repeat_lyrics():
    print_lyrics()
    print_lyrics()

repeat_lyrics()
\end{verbatim}
\afterverb
%

상기 프로그램은 두개의 함수(\verb"print_lyrics", \verb"repeat_lyrics")를 담고 있다.
함수정의는 다른 명령문과 동일하게 수행되지만, 함수 개체를 생성한다는 점에서 차이가 있다.
함수 내부의 명령문은 함수가 호출되기 전까지 수행되지 않고, 함수 정의는 출력값도 생성하지 않는다.

\index{use before def}

예상하듯이, 함수를 실행하기 전에 함수를 생성해야 한다. 다시 말해서, 처음으로 호출되기 전에
함수 정의가 실행되어야 한다.

\begin{ex}
상기 프로그램의 마지막 줄을 최상단으로 옮겨서 함수 정의 전에 호출되도록 프로그램을 고쳐보세요.
프로그램을 실행서 오류 메시지를 확인하세요.
\end{ex}

\begin{ex}
함수 호출을 맨 마지막으로 옮기고, \verb"repeat_lyrics" 함수 정의 뒤에 \verb"print_lyrics" 함수를 옮기세요.
프로그램을 실행하게 되면 무슨 일이 발생하나요?
\end{ex}


\section{실행 흐름}
\index{flow of execution}

함수가 첫 사용전에 정의되는 것을 확인하기 위해서, 명령문의 실행 순서를 파악해야 하는데 이를 {\bf 실행 흐름(flow of execution)}이라고 한다.

프로그램 실행은 항상 프로그램의 첫 명령문부터 시작한다. 명령문은 한번에 하나씩 위에서 아래로 실행된다.

함수 \emph{정의(definitions)}가 프로그램의 실행 순서를 바꾸지는 않는다. 하지만, 함수 내부의 명령문은 함수가 호출될 때까지 실행이 되지 않는 것을 기억합니다. 

함수 호출은 프로그램의 실행 흐름을 우회하는 것과 같습니다. 다음 실행 명령문으로 가기 전에 실행 흐름은 함수의 몸통 부문으로 건너 뛰어 실행하고는 다시 건너 뛰기를 
시작한 지점으로 다시 돌아온다.

하나의 함수가 또 다른 함수를 호출한다는 것을 기억할 때까지는 매우 간단하게 들립니다.
함수 중간에서 프로그램이 또다른 함수의 명령문을 수행할지도 모릅니다. 하지만, 새로운 함수가 실행되는 중간에 프로그램이 또 다른 함수를 실행할지도 모릅니다!
 
다행스럽게도, 파이썬은 프로그램의 실행 위치를 정확히 추적합니다. 그래서, 함수가 실행을 완료할 때 마다, 프로그램이 함수를 호출해서 떠난 지점으로 정확히 되돌려 놓습니다. 프로그램의 마지막에 도달했을 때, 프로그램은 종료합니다.

조금 복잡한 이야기의 교훈은 무엇일까요? 프로그램을 읽을 때, 위에서부터 아래로 읽을 필요는 없습니다. 때때로, 실행 흐름을 따르는 것이 좀더 이치에 맞습니다.


\section{매개 변수(parameter)와 인수(argument)}
\label{parameters}
\index{parameter}
\index{function parameter}
\index{argument}
\index{function argument}

지금까지 살펴본 몇몇 내장 함수는 인수를 요구합니다. 예를 들어, {\tt math.sin} 함수를 호출할 때, 숫자를 인수로 넘겨야 합니다.
어떤 함수는 2개 이상의 인수를 받습니다. {\tt math.pow} 는 밑과 지수 2개의 인수가 필요합니다. 

인수는 함수 내부에서  {\bf 매개 변수(parameters)}로 불리는 변수로 할당됩니다.
하나의 인수를 받는 사용자 정의 함수가 예제로 있습니다.  

\index{parentheses!parameters in}

\beforeverb
\begin{verbatim}
def print_twice(bruce):
    print bruce
    print bruce
\end{verbatim}
\afterverb
%

사용자 정의 함수는 인수를 받아 {\tt bruce} 매개변수에 할당한다. 함수가 호출될 때, 매개변수의 값(무엇이든지 관계 없이)을 두번 출력합니다.

사용자 정의 함수는 출력가능한 임의의 값에 작동합니다.

\beforeverb
\begin{verbatim}
>>> print_twice('Spam')
Spam
Spam
>>> print_twice(17)
17
17
>>> print_twice(math.pi)
3.14159265359
3.14159265359
\end{verbatim}
\afterverb
%

내장함수에 적용되는 동일한 조합 규칙이 사용자 정의 함수에도 적용되어서, \verb"print_twice" 함수의 인수로 어떤 종류의 표현식도 가능합니다. 

\index{composition}

\beforeverb
\begin{verbatim}
>>> print_twice('Spam '*4)
Spam Spam Spam Spam
Spam Spam Spam Spam
>>> print_twice(math.cos(math.pi))
-1.0
-1.0
\end{verbatim}
\afterverb
%

인수는 함수가 호출되기 전에 평가가 완료되어서, 예제에서 \verb"'Spam '*4"과 {\tt math.cos(math.pi)}은 단지 1회만 평가가 됩니다.

\index{argument}

인수로 변수도 사용이 가능합니다.

\beforeverb
\begin{verbatim}
>>> michael = 'Eric, the half a bee.'
>>> print_twice(michael)
Eric, the half a bee.
Eric, the half a bee.
\end{verbatim}
\afterverb
%

인수로 넘기는 변수명({\tt michael})은 매개 변수명({\tt bruce})과 아무런 연관이 없다.
무슨 값이 호출된든 호출하는 측에서는 상관이 없다. 여기 \verb"print_twice" 함수에서 누구나 {\tt bruce}라고 부른다. 

\section{열매 함수(fruitful function)와 빈 함수(void function)}

\index{fruitful function}
\index{void function}
\index{function, fruitful}
\index{function, void} 

수학 함수와 같은 몇몇 함수는 결과를 만들어 낸다. 좀더 좋은 이름이 없어서, 결과를 만들어 내는 함수를 {\bf 열매함수(fruitful functions)}라고 명명한다.
\verb"print_twice"와 같은 명령을 수행하지만, 결과를 만들어 내지 않는 함수를 {\bf 빈 함수(void functions)}라고 부른다.

열매 함수를 호출할 때는 결과 값을 가지고 뭔가를 하려고 한다. 예를 들어, 결과값을 변수에 할당하거나, 표현식의 일부로 재사용할 수 있다.

\beforeverb
\begin{verbatim}
x = math.cos(radians)
golden = (math.sqrt(5) + 1) / 2
\end{verbatim}
\afterverb
%
인터랙티브 모드에서 함수를 호출할 때, 파이썬은 결과를 화면에 출력한다.

\beforeverb
\begin{verbatim}
>>> math.sqrt(5)
2.2360679774997898
\end{verbatim}
\afterverb
%
하지만, 스크립트에서 열매함수를 호출하고 변수에 결과값을 저장하지 않으면 반환되는 결과값은 안개속에 사라져간다!

\beforeverb
\begin{verbatim}
math.sqrt(5)
\end{verbatim}
\afterverb
%
이 스크립트는 제곱근 5의 값을 연산하지만, 변수에 결과값을 저장하거나, 화면에 출력하지 않아서 그 다지 유용하지는 않다.

\index{interactive mode}
\index{script mode}

빈 함수(Void functions)는 화면에 뭔가 출력하거나 뭔가 다른 효과를 가지지만, 반환값은 없다.
빈 함수를 사용하여 결과에 변수를 할당하면, {\tt None}으로 불리는 특별한 값을 얻게 된다.

\index{None special value}
\index{special value!None}

\beforeverb
\begin{verbatim}
>>> result = print_twice('Bing')
Bing
Bing
>>> print result
None
\end{verbatim}
\afterverb
%

{\tt None} 값은 자신만의 특별한 값을 가지며, 문자열 \verb"'None'" 과는 같지 않다. 

\beforeverb
\begin{verbatim}
>>> print type(None)
<type 'NoneType'>
\end{verbatim}
\afterverb
%

함수에서 결과를 반환하기 위해서, 함수내부에 {\tt return}문을 사용한다.
예를 들어, 두 숫자를 더해서 결과를 반환하는 {\tt addtwo}라는 간단한 함수를 작성할 수 있다. 

\beforeverb
\begin{verbatim}
def addtwo(a, b):
    added = a + b
    return added

x = addtwo(3, 5)
print x
\end{verbatim}
\afterverb
%

상기 스크립트가 실행될 때 print 문은 ``8''을 출력한다. 왜냐하면, 3과 5를 인수로 받는 {\tt addtwo} 함수가 호출되기 때문이다.
함수 내부에 매개 변수 {\tt a}, {\tt b}는 각각 3, 5이다.
 {\tt addtwo} 함수는 두 숫자의 덧셈을 수행하고  {\tt added}라는 로컬 변수에 저장하고, {\tt return}문을 사용해서 덧셈 결과를 반환하고,
  {\tt x} 라는 변수에 할당하여 출력한다.

\section{왜 함수를 사용하는가?}
\index{function, reasons for}

프로그램을 함수로 나누는 고생을 할 가치가 왜 있는지 불명확할지 모릅니다. 여기 몇 가지 이유가 있습니다.

\begin{itemize}

\item 명령문을 그룹으로 만들어 새로운 함수로 명명하는 것은 프로그램을 읽고, 이해하고, 디버그하기 좋게 합니다. 

\item 함수는 반복 코드를 제거해서 프로그램을 작고 콤팩트하게 만듭니다. 후에 프로그램에 수정사항이 생기면, 단지 한 곳에서만 수정을 하면 됩니다.

\item 긴 프로그램을 함수로 나누어 작성하는 것은 작은 부분에서 버그를 수정할 수 있게 하고 이를 조합해서 전체 온전한 프로그램을 만들 수 있습니다.

\item 잘 설계된 함수는 종종 많은 프로그램에 유용하게 사용됩니다. 잘 설계된 프로그램을 작성하고 디버그를 해서 오류가 없이 만들게 되면, 나중에 재사용이 용이합니다.

\end{itemize}

책의 나머지 부분에서 이 개념을 설명하는 함수 정의를 종종 사용할 것입니다. 함수를 만들고 사용하는 기술의 일부는 "리스트에서 가장 작은 값을 찾아내는 것"과 같은 
생각을 적절하게 추상화하여 함수를 작성하는 것입니다. 나중에, 리스트에서 가장 작은 값을 찾아내는 코드를 보여 줄 것입니다. 리스트를 인수로 받아 가장 작은 값을 
반환하는 {\tt min} 함수를 작성해서 여러분에게 보여드릴 것입니다.

\section{Debugging}
\label{editor}
\index{debugging}

If you are using a text editor to write your scripts, you might
run into problems with spaces and tabs.  The best way to avoid
these problems is to use spaces exclusively (no tabs).  Most text
editors that know about Python do this by default, but some
don't.

\index{whitespace}

Tabs and spaces are usually invisible, which makes them
hard to debug, so try to find an editor that manages indentation
for you.

Also, don't forget to save your program before you run it.  Some
development environments do this automatically, but some don't.
In that case the program you are looking at in the text editor
is not the same as the program you are running.

Debugging can take a long time if you keep running the same,
incorrect, program over and over!

Make sure that the code you are looking at is the code you are running.
If you're not sure, put something like \verb"print 'hello'" at the
beginning of the program and run it again.  If you don't see
\verb"hello", you're not running the right program!




\section{Glossary}

\begin{description}

\item[algorithm:]  A general process for solving a category of
problems.
\index{algorithm}

\item[argument:]  A value provided to a function when the function is called.
This value is assigned to the corresponding parameter in the function.
\index{argument}

\item[body:] The sequence of statements inside a function definition.
\index{body}

\item[composition:] Using an expression as part of a larger expression,
or a statement as part of a larger statement.
\index{composition}

\item[deterministic:] Pertaining to a program that does the same
thing each time it runs, given the same inputs.
\index{deterministic}

\item[dot notation:]  The syntax for calling a function in another
module by specifying the module name followed by a dot (period) and
the function name.
\index{dot notation}

\item[flow of execution:]  The order in which statements are executed during
a program run.
\index{flow of execution}

\item[fruitful function:] A function that returns a value.
\index{fruitful function}

\item[function:] A named sequence of statements that performs some
useful operation.  Functions may or may not take arguments and may or
may not produce a result.
\index{function}

\item[function call:] A statement that executes a function. It
consists of the function name followed by an argument list.
\index{function call}

\item[function definition:]  A statement that creates a new function,
specifying its name, parameters, and the statements it executes.
\index{function definition}

\item[function object:]  A value created by a function definition.
The name of the function is a variable that refers to a function
object.
\index{function definition}

\item[header:] The first line of a function definition.
\index{header}

\item[import statement:] A statement that reads a module file and creates
a module object.
\index{import statement}
\index{statement!import}

\item[module object:] A value created by an {\tt import} statement
that provides access to the data and code defined in a module.
\index{module}

\item[parameter:] A name used inside a function to refer to the value
passed as an argument.
\index{parameter}

\item[pseudorandom:] Pertaining to a sequence of numbers that appear
to be random, but are generated by a deterministic program.
\index{pseudorandom}

\item[return value:]  The result of a function.  If a function call
is used as an expression, the return value is the value of
the expression.
\index{return value}

\item[void function:] A function that doesn't return a value.
\index{void function}


\end{description}


\section{Exercises}

\begin{ex}
What is the purpose of the "def" keyword in Python?

a) It is slang that means "the following code is really cool"\\
b) It indicates the start of a function\\
c) It indicates that the following indented section of code is to be stored for later\\
d) b and c are both true\\
e) None of the above
\end{ex}

\begin{ex}
What will the following Python program print out?

\beforeverb
\begin{verbatim}
def fred():
   print "Zap"

def jane():
   print "ABC"

jane()
fred()
jane()
\end{verbatim}
\afterverb
%
a) Zap ABC jane fred jane\\
b) Zap ABC Zap\\
c) ABC Zap jane\\
d) ABC Zap ABC\\
e) Zap Zap Zap
\end{ex}

\begin{ex}
Rewrite your pay computation with time-and-a-half for overtime
and create a function called {\tt computepay} which takes
two parameters ({\tt hours} and {\tt rate}).

\begin{verbatim}
Enter Hours: 45
Enter Rate: 10
Pay: 475.0
\end{verbatim}
\end{ex}

\begin{ex}
Rewrite the grade program from the previous chapter 
using a function called {\tt computegrade} that takes
a score as its parameter and returns a grade as a string.

\begin{verbatim}
Score   Grade
> 0.9     A
> 0.8     B
> 0.7     C
> 0.6     D
<= 0.6    F

Program Execution:

Enter score: 0.95
A

Enter score: perfect
Bad score

Enter score: 10.0
Bad score

Enter score: 0.75
C

Enter score: 0.5
F
\end{verbatim}

Run the program repeatedly to test the various different values
for input.
\end{ex}


