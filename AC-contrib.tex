% The contents of this file is 
% Copyright (c) 2009-  Charles R. Severance, All Righs Reserved

\chapter{공헌(contribution)}
\section*{''정보교육을 위한 파이썬''에 공헌하신 분 목록}

Bruce Shields 초기 초안을 편집,
Sarah Hegge,
Steven Cherry,
Sarah Kathleen Barbarow,
Andrea Parker,
Radaphat Chongthammakun,
Megan Hixon,
Kirby Urner,
Sarah Kathleen Barbrow,
Katie Kujala,
Noah Botimer,
Emily Alinder,
Mark Thompson-Kular,
James Perry,
Eric Hofer,
Eytan Adar,
Peter Robinson,
Deborah J. Nelson,
Jonathan C. Anthony,
Eden Rassette,
Jeannette Schroeder,
Justin Feezell,
Chuanqi Li,
Gerald Gordinier,
Gavin Thomas Strassel,
Ryan Clement,
Alissa Talley,
Caitlin Holman,
Yong-Mi Kim,
Karen Stover,
Cherie Edmonds,
Maria Seiferle,
Romer Kristi D. Aranas (RK),
Grant Boyer,
Hedemarrie Dussan,

% CONTRIB

\section*{``Think Python'' 서문}

\subsection*{``Think Python'' 특이한 역사}

(Allen B. Downey)

1999년 1월 자바로 프로그램 기초 과목 수업을 준비하고 있었다.
세번에 걸쳐서 수업을 했지만 좌절하고 있었다.
학급에서 낙제비율이 무척이나 높았고, 정상적으로 이수한 학생의 전반적인 성취도가 무척 낮았다.

목격한 여러 문제점 중의 하는 책이다.
자바 책이 너무 방대하고 불필요하게 상세한 정보가 너무 많았고 어떻게 프로그램을 작성하는지에 대한
높은 수준의 지침은 부족했다.
학생들 모두 뚜껑문 효과(trapdoor effect)로 고생했다. 즉, 쉽게 시작하고, 점진적으로 나아가지고나서 5장 주변에 
하위권 학생이 떨어져 나가는 것이다. 너무 많은 새로운 교재를 너무 빨리 학습해야했고,
저자는 학기의 나머지를 수습하는데 사용했다.

첫 수업시작하기 2주일 전에 직접 책을 쓰기로 마음먹었다.
목표는 다음과 같다.

\begin{itemize}

\item 짧게 한다. 읽지 않는 50 페이지보다 읽히는 10 페이지가 더 낫다.

\item 용어에 주의한다. 전문용어를 최소하하고 첫번째 사용에 각 용어를 정의한다.

\item 점진적으로 구축해 나간다. 뚜껑문 효과를 피하기 위해서, 가장 어려운 주제를 잡고 일련의 작은 단계로 쪼갠다.

\item 프로그래밍 언어보다 프로그래밍에 집중한다. 최소한의 유용한 자바 부분을 포함하고 나머지는 생략한다.

\end{itemize}

제목이 필요했고, 즉흥적으로 \emph{어떻게 컴퓨터 과학자처럼 생각하기 (How to Think Like a Computer Scientist)}로 정했다.

첫 버젼은 엉성했지만 실질적으로 작동했다. 학생들은 읽고, 충분히 이해해서 수업시간을 어렵고, 흥미로운 주제에 좀더 시간을 쓸 수 있었고,
가장 중요한 것은 학생들이 실습을 하게된 것이다.

GNU 공개 문서 라이센스(GNU Free Documentation License)로 책을 배포해서 누구나 복사, 편집, 배포할 수 있게 했다.

\index{GNU Free Documentation License}
\index{Free Documentation License, GNU}

다음에 멋진 일이 생겼다. 버지니아 고등학교 교사인 Jeff Elkner 선생님이 책을 채용해서 파이썬으로 번역했다.
Jeff Elkner 선생님은 번역본을 보내왔고, 책을 읽고서 파이썬을 공부하게 되는 좀 특이한 경험을 했다.

Jeff와 저자는 책을 개작했고 Chris Meyers가 사례 연구를 추가했다. 2001년 GNU 공개 문서 라이센스로 
\emph{How to Think Like a Computer Scientist: Learning with Python} 제목으로 배포했다.
Green Tea Press 출판사에서 책을 출판해서 Amazon.com과 대학 서점을 통해서 제본된 책을 팔기 시작했다.
Green Tea Press 출판사의 다른 책들은 \url{greenteapress.com}에서 살펴볼 수 있다.

2003년 Olin College에서 수업을 시작해서 처음으로 파이썬을 가르치게 되었다.
자바와 비교하여 놀라웠다. 학생들이 덜 고생하고, 더 많이 배우고, 좀더 흥미로운 프로젝트를 수행하고, 전반적으로 훨씬 재미있게 되었다.

지난 5년 동안 책을 계속적으로 개발하고, 오류를 수정하고, 예제을 향상하고, 교재, 특히 연습문제를 추가했다.
2008년 대대적인 수정 작업을 시작했는데 동시에 차기 개정판에 관심을 보인 Cambridge University Press 편집자와 계약했다. 좋은 시점이다.

이 책을 즐기고, 컴퓨터 과학자처럼 적어도 약간은 프로그램하고 생각하는 것을 배우는데 도움이 되기를 바랍니다.

\subsection*{``Think Python'' 감사의 글}

(Allen B. Downey)
처음으로 가장 중요하게, Jeff Elkner에게 감사드린다. 자바 책을 파이썬으로 번역해서
이 프로젝트가 시작되게 했고 가장 좋아하는 언어가 된 파이썬을 소개해 주었다.

Chris Meyers에게도 감사의 말씀을 드린다. \emph{How to Think Like a Computer Scientist} 책의 몇개 부분에 기여해 주셨다.

Jeff와 Chris와 협동작업을 할수 있게 만든 GNU 공개 문서 라이센스를 개발한 
자유 소프트웨어 재단(Free Software Foundation)에 감사한다.

\index{GNU Free Documentation License}
\index{Free Documentation License, GNU}

또한, \emph{How to Think Like a Computer Scientist} 책을 작업하신 Lulu의 편집자에게 감사한다.

책의 초기 버젼을 함께 작업한 모든 학생들과 수정과 제안을 보내주신 부론에 나온 모든 기여자에게 감사한다.

그리고, 집사람 Lisa, Green Tea Press 그리고 다른 모든 것에도 감사한다.

Allen B. Downey \\
Needham MA\\

Allen Downey 
컴퓨터 과학 부교수
Franklin W. Olin College of Engineering

\section*{``Think Python'' 공헌자 목록}

\index{contributors}

(Allen B. Downey)

날카로운 눈과 사려깊은 100명 이상의 독자가 지난 몇년동안 수정사항과 제안을 보내주었다.
이 프로젝트의 공헌과 열정은 매우 큰 도움이었다.

이들 참여자로부터의 각각의 공헌의 본질에 대한 자세한 사항은 ``Think Python'' 본문에서 확인하세요.

Lloyd Hugh Allen,
Yvon Boulianne,
Fred Bremmer,
Jonah Cohen,
Michael Conlon,
Benoit Girard,
Courtney Gleason and Katherine Smith,
Lee Harr,
James Kaylin,
David Kershaw,
Eddie Lam,
Man-Yong Lee,
David Mayo,
Chris McAloon,
Matthew J. Moelter,
Simon Dicon Montford,
John Ouzts,
Kevin Parks,
David Pool,
Michael Schmitt,
Robin Shaw,
Paul Sleigh,
Craig T. Snydal,
Ian Thomas,
Keith Verheyden,
Peter Winstanley,
Chris Wrobel,
Moshe Zadka,
Christoph Zwerschke,
James Mayer,
Hayden McAfee,
Angel Arnal,
Tauhidul Hoque and Lex Berezhny,
Dr. Michele Alzetta,
Andy Mitchell,
Kalin Harvey,
Christopher P. Smith,
David Hutchins,
Gregor Lingl,
Julie Peters,
Florin Oprina,
D.~J.~Webre,
Ken,
Ivo Wever,
Curtis Yanko,
Ben Logan,
Jason Armstrong,
Louis Cordier,
Brian Cain,
Rob Black,
Jean-Philippe Rey at Ecole Centrale Paris,
Jason Mader at George Washington University made a number
Jan Gundtofte-Bruun,
Abel David and Alexis Dinno,
Charles Thayer,
Roger Sperberg,
Sam Bull,
Andrew Cheung,
C. Corey Capel,
Alessandra,
Wim Champagne,
Douglas Wright,
Jared Spindor,
Lin Peiheng,
Ray Hagtvedt,
Torsten H\"{u}bsch,
Inga Petuhhov,
Arne Babenhauserheide,
Mark E. Casida,
Scott Tyler,
Gordon Shephard,
Andrew Turner,
Adam Hobart,
Daryl Hammond and Sarah Zimmerman,
George Sass,
Brian Bingham,
Leah Engelbert-Fenton,
Joe Funke,
Chao-chao Chen,
Jeff Paine,
Lubos Pintes,
Gregg Lind and Abigail Heithoff,
Max Hailperin,
Chotipat Pornavalai,
Stanislaw Antol,
Eric Pashman,
Miguel Azevedo,
Jianhua Liu,
Nick King,
Martin Zuther,
Adam Zimmerman,
Ratnakar Tiwari,
Anurag Goel,
Kelli Kratzer,
Mark Griffiths,
Roydan Ongie,
Patryk Wolowiec,
Mark Chonofsky,
Russell Coleman,
Wei Huang,
Karen Barber,
Nam Nguyen,
St\'{e}phane Morin,
and
Paul Stoop.

