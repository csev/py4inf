% LaTeX source for ``Python for Informatics: Exploring Information''
% Copyright (c)  2009  Charles R. Severance

% Permission is granted to copy, distribute and/or modify this
% document under the terms of the GNU Free Documentation License,
% Version 1.1  or any later version published by the Free Software
% Foundation; with no Invariant Sections, no Front-Cover Texts,
% and no Back-Cover Texts.

% This distribution includes a file named fdl.tex that contains the text
% of the GNU Free Documentation License.  If it is missing, you can obtain
% it from www.gnu.org or by writing to the Free Software Foundation,
% Inc., 59 Temple Place - Suite 330, Boston, MA 02111-1307, USA.
%

%\documentclass[10pt,b5paper]{book}
\documentclass[10pt]{book}
\usepackage[width=5.5in,height=8.5in,
  hmarginratio=3:2,vmarginratio=1:1]{geometry}

\usepackage{pslatex}
\usepackage{url}
\usepackage{fancyhdr}
\usepackage{graphicx}
\usepackage{amsmath, amsthm, amssymb}
\usepackage{exercise}
\usepackage{makeidx}
\usepackage{setspace}
\usepackage{hevea}
\usepackage{upquote}

\newcommand{\thetitle}{Python for Informatics: Exploring Information}
\newcommand{\theversion}{0.0.1}

\makeindex

\begin{document}

\frontmatter

% LATEXONLY

\input{latexonly}

\newtheorem{ex}{Exercise}[chapter]

\begin{latexonly}

\renewcommand{\blankpage}{\thispagestyle{empty} \quad \newpage}

%\blankpage
%\blankpage

% TITLE PAGES FOR LATEX VERSION

%-half title--------------------------------------------------
\thispagestyle{empty}

\begin{flushright}
\vspace*{2.0in}

\begin{spacing}{3}
{\huge Python for Informatics}\\
{\Large Exploring Information}\\
\end{spacing}

\vspace{0.25in}

Version \theversion

\vfill

\end{flushright}

%--verso------------------------------------------------------

\blankpage
\blankpage
%\clearemptydoublepage
%\pagebreak
%\thispagestyle{empty}
%\vspace*{6in}

%--title page--------------------------------------------------
\pagebreak
\thispagestyle{empty}

\begin{flushright}
\vspace*{2.0in}

\begin{spacing}{3}
{\huge Python for Informatics}\\
{\Large Exploring Information}
\end{spacing}

\vspace{0.25in}

Version \theversion

\vspace{1in}


{\Large
Charles Severance\\
Allen B. Downey\\
}


% \vspace{0.5in}

% {\Large Green Tea Press}

% {\small Needham, Massachusetts}

%\includegraphics[width=1in]{figs/logo1.eps}
\vfill

\end{flushright}


%--copyright--------------------------------------------------
\pagebreak
\thispagestyle{empty}

{\small
Copyright \copyright ~2009 Charles Severance.


Printing history:

\begin{description}

\item[December 2009:] Begine the work to remix 
{\em Think Python: How to Think Like
a Computer Scientist}
to produce 
{\em Python for Informatics: Exploring Information}

\item[June 2008:] Major revision, changed title to
{\em Think Python: How to Think Like
a Computer Scientist}.

\item[August 2007:] Major revision, changed title to
{\em How to Think Like a (Python) Programmer}.

\item[April 2002:] First edition of {\em How to Think Like
a Computer Scientist}.

\end{description}

\vspace{0.2in}

% \begin{flushleft}
% Green Tea Press       \\
% 9 Washburn Ave \\
% Needham MA 02492
% \end{flushleft}

Permission is granted to copy, distribute, and/or modify this document
under the terms of the GNU Free Documentation License, Version 1.1 or
any later version published by the Free Software Foundation; with no
Invariant Sections, no Front-Cover Texts, and with no Back-Cover Texts.

The GNU Free Documentation License is available from {\tt www.gnu.org}
or by writing to the Free Software Foundation, Inc., 59 Temple Place,
Suite 330, Boston, MA 02111-1307, USA.

The original form of this book is \LaTeX\ source code.  Compiling this
\LaTeX\ source has the effect of generating a device-independent
representation of a textbook, which can be converted to other formats
and printed.

The \LaTeX\ source for the 
{\em Think Python} version of this book is available from
\url{http://www.thinkpython.com}.
The \LaTeX\ source for the 
{\em Python for Informatics}
version of the book is available (for the moment) from 
\url{https://source.sakaiproject.org/contrib//csev/trunk/pyinf/}.

\vspace{0.2in}

} % end small

\end{latexonly}


% HTMLONLY

\begin{htmlonly}

% TITLE PAGE FOR HTML VERSION

{\Large \thetitle}

{\large 
Charles Severance\\
Allen B. Downey}

Version \theversion

\setcounter{chapter}{-1}

\end{htmlonly}

\chapter{Preface}

\section*{Remixing an Open Book}

It is quite natural for academics who are continuously told to 
"publish or perish" to want to always create something from scratch
that is their own contribution.   This book is an for me 
experiment in "re-mixing"
an book titled
{\em Think Python: How to Think Like
a Computer Scientist}
written by Allen B. Downey, Jeff Elkner and others.

In Decmber of 2009, I was preparing to teach
SI502 - Networked Programming at the University of Michigan
for the fifth semester in a row and decided it was time
to write a Python textbook that focused on exploring data
rather than a first programming course for Computer Scientists.
My goal in SI502 was to teach people life-long data handling 
skills using Python.  Few of my
students would ever be professional computer programmers - instead, they
would be librarians, managers, lawyers, biologists, economists, etc.
In short, they represented every field outside of Computer Science.

I never seemed to find the perfect book for my course so I set out 
to write just such a book.  Luckily at a faculty meeting three weeks
before I was about to start my new book from scratch, 
Dr. Atul Prakash showed me the {\em Think Python} book which he had
been using to teach his Python courses.

{\em Think Python} did follow the structure of most introductory Computer
Science textbooks.  It is a well-witten text with a focus on 
simple explanations and ease of learning.  Jeff Elkner is a High
School teacher and so the language in the book is understandable
at a High School level.

The book was available under the 
GNU Free Documentation License 
so I had all the permission I needed
to use the book as the base material for my book.  As a courtesy, 
I sent a note to Allen and Cambridge Press letting them know what I 
was about to do to their book.

I expect that by the time I am done with {\em Python for Informatics} 
over fifty percent of the book will be new.  The overall structure
will be changed to get to doing data analysis problems as quickly as
possible and have a series of running examples about data analysis.
Then I will add sections on regular expressions, data
vizualization, working with spreadsheet data, structured 
query lauguage using SQLIte3, web scraping, and calling REST-based
Application Program Interfaces using Twitter.

What is intereresting is how much of the original {\em Think Python} book
material is completely relevant to this book and how much will fit
right into {\em Python for Informatics} with virtually no change.

By starting with the {\em Think Python} book, I won't have to write the
basic descriptions of the Python language or how to debug programs
and instead focus on the topical material that is the value-add
of {\em Python for Informatics}.

I hope that this book serves an example of why open 
materials are so important to the future of education,
and want to publically thank Allen B. Downey and Cambridge University
Press for their forward looking decision to make the book available
under an open Copyright.   I hope they are pleased with the 
results of my efforts and I hope that you the reader are pleased with
{\em our} collective efforts.

Charles Severance\\
December 19, 2009

Charles Severance is a 
Clinical Assistant Professor 
at the University of Michigan School of Information.

\section*{Preface for ``Think Python''}

\subsection*{The strange history of ``Think Python''}

(Allen B. Downey)

In January 1999 I was preparing to teach an introductory programming
class in Java.  I had taught it three times and I was getting
frustrated.  The failure rate in the class was too high and, even for
students who succeeded, the overall level of achievement was too low.

One of the problems I saw was the books.  
They were too big, with too much unnecessary detail about Java, and
not enough high-level guidance about how to program.  And they all
suffered from the trap door effect: they would start out easy,
proceed gradually, and then somewhere around Chapter 5 the bottom would
fall out.  The students would get too much new material, too fast,
and I would spend the rest of the semester picking up the pieces.

Two weeks before the first day of classes, I decided to write my
own book.  
My goals were:

\begin{itemize}

\item Keep it short.  It is better for students to read 10 pages
than not read 50 pages.

\item Be careful with vocabulary.  I tried to minimize the jargon
and define each term at first use.

\item Build gradually. To avoid trap doors, I took the most difficult
topics and split them into a series of small steps. 

\item Focus on programming, not the programming language.  I included
the minimum useful subset of Java and left out the rest.

\end{itemize}

I needed a title, so on a whim I chose {\em How to Think Like
a Computer Scientist}.

My first version was rough, but it worked.  Students did the reading,
and they understood enough that I could spend class time on the hard
topics, the interesting topics and (most important) letting the
students practice.

I released the book under the GNU Free Documentation License,
which allows users to copy, modify, and distribute the book.

\index{GNU Free Documentation License}
\index{Free Documentation License, GNU}

What happened next is the cool part.  Jeff Elkner, a high school
teacher in Virginia, adopted my book and translated it into
Python.  He sent me a copy of his translation, and I had the
unusual experience of learning Python by reading my own book.

Jeff and I revised the book, incorporated a case study by
Chris Meyers, and in 2001 we released {\em How to Think Like
a Computer Scientist: Learning with Python}, also under
the GNU Free Documentation License.
As Green Tea Press, I published the book and started selling
hard copies through Amazon.com and college book stores.
Other books from Green Tea Press are available at
\url{greenteapress.com}.

In 2003 I started teaching at Olin College and I got to teach
Python for the first time.  The contrast with Java was striking.
Students struggled less, learned more, worked on more interesting
projects, and generally had a lot more fun.

Over the last five years I have continued to develop the book,
correcting errors, improving some of the examples and
adding material, especially exercises.  In 2008 I started work
on a major revision---at the same time, I was
contacted by an editor at Cambridge University Press who
was interested in publishing the next edition.  Good timing!

I hope you enjoy working with this book, and that it helps
you learn to program and think, at least a little bit, like
a computer scientist.

\subsection*{Acknowledgements for ``Think Python''}

(Allen B. Downey)

First and most importantly, I thank Jeff Elkner, who
translated my Java book into Python, which got this project
started and introduced me to what has turned out to be my
favorite language.

I also thank Chris Meyers, who contributed several sections
to {\em How to Think Like a Computer Scientist}.

And I thank the Free Software Foundation for developing
the GNU Free Documentation License, which helped make
my collaboration with Jeff and Chris possible.

\index{GNU Free Documentation License}
\index{Free Documentation License, GNU}

I also thank the editors at Lulu who worked on
{\em How to Think Like a Computer Scientist}.

I thank all the students who worked with earlier
versions of this book and all the contributors (listed
below) who sent in corrections and suggestions.

And I thank my wife, Lisa, for her work on this book, and Green
Tea Press, and everything else, too.

Allen B. Downey \\
Needham MA\\

Allen Downey is an Associate Professor of Computer Science at 
the Franklin W. Olin College of Engineering.


\clearemptydoublepage

% TABLE OF CONTENTS
\begin{latexonly}

\tableofcontents

\clearemptydoublepage

\end{latexonly}

% START THE BOOK
\mainmatter

\chapter{Why should you learn to write programs?}

Writing programs (or programming) is a very creative 
and rewarding activity.  You can write programs for 
many reasons ranging from making your living to solving
a difficult data analysis problem to having fun to helping
someone else solve a problem.  This book assumes that 
{\em everyone} needs to know how to program and that once 
you know how to program, you will figure out what you want 
to do with your newfound skills.  

We are surrounded in our daily lives with computers ranging 
from laptops to cell phones.  We can think of these computers
as our ``personal assistants'' who can take care of many things
on our behalf.  The hardware in our current-day computers 
is essentially built to continuously as us the question, 
``What would you like me to do next?''.

{\large Picture of PDA asking "What Next?"}

Our computers are fast and have vasts amounts of memory and 
could be very helpful to us if we only knew the language to
speak to explain to the computer what we would like it to 
``do next''.  If we knew this language we could tell the 
computer to do tasks on our behalf that were reptitive.  
Interestingly, the kinds of things computers can do best
are often the kinds of things that we humans find boring
and mind-numbing.

For example, look at the first three paragraphs of this
chapter and tell me the most commonly used word and how
many times the word is used.  While you were able to read
and understand the words in a few seconds, counting them
is almost painful because is is not the kind of problem 
that human minds are designed to solve.  For a computer
the opposite is true, reading and understanding text 
from a piece of paper is hard for a computer to do 
but counting the words and telling you how many times
the most used word was used is very easy for the
computer:

\beforeverb
\begin{verbatim}
python words.py
Enter file:words.txt
to 16
\end{verbatim}
\afterverb
%
Our ``personal information analysis assistant'' quickly 
told us that the word ``to'' was used sixteen times in the
first three paragraphs of this chapter.

It is this very fact that computers are good at things 
that humans are not so good at is why you need to become
skilled at talking ``computer language''.  Once you learn
this new language, you can delegate mundane tasks
to your partner (the computer), leaving more time 
for you to do the 
things that you are uniquely suited for.  You bring 
creativity, intuition, and inventiveness to this
partnership.  

\section{Computer hardware architecture}
\index{hardware}
\index{hardware!architecture}

Before we start out learning the language we 
speak to give instructions to computers to 
develop software, we need to learn about more 
about how computers are built.  If you were to take
apart your computer or cell phone and look deep
inside, you would find the following parts:

{\large Picture of CPU, Memory, Etc. "What next?" in a thought bubble from the CPU}

The high-level definitions of these parts are as follows:

\begin{itemize}

\item The {\bf Central Processing Unit} (or CPU) is 
that part of the computer that is built to be obsessed 
with ``what is next?''.  If your computer is rated
at 3.0 Gigahertz, it means that the CPU will ask "What next?"
three billion times per second.  You are going to have to 
learn how to talk fast to keep up with the CPU.

\item The {\bf Main Memory} is used to store information
that the CPU needs in a hurry.  The main memory is nearly as 
fast as the CPU.  But the information stored in the main
memory vanishes when the computer is turned off.

\item The {\bf Secondary Memory} is also used to store
information, but it is much slower than the main memory.
The advantage of the secondary memory is that it can
store information even when there is no power to the
computer.  Examples of secondary memory are disk drives
or flash memory (typically found in USB sticks and portable
music players).

\item The {\bf Input and Output Devices} are simply our
screen, keyboard, mouse, microphone, speaker, touchpad, etc.  
They are all of the ways we interact with the computer.

\item These days, most computers also have a
{\bf Network Connection} to retrieve information over a network.
We can think of the network as a very slow place to store and
retrieve data that might now always be "up".  So in a sense,
the network is a slower and at times unreliable form of
{\bf Secondary Memory}

\end{itemize}

While most of the detail of how these components work is best left 
to computer builders, it helps to have a some terminology
so we can talk about these different parts as we write our programs.

As a programmer, your job is to use and orchestrate 
each of these resources to solve the problem that you need solving
and analyze the data you need analysed.  As a programmer you will 
mostly be "talking" to the CPU and telling it what to 
do next.  Sometimes you will tell the CPU to use the main memory,
secondary memory, network, or the input/output devices.

{\large Picture of CPU, Memory, Etc. With 
stick figure inside saying "do this" to the CPU}

You need to be the person who answers the CPU's "What next?" 
question.  But it would be very uncomfortable to shrink you 
down to 10cm tall and insert you into the computer just so you 
could issue a command three billion times per second.  So instead,
you must write down your instructions in advance.
We call these stored instructions a {\bf program} and the act 
of writing these instructions down and getting the instructions to 
be correct {\bf programming}.

\section{Understanding programming}

In the rest of this book, we will try to turn you into a person
who is skilled in the art of programming.  In the end you will be a 
{\bf programmer} - perhaps not a professional programmer - but 
at least you will have the skills to look at a data/information
analysis problem and develop a program to solve the problem.

\index{problem solving}

In a sense, you need two skills to be a programmer:

\begin{itemize}

\item First you need to know the programming language (Python) -
you need to know the vocabulary and the grammar.  You need to be able 
spell the words in this new language properly and how to construct 
well-formed ``sentances'' in this new languages.

\item Second you need to to "tell a story".  In writing a story,
you combine words and sentances to convey an idea to the reader. 
There is a skill and art in constructing the story and skill in
story writing is improved by doing some writing and getting some
feedback.  In programming, our program is the "story" and the 
problem you are trying to solve is the "idea".

\end{itemize}

Once you learn one programming language such as Python, you will 
find it much easier to learn a second programming language such
as JavaScript or C++.  The new programming language has very different 
vocabulary and grammar but once you learn problem solving skills, 
they will be the same across all programming languages.

You will learn the "vocabulary" and "sentances" of Python pretty quickly.
It will take longer for you to be able to write a coherent program
to solve a brand new problem.  We teach programming much like we teach
writing.  We start reading and explaining programs and then we write 
simple programs and then write increasingly complex programs over time.
At some point you "get your muse" and see the patterns on your own
and can see more naturally how to take a problem and 
write a program that solves that problem.  And once you get 
to that point, programming becomes a very plesant and creative process.  

We start with the vocabulary and structure of Python programs.  Be patient
as the simple examples remind you of when you started reading for the first
time. 

\section{The Python programming language}
\index{programming language}
\index{language!programming}

The programming language you will learn is Python. Python is
an example of a {\bf high-level language}; other high-level languages
you might have heard of are C, C++, Perl, Java, Ruby, and JavaScript. 
At times, we will include a few examples of the JavaScript language
to help cement the basic grammar ideas using 
two different ``vocabularies''.

There are
also {\bf low-level languages}, sometimes referred to as ``machine
languages'' or ``assembly languages.''  Loosely speaking, computers
can only execute programs written in low-level languages.  So
programs written in a high-level language have to be processed before
they can run.  This extra processing takes some time, which is a small
disadvantage of high-level languages.

\index{portability}
\index{high-level language}
\index{low-level language}
\index{language!high-level}
\index{language!low-level}

The advantages are enormous.  First, it is much easier to program
in a high-level language.  Programs written in a high-level language
take less time to write, they are shorter and easier to read, and they
are more likely to be correct.  Second, high-level languages are {\bf
portable}, meaning that they can run on different kinds of computers
with few or no modifications.  Low-level programs can run on only one
kind of computer and have to be rewritten to run on another.

Due to these advantages, almost all programs are written in high-level
languages.  Low-level languages are used only for a few specialized
applications.

\index{compile}
\index{interpret}

Two kinds of programs process high-level languages
into low-level languages: {\bf interpreters} and {\bf compilers}.
An interpreter reads a high-level program and executes it, meaning that it
does what the program says.  It processes the program a little at a time,
alternately reading lines and performing computations.

\beforefig
\centerline{\includegraphics[height=0.77in]{figs/interpret.eps}}
\afterfig

\index{source code}
\index{object code}
\index{executable}

A compiler reads the program and translates it completely before the
program starts running.  In this context, the high-level program is
called the {\bf source code}, and the translated program is called the
{\bf object code} or the {\bf executable}.  Once a program is
compiled, you can execute it repeatedly without further translation.

\beforefig
\centerline{\includegraphics[height=0.77in]{figs/compile.eps}}
\afterfig

Python is considered an interpreted language because Python programs
are executed by an interpreter.  There are two ways to use the
interpreter: {\bf interactive mode} and {\bf script mode}. In
interactive mode, you type Python programs and the interpreter prints
the result:

\index{interactive mode}
\index{script mode}

\beforeverb
\begin{verbatim}
>>> 1 + 1
2
>>>
\end{verbatim}
\afterverb
%
The chevron, {\tt >>>}, is the
{\bf prompt} the interpreter uses to indicate that it is ready.  If
you type {\tt 1 + 1}, the interpreter replies {\tt 2}.
The chevron is the Python interpreter's way of asking you, "What
do you want me to do next?".  You will notice that as soon as 
Python finishes one statement - it immediately is ready for you to type 
another statement.

\index{prompt}

Typing commands into the Python interpreter is a great way to experiment
with Python's features, but it is a bad way to type in many commands 
to solve a more complex problem.  When we want to write a program, 
we use a text editor to write the Python instructions into a file,
which is called a {\bf script}.  By
convention, Python scripts have names that end with {\tt .py}.

\index{script}

To execute the script, you have to tell the interpreter the name of
the file.  In a UNIX command window, you would type {\tt python
dinsdale.py}.  In other development environments, the details of
executing scripts are different.  You can find instructions for
your environment at the Python Website \url{python.org}.

\index{testing!interactive mode}

Working in interactive mode is convenient for testing small pieces of
code because you can type and execute them immediately.  But for
anything more than a few lines, you should save your code
as a script so you can modify and execute it in the future.


\section{What is a program?}

A {\bf program} is a sequence of instructions that specifies how to
perform a computation.  The computation might be something
mathematical, such as solving a system of equations or finding the
roots of a polynomial, but it can also be a symbolic computation, such
as searching and replacing text in a document or (strangely enough)
compiling a program.

\index{program}

The details look different in different languages, but a few basic
instructions appear in just about every language:

\begin{description}

\item[input:] Get data from the keyboard, a file, or some
other device.

\item[output:] Display data on the screen or send data to a
file or other device.

\item[math:] Perform basic mathematical operations like addition and
multiplication.

\item[conditional execution:] Check for certain conditions and
execute the appropriate sequence of statements.

\item[repetition:] Perform some action repeatedly, usually with
some variation.

\item[reuse:] Write a set of instructions once and give them a name
and then reuse those instructions as needed throuhgout your program.,

\end{description}

Believe it or not, that's pretty much all there is to it.  Every
program you've ever used, no matter how complicated, is made up of
instructions that look pretty much like these.  So you can think of
programming as the process of breaking a large, complex task
into smaller and smaller subtasks until the subtasks are
simple enough to be performed with one of these basic instructions.

\index{algorithm}

That may be a little vague, but we will come back to this topic
when we talk about {\bf algorithms}.

\section{What is debugging?}
\index{debugging}
\index{bug}

Programming is error-prone.  For whimsical reasons, programming errors
are called {\bf bugs} and the process of tracking them down is called
{\bf debugging}.

\index{debugging}
\index{bug}

Three kinds of errors can occur in a program: syntax errors, runtime 
errors, and semantic errors. It is useful
to distinguish between them in order to track them down more quickly.

\subsection{Syntax errors}
\index{syntax error}
\index{error!syntax}
\index{error message}

Python can only execute a program if the syntax is
correct; otherwise, the interpreter displays an error message.
{\bf Syntax} refers to the structure of a program and the rules about
that structure. \index{syntax} 
For example, parentheses have to come in matching pairs, so
{\tt (1 + 2)} is legal, but {\tt 8)} is a {\bf syntax error}.

\index{parentheses!matching}
\index{syntax}
\index{cummings, e. e.}

In English readers can tolerate most syntax errors, which is why we
can read the poetry of e. e. cummings without spewing error messages.
Python is not so forgiving.  If there is a single syntax error
anywhere in your program, Python will display an error message and quit,
and you will not be able to run your program. During the first few
weeks of your programming career, you will probably spend a lot of
time tracking down syntax errors.  As you gain experience, you will
make fewer errors and find them faster.

\subsection{Runtime errors}
\label{runtime}
\index{runtime error}
\index{error!runtime}
\index{exception}
\index{safe language}
\index{language!safe}

The second type of error is a runtime error, so called because the
error does not appear until after the program has started running.
These errors are also called {\bf exceptions} because they usually
indicate that something exceptional (and bad) has happened.

Runtime errors are rare in the simple programs you will see in the
first few chapters, so it might be a while before you encounter one.


\subsection{Semantic errors}
\index{semantics}
\index{semantic error}
\index{error!semantic}
\index{error message}

The third type of error is the {\bf semantic error}.  If there is a
semantic error in your program, it will run successfully in the sense
that the computer will not generate any error messages, but it will
not do the right thing.  It will do something else.  Specifically, it
will do what you told it to do.

The problem is that the program you wrote is not the program you
wanted to write.  The meaning of the program (its semantics) is wrong.
Identifying semantic errors can be tricky because it requires you to work
backward by looking at the output of the program and trying to figure
out what it is doing.

\subsection{Experimental debugging}

One of the most important skills you will acquire is debugging.
Although it can be frustrating, debugging is one of the most
intellectually rich, challenging, and interesting parts of
programming.

\index{experimental debugging}
\index{debugging!experimental}

In some ways, debugging is like detective work.  You are confronted
with clues, and you have to infer the processes and events that led
to the results you see.

Debugging is also like an experimental science.  Once you have an idea
about what is going wrong, you modify your program and try again.  If
your hypothesis was correct, then you can predict the result of the
modification, and you take a step closer to a working program.  If
your hypothesis was wrong, you have to come up with a new one.  As
Sherlock Holmes pointed out, ``When you have eliminated the
impossible, whatever remains, however improbable, must be the truth.''
(A. Conan Doyle, {\em The Sign of Four})

\index{Holmes, Sherlock}
\index{Doyle, Arthur Conan}

For some people, programming and debugging are the same thing.  That
is, programming is the process of gradually debugging a program until
it does what you want.  The idea is that you should start with a
program that does {\em something} and make small modifications,
debugging them as you go, so that you always have a working program.

For example, Linux is an operating system that contains thousands of
lines of code, but it started out as a simple program Linus Torvalds
used to explore the Intel 80386 chip.  According to Larry Greenfield,
``One of Linus's earlier projects was a program that would switch
between printing AAAA and BBBB.  This later evolved to Linux.''
({\em The Linux Users' Guide} Beta Version 1).

\index{Linux}

Later chapters will make more suggestions about debugging and other
programming practices.

\section{Building ``sentances'' in Python}
\index{formal language}
\index{natural language}
\index{language!formal}
\index{language!natural}

The rules (or grammar) of Python are simpler and more precise than 
the rules of a natural language that we use to speak and write.

{\bf Natural languages} are the languages people speak,
such as English, Spanish, and French.  They were not designed
by people (although people try to impose some order on them);
they evolved naturally.

{\bf Formal languages} are languages that are designed by people for
specific applications.  For example, the notation that mathematicians
use is a formal language that is particularly good at denoting
relationships among numbers and symbols.  Chemists use a formal
language to represent the chemical structure of molecules.  And
most importantly:

\begin{quote}
{\bf Programming languages are formal languages that have been
designed to express computations.}
\end{quote}

Formal languages tend to have strict rules about syntax.  For example,
$3 + 3 = 6$ is a syntactically correct mathematical statement, but 
$3 + = 3 \mbox{\$} 6$ is not.  $H_2O$ is a syntactically correct
chemical formula, but $_2Zz$ is not.

Syntax rules come in two flavors, pertaining to {\bf tokens} and
structure.  Tokens are the basic elements of the language, such as
words, numbers, and chemical elements.  One of the problems with $3 +
= 3 \mbox{\$} 6$ is that $\$$ is not a legal token in mathematics
(at least as far as I know).  Similarly, $_2Zz$ is not legal because
there is no element with the abbreviation $Zz$.

\index{token}
\index{structure}

The second type of syntax error pertains to the structure of a
statement; that is, the way the tokens are arranged.  The statement $3
+ = 3 \mbox{\$} 6$ is illegal because even though $+$ and $=$ are
legal tokens, you can't have one right after the other.  Similarly,
in a chemical formula the subscript comes after the element name, not
before.

\begin{ex}
Write a well-structured English
sentence with invalid tokens in it.  Then write another sentence
with all valid tokens but with invalid structure.
\end{ex}

When you read a sentence in English or a statement in a formal
language, you have to figure out what the structure of the sentence is
(although in a natural language you do this subconsciously).  This
process is called {\bf parsing}.

\index{parse}

For example, when you hear the sentence, ``The penny dropped,'' you
understand that ``the penny'' is the subject and ``dropped'' is the
predicate.  Once you have parsed a sentence, you can figure out what it
means, or the semantics of the sentence.  Assuming that you know
what a penny is and what it means to drop, you will understand the
general implication of this sentence.

Although formal and natural languages have many features in
common---tokens, structure, syntax, and semantics---there are some
differences:

\index{ambiguity}
\index{redundancy}
\index{literalness}

\begin{description}

\item[ambiguity:] Natural languages are full of ambiguity, which
people deal with by using contextual clues and other information.
Formal languages are designed to be nearly or completely unambiguous,
which means that any statement has exactly one meaning,
regardless of context.

\item[redundancy:] In order to make up for ambiguity and reduce
misunderstandings, natural languages employ lots of
redundancy.  As a result, they are often verbose.  Formal languages
are less redundant and more concise.

\item[literalness:] Natural languages are full of idiom and metaphor.
If I say, ``The penny dropped,'' there is probably no penny and
nothing dropping\footnote{This idiom means that someone realized something
after a period of confusion.}.  Formal languages
mean exactly what they say.

\end{description}

People who grow up speaking a natural language---everyone---often have a
hard time adjusting to formal languages.  In some ways, the difference
between formal and natural language is like the difference between
poetry and prose, but more so:

\index{poetry}
\index{prose}

\begin{description}

\item[Poetry:] Words are used for their sounds as well as for
their meaning, and the whole poem together creates an effect or
emotional response.  Ambiguity is not only common but often
deliberate.

\item[Prose:] The literal meaning of words is more important,
and the structure contributes more meaning.  Prose is more amenable to
analysis than poetry but still often ambiguous.

\item[Programs:] The meaning of a computer program is unambiguous
and literal, and can be understood entirely by analysis of the
tokens and structure.

\end{description}

Here are some suggestions for reading programs (and other formal
languages).  First, remember that formal languages are much more dense
than natural languages, so it takes longer to read them.  Also, the
structure is very important, so it is usually not a good idea to read
from top to bottom, left to right.  Instead, learn to parse the
program in your head, identifying the tokens and interpreting the
structure.  Finally, the details matter.  Small errors in
spelling and punctuation, which you can get away
with in natural languages, can make a big difference in a formal
language.

\section{The first program}
\label{hello}

\index{Hello, World}

Traditionally, the first program you write in a new language
is called ``Hello, World!'' because all it does is display the
words, ``Hello, World!''  In Python, it looks like this:

\beforeverb
\begin{verbatim}
print 'Hello, World!'
\end{verbatim}
\afterverb
%
This is an example of a {\bf print statement}\footnote{In Python 3.0,
  {\tt print} is a function, not a statement, so the syntax is {\tt
    print('Hello, World!')}.  We will get to functions soon!}, which
doesn't actually print anything on paper.  It displays a value on the
screen.  In this case, the result is the words

\index{Python 3.0}

\beforeverb
\begin{verbatim}
Hello, World!
\end{verbatim}
\afterverb
%
The quotation marks in the program mark the beginning and end
of the text to be displayed; they don't appear in the result.

\index{quotation mark}
\index{print statement}
\index{statement!print}

Some people judge the quality of a programming language by the
simplicity of the ``Hello, World!'' program.  By this standard, Python
does about as well as possible.


\section{Debugging}
\index{debugging}

It is a good idea to read this book in front of a computer so you can
try out the examples as you go.  You can run most of the examples in
interactive mode, but if you put the code into a script, it is easier
to try out variations.

Whenever you are experimenting with a new feature, you should try
to make mistakes.  For example, in the ``Hello, world!'' program,
what happens if you leave out one of the quotation marks?  What
if you leave out both?  What if you spell {\tt print} wrong?

\index{error message}

This kind of experiment helps you remember what you read; it also helps
with debugging, because you get to know what the error messages mean.
It is better to make mistakes now and on purpose than later
and accidentally.

Programming, and especially debugging, sometimes brings out strong
emotions.  If you are struggling with a difficult bug, you might 
feel angry, despondent or embarrassed.

There is evidence that people naturally respond to computers as if
they were people\footnote{See Reeves and Nass, {\it The Media
    Equation: How People Treat Computers, Television, and New Media
    Like Real People and Places}.}.  When they work well, we think
of them as teammates, and when they are obstinate or rude, we
respond to them the same way we respond to rude,
obstinate people.

\index{debugging!emotional response}
\index{emotional debugging}

Preparing for these reactions might help you deal with them.
One approach is to think of the computer as an employee with
certain strengths, like speed and precision, and
particular weaknesses, like lack of empathy and inability
to grasp the big picture.

Your job is to be a good manager: find ways to take advantage
of the strengths and mitigate the weaknesses.  And find ways
to use your emotions to engage with the problem,
without letting your reactions interfere with your ability
to work effectively.

Learning to debug can be frustrating, but it is a valuable skill
that is useful for many activities beyond programming.  At the
end of each chapter there is a debugging section, like this one,
with my thoughts about debugging.  I hope they help!


\section{Glossary}

\begin{description}

\item[central processing unit:] The heart of any computer.  It is what
runs the software that we write.

\item[main memory:] Stores programs and data.  Main memory loses 
its information when the power is tunred off.

\item[secondary memory:] Stores programs and data and retains its 
information even when the power is turned off.  Generally slower 
than main memory.  Examples of secondary memory include disk 
drives and flash member in USB sticks.

\item[problem solving:]  The process of formulating a problem, finding
a solution, and expressing the solution.
\index{problem solving}

\item[high-level language:]  A programming language like Python that
is designed to be easy for humans to read and write.
\index{high-level language}

\item[low-level language:]  A programming language that is designed
to be easy for a computer to execute; also called ``machine language'' or
``assembly language.''
\index{low-level language}

\item[portability:]  A property of a program that can run on more
than one kind of computer.
\index{portability}

\item[interpret:]  To execute a program in a high-level language
by translating it one line at a time.
\index{interpret}

\item[compile:]  To translate a program written in a high-level language
into a low-level language all at once, in preparation for later
execution.
\index{compile}

\item[source code:]  A program in a high-level language before
being compiled.
\index{source code}

\item[object code:]  The output of the compiler after it translates
the program.
\index{object code}

\item[executable:]  Another name for object code that is ready
to be executed.
\index{executable}

\item[prompt:] Characters displayed by the interpreter to indicate
that it is ready to take input from the user.
\index{prompt}

\item[script:] A program stored in a file (usually one that will be
interpreted).
\index{script}

\item[interactive mode:] A way of using the Python interpreter by
typing commands and expressions at the prompt.
\index{interactive mode}

\item[script mode:] A way of using the Python interpreter to read
and execute statements in a script.
\index{script mode}

\item[program:] A set of instructions that specifies a computation.
\index{program}

\item[algorithm:]  A general process for solving a category of
problems.
\index{algorithm}

\item[bug:]  An error in a program.
\index{bug}

\item[debugging:]  The process of finding and removing any of the
three kinds of programming errors.
\index{debugging}

\item[syntax:]  The structure of a program.
\index{syntax}

\item[syntax error:]  An error in a program that makes it impossible
to parse (and therefore impossible to interpret).
\index{syntax error}

\item[exception:]  An error that is detected while the program is running.
\index{exception}

\item[semantics:]  The meaning of a program.
\index{semantics}

\item[semantic error:]   An error in a program that makes it do something
other than what the programmer intended.
\index{semantic error}

\item[natural language:]  Any one of the languages that people speak that
evolved naturally.
\index{natural language}

\item[formal language:]  Any one of the languages that people have designed
for specific purposes, such as representing mathematical ideas or
computer programs; all programming languages are formal languages.
\index{formal language}

\item[token:]  One of the basic elements of the syntactic structure of
a program, analogous to a word in a natural language.
\index{token}

\item[parse:]  To examine a program and analyze the syntactic structure.
\index{parse}

\item[print statement:]  An instruction that causes the Python
interpreter to display a value on the screen.
\index{print statement}
\index{statement!print}


\end{description}


\section{Exercises}

\begin{ex}
Use a web browser to go to the Python Website \url{python.org}.
This page contains information about Python and links
to Python-related pages, and it gives you the ability to search
the Python documentation.

For example, if you enter {\tt print} in the search window, the
first link that appears is the documentation of the {\tt print}
statement.  At this point, not all of it will make sense to you,
but it is good to know where it is.

\index{documentation}
\index{python.org}
\end{ex}

\begin{ex}
Start the Python interpreter and type {\tt help()} to start the online
help utility.  Or you can type \verb"help('print')" to get information
about the {\tt print} statement.

If this example doesn't work, you
may need to install additional Python documentation or set an
environment variable; the details depend on your operating system and
version of Python.

\index{help utility}
\end{ex}

\begin{ex}
Start the Python interpreter and use it as a calculator.
Python's syntax for math operations is almost the same as
standard mathematical notation.  For example, the symbols
{\tt +}, {\tt -} and {\tt /} denote addition, subtraction
and division, as you would expect.  The symbol for
multiplication is {\tt *}.

If you run a 10 kilometer race in 43 minutes 30 seconds, what is your
average time per mile?  What is your average speed in miles per hour?
(Hint: there are 1.61 kilometers in a mile).

\index{calculator}
\index{running pace}

\end{ex}

% XXXX


\chapter{Variables, expressions and statements}

\section{Values and types}
\index{value}
\index{type}
\index{string}

A {\bf value} is one of the basic things a program works with,
like a letter or a
number.  The values we have seen so far
are {\tt 1}, {\tt 2}, and
\verb"'Hello, World!'".

These values belong to different {\bf types}:
{\tt 2} is an integer, and \verb"'Hello, World!'" is a {\bf string},
so-called because it contains a ``string'' of letters.
You (and the interpreter) can identify
strings because they are enclosed in quotation marks.

\index{quotation mark}

The print statement also works for integers.

\beforeverb
\begin{verbatim}
>>> print 4
4
\end{verbatim}
\afterverb
%
If you are not sure what type a value has, the interpreter can tell you.

\beforeverb
\begin{verbatim}
>>> type('Hello, World!')
<type 'str'>
>>> type(17)
<type 'int'>
\end{verbatim}
\afterverb
%
Not surprisingly, strings belong to the type {\tt str} and
integers belong to the type {\tt int}.  Less obviously, numbers
with a decimal point belong to a type called {\tt float},
because these numbers are represented in a
format called {\bf floating-point}.

\index{type}
\index{string type}
\index{type!str}
\index{int type}
\index{type!int}
\index{float type}
\index{type!float}

\beforeverb
\begin{verbatim}
>>> type(3.2)
<type 'float'>
\end{verbatim}
\afterverb
%
What about values like \verb"'17'" and \verb"'3.2'"?
They look like numbers, but they are in quotation marks like
strings.

\index{quotation mark}

\beforeverb
\begin{verbatim}
>>> type('17')
<type 'str'>
>>> type('3.2')
<type 'str'>
\end{verbatim}
\afterverb
%
They're strings.

When you type a large integer, you might be tempted to use commas
between groups of three digits, as in {\tt 1,000,000}.  This is not a
legal integer in Python, but it is legal:

\beforeverb
\begin{verbatim}
>>> print 1,000,000
1 0 0
\end{verbatim}
\afterverb
%
Well, that's not what we expected at all!  Python interprets {\tt
  1,000,000} as a comma-separated sequence of integers, which it
prints with spaces between.

\index{semantic error}
\index{error!semantic}
\index{error message}

This is the first example we have seen of a semantic error: the code
runs without producing an error message, but it doesn't do the
``right'' thing.


\section{Variables}
\index{variable}
\index{assignment statement}
\index{statement!assignment}

One of the most powerful features of a programming language is the
ability to manipulate {\bf variables}.  A variable is a name that
refers to a value.

An {\bf assignment statement} creates new variables and gives
them values:

\beforeverb
\begin{verbatim}
>>> message = 'And now for something completely different'
>>> n = 17
>>> pi = 3.1415926535897931
\end{verbatim}
\afterverb
%
This example makes three assignments.  The first assigns a string
to a new variable named {\tt message};
the second gives the integer {\tt 17} to {\tt n}; the third
assigns the (approximate) value of $\pi$ to {\tt pi}.

\index{state diagram}
\index{diagram!state}

A common way to represent variables on paper is to write the name with
an arrow pointing to the variable's value.  This kind of figure is
called a {\bf state diagram} because it shows what state each of the
variables is in (think of it as the variable's state of mind).
This diagram shows the result of the previous example:

\beforefig
\centerline{\includegraphics{figs/state2.eps}}
\afterfig

To display the value of a variable, you can use a print statement:

\beforeverb
\begin{verbatim}
>>> print n
17
>>> print pi
3.14159265359
\end{verbatim}
\afterverb
%
The type of a variable is the type of the value it refers to.

\beforeverb
\begin{verbatim}
>>> type(message)
<type 'str'>
>>> type(n)
<type 'int'>
>>> type(pi)
<type 'float'>
\end{verbatim}
\afterverb
%
\begin{ex}
If you type an integer with a leading zero, you might get
a confusing error:

\beforeverb
\begin{verbatim}
>>> zipcode = 02492
                  ^
SyntaxError: invalid token
\end{verbatim}
\afterverb

Other numbers seem to work, but the results are bizarre:

\beforeverb
\begin{verbatim}
>>> zipcode = 02132
>>> print zipcode
1114
\end{verbatim}
\afterverb

Can you figure out what is going on?  Hint: print the
values {\tt 01}, {\tt 010}, {\tt 0100} and {\tt 01000}.

\index{octal}

\end{ex}




\section{Variable names and keywords}
\index{keyword}

Programmers generally choose names for their variables that
are meaningful---they document what the variable is used for.

Variable names can be arbitrarily long.  They can contain
both letters and numbers, but they have to begin with a letter.
It is legal to use uppercase letters, but it is a good idea
to begin variable names with a lowercase letter (you'll
see why later).

The underscore character (\verb"_") can appear in a name.
It is often used in names with multiple words, such as
\verb"my_name" or \verb"airspeed_of_unladen_swallow".

\index{underscore character}

If you give a variable an illegal name, you get a syntax error:

\beforeverb
\begin{verbatim}
>>> 76trombones = 'big parade'
SyntaxError: invalid syntax
>>> more@ = 1000000
SyntaxError: invalid syntax
>>> class = 'Advanced Theoretical Zymurgy'
SyntaxError: invalid syntax
\end{verbatim}
\afterverb
%
{\tt 76trombones} is illegal because it does not begin with a letter.
{\tt more@} is illegal because it contains an illegal character, {\tt
@}.  But what's wrong with {\tt class}?

It turns out that {\tt class} is one of Python's {\bf keywords}.  The
interpreter uses keywords to recognize the structure of the program,
and they cannot be used as variable names.

\index{keyword}

Python has 31 keywords\footnote{In Python 3.0, {\tt exec} is no
longer a keyword, but {\tt nonlocal} is.}:

\beforeverb
\begin{verbatim}
and       del       from      not       while    
as        elif      global    or        with     
assert    else      if        pass      yield    
break     except    import    print              
class     exec      in        raise              
continue  finally   is        return             
def       for       lambda    try
\end{verbatim}
\afterverb
%
You might want to keep this list handy.  If the interpreter complains
about one of your variable names and you don't know why, see if it
is on this list.


\section{Statements}

A statement is a unit of code that the Python interpreter can
execute.  We have seen two kinds of statements: print
and assignment.

\index{statement}
\index{interactive mode}
\index{script mode}

When you type a statement in interactive mode, the interpreter
executes it and displays the result, if there is one.

A script usually contains a sequence of statements.  If there
is more than one statement, the results appear one at a time
as the statements execute.

For example, the script

\beforeverb
\begin{verbatim}
print 1
x = 2
print x
\end{verbatim}
\afterverb
%
produces the output

\beforeverb
\begin{verbatim}
1
2
\end{verbatim}
\afterverb
%
The assignment statement produces no output.


\section{Operators and operands}
\index{operator, arithmetic}
\index{arithmetic operator}
\index{operand}
\index{expression}

{\bf Operators} are special symbols that represent computations like
addition and multiplication.  The values the operator is applied to
are called {\bf operands}.

The operators {\tt +}, {\tt -}, {\tt *}, {\tt /} and {\tt **}
perform addition, subtraction, multiplication, division and
exponentiation, as in the following examples:

\beforeverb
\begin{verbatim}
20+32   hour-1   hour*60+minute   minute/60   5**2   (5+9)*(15-7)
\end{verbatim}
\afterverb
%
In some other languages, \verb"^" is used for exponentiation, but
in Python it is a bitwise operator called XOR.  I won't cover
bitwise operators in this book, but you can read about
them at \url{wiki.python.org/moin/BitwiseOperators}.

\index{bitwise operator}
\index{operator!bitwise}

%When a variable name appears in the place of an operand, it
%is replaced with its value before the operation is
%performed.

The division operator might not do what you expect:

\beforeverb
\begin{verbatim}
>>> minute = 59
>>> minute/60
0
\end{verbatim}
\afterverb
%
The value of {\tt minute} is 59, and in conventional arithmetic 59
divided by 60 is 0.98333, not 0.  The reason for the discrepancy is
that Python is performing {\bf floor division}\footnote{In Python 3.0,
the result of this division is a {\tt float}.  The new operator
{\tt //} performs integer division.}.

\index{Python 3.0}
\index{floor division}
\index{floating-point division}
\index{division!floor}
\index{division!floating-point}

When both of the operands are integers, the result is also an
integer; floor division chops off the fraction
part, so in this example it rounds down to zero.

If either of the operands is a floating-point number, Python performs
floating-point division, and the result is a {\tt float}:

\beforeverb
\begin{verbatim}
>>> minute/60.0
0.98333333333333328
\end{verbatim}
\afterverb


\section{Expressions}

An {\bf expression} is a combination of values, variables, and operators.
A value all by itself is considered an expression, and so is
a variable, so the following are all legal expressions
(assuming that the variable {\tt x} has been assigned a value):

\index{expression}
\index{evaluate}

\beforeverb
\begin{verbatim}
17
x
x + 17
\end{verbatim}
\afterverb
%
If you type an expression in interactive mode, the interpreter
{\bf evaluates} it and displays the result:

\beforeverb
\begin{verbatim}
>>> 1 + 1
2
\end{verbatim}
\afterverb
%
But in a script, an expression all by itself doesn't
do anything!  This is a common
source of confusion for beginners.

\begin{ex}
Type the following statements in the Python interpreter to see
what they do:

\beforeverb
\begin{verbatim}
5
x = 5
x + 1
\end{verbatim}
\afterverb
%
Now put the same statements into a script and run it.  What
is the output?  Modify the script by transforming each
expression into a print statement and then run it again.
\end{ex}


\section{Order of operations}
\index{order of operations}
\index{rules of precedence}
\index{PEMDAS}

When more than one operator appears in an expression, the order of
evaluation depends on the {\bf rules of precedence}.  For
mathematical operators, Python follows mathematical convention.
The acronym {\bf PEMDAS} is a useful way to
remember the rules:

\index{parentheses!overriding precedence}

\begin{itemize}

\item {\bf P}arentheses have the highest precedence and can be used 
to force an expression to evaluate in the order you want. Since
expressions in parentheses are evaluated first, {\tt 2 * (3-1)} is 4,
and {\tt (1+1)**(5-2)} is 8. You can also use parentheses to make an
expression easier to read, as in {\tt (minute * 100) / 60}, even
if it doesn't change the result.

\item {\bf E}xponentiation has the next highest precedence, so
{\tt 2**1+1} is 3, not 4, and {\tt 3*1**3} is 3, not 27.

\item {\bf M}ultiplication and {\bf D}ivision have the same precedence,
which is higher than {\bf A}ddition and {\bf S}ubtraction, which also
have the same precedence.  So {\tt 2*3-1} is 5, not 4, and
{\tt 6+4/2} is 8, not 5.

\item Operators with the same precedence are evaluated from left to 
right.  So in the expression {\tt degrees / 2 * pi}, the division
happens first and the result is multiplied by {\tt pi}.  
To divide by $2 \pi$, you can reorder the operands or use parentheses.

\end{itemize}


\section{String operations}
\index{string!operation}
\index{operator!string}

In general, you cannot perform mathematical operations on strings, even
if the strings look like numbers, so the following are illegal:

\beforeverb
\begin{verbatim}
'2'-'1'    'eggs'/'easy'    'third'*'a charm'
\end{verbatim}
\afterverb
%
The {\tt +} operator works with strings, but it
might not do what you expect: it performs
{\bf concatenation}, which means joining the strings by
linking them end-to-end.  For example:

\index{concatenation}

\beforeverb
\begin{verbatim}
first = 'throat'
second = 'warbler'
print first + second
\end{verbatim}
\afterverb
%
The output of this program is {\tt throatwarbler}.

The {\tt *} operator also works on strings; it performs repetition.
For example, \verb"'Spam'*3" is \verb"'SpamSpamSpam'".  If one of the operands
is a string, the other has to be an integer.

This use of {\tt +} and {\tt *} makes sense by
analogy with addition and multiplication.  Just as {\tt 4*3} is
equivalent to {\tt 4+4+4}, we expect \verb"'Spam'*3" to be the same as
\verb"'Spam'+'Spam'+'Spam'", and it is.  On the other hand, there is a
significant way in which string concatenation and repetition are
different from integer addition and multiplication.
Can you think of a property that addition has
that string concatenation does not?

\index{commutativity}


\section{Comments}
\index{comment}

As programs get bigger and more complicated, they get more difficult
to read.  Formal languages are dense, and it is often difficult to
look at a piece of code and figure out what it is doing, or why.

For this reason, it is a good idea to add notes to your programs to explain
in natural language what the program is doing.  These notes are called
{\bf comments}, and they start with the \verb"#" symbol:

\beforeverb
\begin{verbatim}
# compute the percentage of the hour that has elapsed
percentage = (minute * 100) / 60
\end{verbatim}
\afterverb
%
In this case, the comment appears on a line by itself.  You can also put
comments at the end of a line:

\beforeverb
\begin{verbatim}
percentage = (minute * 100) / 60     # percentage of an hour
\end{verbatim}
\afterverb
%
Everything from the {\tt \#} to the end of the line is ignored---it
has no effect on the program.

Comments are most useful when they document non-obvious features of
the code.  It is reasonable to assume that the reader can figure out
{\em what} the code does; it is much more useful to explain {\em why}.

This comment is redundant with the code and useless:

\beforeverb
\begin{verbatim}
v = 5     # assign 5 to v
\end{verbatim}
\afterverb
%
This comment contains useful information that is not in the code:

\beforeverb
\begin{verbatim}
v = 5     # velocity in meters/second. 
\end{verbatim}
\afterverb
%
Good variable names can reduce the need for comments, but
long names can make complex expressions hard to read, so there is
a tradeoff.

\section{Debugging}
\index{debugging}

At this point the syntax error you are most likely to make is
an illegal variable name, like {\tt class} and {\tt yield}, which
are keywords, or \verb"odd~job" and \verb"US$", which contain
illegal characters.

\index{syntax error}
\index{error!syntax}

If you put a space in a variable name, Python thinks it is two
operands without an operator:

\beforeverb
\begin{verbatim}
>>> bad name = 5
SyntaxError: invalid syntax
\end{verbatim}
\afterverb
%
For syntax errors, the error messages don't help much.
The most common messages are {\tt SyntaxError: invalid syntax} and
{\tt SyntaxError: invalid token}, neither of which is very informative.

\index{error message}
\index{use before def}
\index{exception}
\index{runtime error}
\index{error!runtime}

The runtime error you are most likely to make is a ``use before
def;'' that is, trying to use a variable before you have assigned
a value.  This can happen if you spell a variable name wrong:

\beforeverb
\begin{verbatim}
>>> principal = 327.68
>>> interest = principle * rate
NameError: name 'principle' is not defined
\end{verbatim}
\afterverb
%
Variables names are case sensitive, so {\tt LaTeX} is not the
same as {\tt latex}.

\index{case-sensitivity, variable names}
\index{semantic error}
\index{error!semantic}

At this point the most likely cause of a semantic error is
the order of operations.  For example, to evaluate $\frac{1}{2 \pi}$,
you might be tempted to write

\beforeverb
\begin{verbatim}
>>> 1.0 / 2.0 * pi
\end{verbatim}
\afterverb
%
But the division happens first, so you would get $\pi / 2$, which
is not the same thing!  There is no way for Python
to know what you meant to write, so in this case you don't
get an error message; you just get the wrong answer.

\index{order of operations}


\section{Glossary}

\begin{description}

\item[value:]  One of the basic units of data, like a number or string, 
that a program manipulates.
\index{value}

\item[type:] A category of values.  The types we have seen so far
are integers (type {\tt int}), floating-point numbers (type {\tt
float}), and strings (type {\tt str}).
\index{type}

\item[integer:] A type that represents whole numbers.
\index{integer}

\item[floating-point:] A type that represents numbers with fractional
parts.
\index{floating-point}

\item[string:] A type that represents sequences of characters.
\index{string}

\item[variable:]  A name that refers to a value.
\index{variable}

\item[statement:]  A section of code that represents a command or action.  So
far, the statements we have seen are assignments and print statements.
\index{statement}

\item[assignment:]  A statement that assigns a value to a variable.
\index{assignment}

\item[state diagram:]  A graphical representation of a set of variables and the
values they refer to.
\index{state diagram}

\item[keyword:]  A reserved word that is used by the compiler to parse a
program; you cannot use keywords like {\tt if}, {\tt  def}, and {\tt while} as
variable names.
\index{keyword}

\item[operator:]  A special symbol that represents a simple computation like
addition, multiplication, or string concatenation.
\index{operator}

\item[operand:]  One of the values on which an operator operates.
\index{operand}

\item[floor division:] The operation that divides two numbers and chops off
the fraction part.
\index{floor division}

\item[expression:]  A combination of variables, operators, and values that
represents a single result value.
\index{expression}

\item[evaluate:]  To simplify an expression by performing the operations
in order to yield a single value.

\item[rules of precedence:]  The set of rules governing the order in which
expressions involving multiple operators and operands are evaluated.
\index{rules of precedence}
\index{precedence}

\item[concatenate:]  To join two operands end-to-end.
\index{concatenation}

\item[comment:]  Information in a program that is meant for other
programmers (or anyone reading the source code) and has no effect on the
execution of the program.
\index{comment}

\end{description}


\section{Exercises}

\begin{ex}
Assume that we execute the following assignment statements:

\begin{verbatim}
width = 17
height = 12.0
delimiter = '.'
\end{verbatim}

For each of the following expressions, write the value of the
expression and the type (of the value of the expression).

\begin{enumerate}

\item {\tt width/2}

\item {\tt width/2.0}

\item {\tt height/3}

\item {\tt 1 + 2 * 5}

\item {\tt delimiter * 5}

\end{enumerate}

Use the Python interpreter to check your answers.
\end{ex}

\begin{ex}
Practice using the Python interpreter as a calculator: 
\index{calculator}

\begin{enumerate}

\item The volume of a sphere with radius $r$ is $\frac{4}{3} \pi r^3$.
  What is the volume of a sphere with radius 5?  Hint: 392.6 is wrong!

\item Suppose the cover price of a book is \$24.95, but bookstores get a
  40\% discount.  Shipping costs \$3 for the first copy and 75 cents
  for each additional copy.  What is the total wholesale cost for
  60 copies?

\item If I leave my house at 6:52 am and run 1 mile at an easy pace
  (8:15 per mile), then 3 miles at tempo (7:12 per mile) and 1 mile at
  easy pace again, what time do I get home for breakfast?

\index{running pace}

\end{enumerate}
\end{ex}


\chapter{Conditional execution}

\section{Modulus operator}

\index{modulus operator}
\index{operator!modulus}

The {\bf modulus operator} works on integers and yields the remainder
when the first operand is divided by the second.  In Python, the
modulus operator is a percent sign (\verb"%").  The syntax is the same
as for other operators:

\beforeverb
\begin{verbatim}
>>> quotient = 7 / 3
>>> print quotient
2
>>> remainder = 7 % 3
>>> print remainder
1
\end{verbatim}
\afterverb
%
So 7 divided by 3 is 2 with 1 left over.

The modulus operator turns out to be surprisingly useful.  For
example, you can check whether one number is divisible by another---if
{\tt x \% y} is zero, then {\tt x} is divisible by {\tt y}.

\index{divisibility}

Also, you can extract the right-most digit
or digits from a number.  For example, {\tt x \% 10} yields the
right-most digit of {\tt x} (in base 10).  Similarly {\tt x \% 100}
yields the last two digits.


\section{Boolean expressions}
\index{boolean expression}
\index{expression!boolean}
\index{logical operator}
\index{operator!logical}

A {\bf boolean expression} is an expression that is either true
or false.  The following examples use the 
operator {\tt ==}, which compares two operands and produces
{\tt True} if they are equal and {\tt False} otherwise:

\beforeverb
\begin{verbatim}
>>> 5 == 5
True
>>> 5 == 6
False
\end{verbatim}
\afterverb
%
{\tt True} and {\tt False} are special
values that belong to the type {\tt bool}; they are not strings:

\index{True special value}
\index{False special value}
\index{special value!True}
\index{special value!False}
\index{bool type}
\index{type!bool}

\beforeverb
\begin{verbatim}
>>> type(True)
<type 'bool'>
>>> type(False)
<type 'bool'>
\end{verbatim}
\afterverb
%
The {\tt ==} operator is one of the {\bf comparison operators}; the
others are:

\beforeverb
\begin{verbatim}
      x != y               # x is not equal to y
      x > y                # x is greater than y
      x < y                # x is less than y
      x >= y               # x is greater than or equal to y
      x <= y               # x is less than or equal to y
\end{verbatim}
\afterverb
%
Although these operations are probably familiar to you, the Python
symbols are different from the mathematical symbols.  A common error
is to use a single equal sign ({\tt =}) instead of a double equal sign
({\tt ==}).  Remember that {\tt =} is an assignment operator and
{\tt ==} is a comparison operator.   There is no such thing as
{\tt =<} or {\tt =>}.

\index{comparison operator}
\index{operator!comparison}


\section {Logical operators}
\index{logical operator}
\index{operator!logical}

There are three {\bf logical operators}: {\tt and}, {\tt
or}, and {\tt not}.  The semantics (meaning) of these operators is
similar to their meaning in English.  For example,
{\tt x > 0 and x < 10} is true only if {\tt x} is greater than 0
{\em and} less than 10.

\index{and operator}
\index{or operator}
\index{not operator}
\index{operator!and}
\index{operator!or}
\index{operator!not}

{\tt n\%2 == 0 or n\%3 == 0} is true if {\em either} of the conditions
is true, that is, if the number is divisible by 2 {\em or} 3.

Finally, the {\tt not} operator negates a boolean
expression, so {\tt not (x > y)} is true if {\tt x > y} is false,
that is, if {\tt x} is less than or equal to {\tt y}.

Strictly speaking, the operands of the logical operators should be
boolean expressions, but Python is not very strict.
Any nonzero number is interpreted as ``true.''

\beforeverb
\begin{verbatim}
>>> 17 and True
True
\end{verbatim}
\afterverb
%
This flexibility can be useful, but there are some subtleties to
it that might be confusing.  You might want to avoid it (unless
you know what you are doing).


\section{Conditional execution}
\label{conditional execution}

\index{conditional statement}
\index{statement!conditional}
\index{if statement}
\index{statement!if}
\index{conditional execution}

In order to write useful programs, we almost always need the ability
to check conditions and change the behavior of the program
accordingly.  {\bf Conditional statements} give us this ability.  The
simplest form is the {\tt if} statement:

\beforeverb
\begin{verbatim}
if x > 0:
    print 'x is positive'
\end{verbatim}
\afterverb
%
The boolean expression after the {\tt if} statement is
called the {\bf condition}.  If it is true, then the indented
statement gets executed.  If not, nothing happens.

\index{condition}
\index{compound statement}
\index{statement!compound}

{\tt if} statements have the same structure as function definitions:
a header followed by an indented block.  Statements like this are
called {\bf compound statements}.

There is no limit on the number of statements that can appear in
the body, but there has to be at least one.
Occasionally, it is useful to have a body with no statements (usually
as a place keeper for code you haven't written yet).  In that
case, you can use the {\tt pass} statement, which does nothing.

\index{pass statement}
\index{statement!pass}

\beforeverb
\begin{verbatim}
if x < 0:
    pass          # need to handle negative values!
\end{verbatim}
\afterverb
%

\section{Alternative execution}
\label{alternative execution}

\index{alternative execution}
\index{else keyword}
\index{keyword!else}

A second form of the {\tt if} statement is {\bf alternative execution},
in which there are two possibilities and the condition determines
which one gets executed.  The syntax looks like this:

\beforeverb
\begin{verbatim}
if x%2 == 0:
    print 'x is even'
else:
    print 'x is odd'
\end{verbatim}
\afterverb
%
If the remainder when {\tt x} is divided by 2 is 0, then we
know that {\tt x} is even, and the program displays a message to that
effect.  If the condition is false, the second set of statements is
executed.  Since the condition must be true or false, exactly one of
the alternatives will be executed.  The alternatives are called
{\bf branches}, because they are branches in the flow of execution.

\index{branch}



\section{Chained conditionals}
\index{chained conditional}
\index{conditional!chained}

Sometimes there are more than two possibilities and we need more than
two branches.  One way to express a computation like that is a {\bf
chained conditional}:

\beforeverb
\begin{verbatim}
if x < y:
    print 'x is less than y'
elif x > y:
    print 'x is greater than y'
else:
    print 'x and y are equal'
\end{verbatim}
\afterverb
%
{\tt elif} is an abbreviation of ``else if.''  Again, exactly one
branch will be executed.  There is no limit on the number of {\tt
elif} statements.  If there is an {\tt else} clause, it has to be
at the end, but there doesn't have to be one.

\index{elif keyword}
\index{keyword!elif}


\beforeverb
\begin{verbatim}
if choice == 'a':
    draw_a()
elif choice == 'b':
    draw_b()
elif choice == 'c':
    draw_c()
\end{verbatim}
\afterverb
%
Each condition is checked in order.  If the first is false,
the next is checked, and so on.  If one of them is
true, the corresponding branch executes, and the statement
ends.  Even if more than one condition is true, only the
first true branch executes.  


\section{Nested conditionals}
\index{nested conditional}
\index{conditional!nested}

One conditional can also be nested within another.  We could have
written the trichotomy example like this:

\beforeverb
\begin{verbatim}
if x == y:
    print 'x and y are equal'
else:
    if x < y:
        print 'x is less than y'
    else:
        print 'x is greater than y'
\end{verbatim}
\afterverb
%
The outer conditional contains two branches.  The
first branch contains a simple statement.  The second branch
contains another {\tt if} statement, which has two branches of its
own.  Those two branches are both simple statements,
although they could have been conditional statements as well.

Although the indentation of the statements makes the structure
apparent, {\bf nested conditionals} become difficult to read very
quickly. In general, it is a good idea to avoid them when you can.

Logical operators often provide a way to simplify nested conditional
statements.  For example, we can rewrite the following code using a
single conditional:

\beforeverb
\begin{verbatim}
if 0 < x:
    if x < 10:
        print 'x is a positive single-digit number.'
\end{verbatim}
\afterverb
%
The {\tt print} statement is executed only if we make it past both
conditionals, so we can get the same effect with the {\tt and} operator:

\beforeverb
\begin{verbatim}
if 0 < x and x < 10:
    print 'x is a positive single-digit number.'
\end{verbatim}
\afterverb



\section{Keyboard input}
\index{keyboard input}

The programs we have written so far are a bit rude in the sense that
they accept no input from the user.  They just do the same thing every
time.

Python provides a built-in function called \verb"raw_input" that gets
input from the keyboard\footnote{In Python 3.0, this function is named
  {\tt input}.}.  When this function is called, the program stops and
waits for the user to type something.  When the user presses {\sf
  Return} or {\sf Enter}, the program resumes and \verb"raw_input"
returns what the user typed as a string.

\index{Python 3.0}
\index{raw\_input function}
\index{function!raw\_input}

\beforeverb
\begin{verbatim}
>>> input = raw_input()
What are you waiting for?
>>> print input
What are you waiting for?
\end{verbatim}
\afterverb
%
Before getting input from the user, it is a good idea to print a
prompt telling the user what to input.  \verb"raw_input" can take a
prompt as an argument:

\index{prompt}

\beforeverb
\begin{verbatim}
>>> name = raw_input('What...is your name?\n')
What...is your name?
Arthur, King of the Britons!
>>> print name
Arthur, King of the Britons!
\end{verbatim}
\afterverb
%
The sequence \verb"\n" at the end of the prompt represents a {\bf newline},
which is a special character that causes a line break.
That's why the user's input appears below the prompt.

\index{newline}

If you expect the user to type an integer, you can try to convert
the return value to {\tt int}:

\beforeverb
\begin{verbatim}
>>> prompt = 'What...is the airspeed velocity of an unladen swallow?\n'
>>> speed = raw_input(prompt)
What...is the airspeed velocity of an unladen swallow?
17
>>> int(speed)
17
\end{verbatim}
\afterverb
%
But if the user types something other than a string of digits,
you get an error:

\beforeverb
\begin{verbatim}
>>> speed = raw_input(prompt)
What...is the airspeed velocity of an unladen swallow?
What do you mean, an African or a European swallow?
>>> int(speed)
ValueError: invalid literal for int()
\end{verbatim}
\afterverb
%
We will see how to handle this kind of error later.

\index{ValueError}
\index{exception!ValueError}


\section{Debugging}
\label{whitespace}
\index{debugging}
\index{traceback}

The traceback Python displays when an error occurs contains
a lot of information, but it can be overwhelming, especially
when there are many frames on the stack.  The most
useful parts are usually:

\begin{itemize}

\item What kind of error it was, and

\item Where it occurred.

\end{itemize}

Syntax errors are usually easy to find, but there are a few
gotchas.  Whitespace errors can be tricky because spaces and
tabs are invisible and we are used to ignoring them.

\index{whitespace}

\beforeverb
\begin{verbatim}
>>> x = 5
>>>  y = 6
  File "<stdin>", line 1
    y = 6
    ^
SyntaxError: invalid syntax
\end{verbatim}
\afterverb
%
In this example, the problem is that the second line is indented by
one space.  But the error message points to {\tt y}, which is
misleading.  In general, error messages indicate where the problem was
discovered, but the actual error might be earlier in the code,
sometimes on a previous line.

\index{error!runtime}
\index{runtime error}

The same is true of runtime errors.  Suppose you are trying
to compute a signal-to-noise ratio in decibels.  The formula
is $SNR_{db} = 10 \log_{10} (P_{signal} / P_{noise})$.  In Python,
you might write something like this:

\beforeverb
\begin{verbatim}
import math
signal_power = 9
noise_power = 10
ratio = signal_power / noise_power
decibels = 10 * math.log10(ratio)
print decibels
\end{verbatim}
\afterverb
%
But when you run it, you get an error message\footnote{In Python 3.0,
  you no longer get an error message; the division operator performs
  floating-point division even with integer operands.}:

\index{exception!OverflowError}
\index{OverflowError}

\beforeverb
\begin{verbatim}
Traceback (most recent call last):
  File "snr.py", line 5, in ?
    decibels = 10 * math.log10(ratio)
OverflowError: math range error
\end{verbatim}
\afterverb
%
The error message indicates line 5, but there is nothing
wrong with that line.  To find the real error, it might be
useful to print the value of {\tt ratio}, which turns out to
be 0.  The problem is in line 4, because dividing two integers
does floor division.  The solution is to represent signal power
and noise power with floating-point values.

\index{floor division}
\index{division!floor}

In general, error messages tell you where the problem was discovered, 
but that is often not where it was caused.


\section{Glossary}

\begin{description}

\item[modulus operator:]  An operator, denoted with a percent sign
({\tt \%}), that works on integers and yields the remainder when one
number is divided by another.
\index{modulus operator}
\index{operator!modulus}

\item[boolean expression:]  An expression whose value is either 
{\tt True} or {\tt False}.
\index{boolean expression}
\index{expression!boolean}

\item[comparison operator:] One of the operators that compares
its operands: {\tt ==}, {\tt !=}, {\tt >}, {\tt <}, {\tt >=}, and {\tt <=}.

\item[logical operator:] One of the operators that combines boolean
expressions: {\tt and}, {\tt or}, and {\tt not}.

\item[conditional statement:]  A statement that controls the flow of
execution depending on some condition.
\index{conditional statement}
\index{statement!conditional}

\item[condition:] The boolean expression in a conditional statement
that determines which branch is executed.
\index{condition}

\item[compound statement:]  A statement that consists of a header
and a body.  The header ends with a colon (:).  The body is indented
relative to the header.
\index{compound statement}

\item[body:] The sequence of statements within a compound statement.
\index{body}

\item[branch:] One of the alternative sequences of statements in
a conditional statement.
\index{branch}

\item[chained conditional:]  A conditional statement with a series
of alternative branches.
\index{chained conditional}
\index{conditional!chained}

\item[nested conditional:]  A conditional statement that appears
in one of the branches of another conditional statement.
\index{nested conditional}
\index{conditional!nested}

\end{description}

\section{Exercises}

\begin{ex}
\index{Fermat's Last Theorem}

Fermat's Last Theorem says that there are no integers
$a$, $b$, and $c$ such that

\[ a^n + b^n = c^n \]
%
for any values of $n$ greater than 2.

\begin{enumerate}

\item Write a function named \verb"check_fermat" that takes four
parameters---{\tt a}, {\tt b}, {\tt c} and {\tt n}---and
that checks to see if Fermat's theorem holds.  If
$n$ is greater than 2 and it turns out to be true that 

\[a^n + b^n = c^n \]
%
the program should print, ``Holy smokes, Fermat was wrong!''
Otherwise the program should print, ``No, that doesn't work.''

\item Write a function that prompts the user to input values
for {\tt a}, {\tt b}, {\tt c} and {\tt n}, converts them to
integers, and uses \verb"check_fermat" to check whether they
violate Fermat's theorem.

\end{enumerate}

\end{ex}


\begin{ex}
\index{triangle}

If you are given three sticks, you may or may not be able to arrange
them in a triangle.  For example, if one of the sticks is 12 inches
long and the other two are one inch long, it is clear that you will
not be able to get the short sticks to meet in the middle.  For any
three lengths, there is a simple test to see if it is possible to form
a triangle:

\begin{quotation}
``If any of the three lengths is greater than the sum of the other
  two, then you cannot form a triangle.  Otherwise, you
  can\footnote{If the sum of two lengths equals the third, they form
    what is called a ``degenerate'' triangle.}.''
\end{quotation}

\begin{enumerate}

\item Write a function named \verb"is_triangle" that takes three
  integers as arguments, and that prints either ``Yes'' or ``No,'' depending
  on whether you can or cannot form a triangle from sticks with the
  given lengths.

\item Write a function that prompts the user to input three stick
  lengths, converts them to integers, and uses \verb"is_triangle" to
  check whether sticks with the given lengths can form a triangle.

\end{enumerate}

\end{ex}

The following exercises use TurtleWorld from Chapter~\ref{turtlechap}:

\index{TurtleWorld}

\begin{ex}

Read the following function and see if you can figure out
what it does.  Then run it (see the examples in Chapter~\ref{turtlechap}).

\beforeverb
\begin{verbatim}
def draw(t, length, n):
    if n == 0:
        return
    angle = 50
    fd(t, length*n)
    lt(t, angle)
    draw(t, length, n-1)
    rt(t, 2*angle)
    draw(t, length, n-1)
    lt(t, angle)
    bk(t, length*n)
\end{verbatim}
\afterverb

\end{ex}


\begin{ex}

\index{Koch curve}

The Koch curve is a fractal that looks something like
this:

\beforefig
\centerline{\includegraphics[height=1in]{figs/koch.eps}}
\afterfig

To draw a Koch curve with length $x$, all you have to do is

\begin{enumerate}

\item Draw a Koch curve with length $x/3$.

\item Turn left 60 degrees.

\item Draw a Koch curve with length $x/3$.

\item Turn right 120 degrees.

\item Draw a Koch curve with length $x/3$.

\item Turn left 60 degrees.

\item Draw a Koch curve with length $x/3$.

\end{enumerate}

The only exception is if $x$ is less than 3.  In that case,
you can just draw a straight line with length $x$.

\begin{enumerate}

\item Write a function called {\tt koch} that takes a turtle and
a length as parameters, and that uses the turtle to draw a Koch
curve with the given length.

\item Write a function called {\tt snowflake} that draws three
Koch curves to make the outline of a snowflake.

You can see my solution at \url{thinkpython.com/code/koch.py}.

\item The Koch curve can be generalized in several ways.  See
\url{wikipedia.org/wiki/Koch_snowflake} for examples and
implement your favorite.

\end{enumerate}
\end{ex}


\chapter{Iteration}
\index{iteration}


\section{Multiple assignment}

\index{assignment}
\index{statement!assignment}
\index{multiple assignment}

As you may have discovered, it is legal to
make more than one assignment to the same variable.  A
new assignment makes an existing variable refer to a new
value (and stop referring to the old value).

\beforeverb
\begin{verbatim}
bruce = 5
print bruce,
bruce = 7
print bruce
\end{verbatim}
\afterverb
%
The output of this program is {\tt 5 7}, because the first time
{\tt bruce} is printed, its value is 5, and the second time, its
value is 7.  The
comma at the end of the first {\tt print} statement suppresses
the newline, which is why both outputs
appear on the same line.

\index{newline}

Here is what {\bf multiple assignment} looks like in a state diagram:

\index{state diagram}
\index{diagram!state}

\beforefig
\centerline{\includegraphics{figs/assign2.eps}}
\afterfig

With multiple assignment it is especially important to distinguish
between an assignment operation and a statement of equality.  Because
Python uses the equal sign ({\tt =}) for assignment, it is tempting to
interpret a statement like {\tt a = b} as a statement of equality. It
is not!

\index{equality and assignment}

First, equality is a symmetric relation and assignment is not.  For
example, in mathematics, if $a = 7$ then $7 = a$.  But in Python, the
statement {\tt a = 7} is legal and {\tt 7 = a} is not.

Furthermore, in mathematics, a statement of equality is either true or
false, for all time.  If $a = b$ now, then $a$ will always equal $b$.
In Python, an assignment statement can make two variables equal, but
they don't have to stay that way:

\beforeverb
\begin{verbatim}
a = 5
b = a    # a and b are now equal
a = 3    # a and b are no longer equal
\end{verbatim}
\afterverb
%
The third line changes the value of {\tt a} but does not change the
value of {\tt b}, so they are no longer equal. 

Although multiple assignment is frequently helpful, you should use it
with caution.  If the values of variables change frequently, it can
make the code difficult to read and debug.


\section{Updating variables}
\label{update}

\index{update}
\index{variable!updating}

One of the most common forms of multiple assignment is an {\bf update},
where the new value of the variable depends on the old.

\beforeverb
\begin{verbatim}
x = x+1
\end{verbatim}
\afterverb
%
This means ``get the current value of {\tt x}, add one, and then
update {\tt x} with the new value.''

If you try to update a variable that doesn't exist, you get an
error, because Python evaluates the right side before it assigns
a value to {\tt x}:

\beforeverb
\begin{verbatim}
>>> x = x+1
NameError: name 'x' is not defined
\end{verbatim}
\afterverb
%
Before you can update a variable, you have to {\bf initialize}
it, usually with a simple assignment:

\index{initialization (before update)}

\beforeverb
\begin{verbatim}
>>> x = 0
>>> x = x+1
\end{verbatim}
\afterverb
%
Updating a variable by adding 1 is called an {\bf increment};
subtracting 1 is called a {\bf decrement}.

\index{increment}
\index{decrement}




\section{The {\tt while} statement}

\index{statement!while}
\index{while loop}
\index{loop!while}
\index{iteration}

Computers are often used to automate repetitive tasks.  Repeating
identical or similar tasks without making errors is something that
computers do well and people do poorly.

Because iteration is so common, Python provides several
language features to make it easier.  One is the {\tt for} statement
we saw in Section~\ref{repetition}.  We'll get back to that later.

Another is the {\tt while} statement.  Here is a version of {\tt
countdown} that uses a {\tt while} statement:

\beforeverb
\begin{verbatim}
def countdown(n):
    while n > 0:
        print n
        n = n-1
    print 'Blastoff!'
\end{verbatim}
\afterverb
%
You can almost read the {\tt while} statement as if it were English.
It means, ``While {\tt n} is greater than 0,
display the value of {\tt n} and then reduce the value of
{\tt n} by 1.  When you get to 0, display the word {\tt Blastoff!}''

\index{flow of execution}

More formally, here is the flow of execution for a {\tt while} statement:

\begin{enumerate}

\item Evaluate the condition, yielding {\tt True} or {\tt False}.

\item If the condition is false, exit the {\tt while} statement
and continue execution at the next statement.

\item If the condition is true, execute the
body and then go back to step 1.

\end{enumerate}

This type of flow is called a {\bf loop} because the third step
loops back around to the top.  

\index{condition}
\index{loop}
\index{body}

The body of the loop should change the value of one or more variables
so that eventually the condition becomes false and the loop
terminates.  Otherwise the loop will repeat forever, which is called
an {\bf infinite loop}.  An endless source of amusement for computer
scientists is the observation that the directions on shampoo,
``Lather, rinse, repeat,'' are an infinite loop.

\index{infinite loop}
\index{loop!infinite}

In the case of {\tt countdown}, we can prove that the loop
terminates because we know that the value of {\tt n} is finite, and we
can see that the value of {\tt n} gets smaller each time through the
loop, so eventually we have to get to 0.  In other
cases, it is not so easy to tell:

\beforeverb
\begin{verbatim}
def sequence(n):
    while n != 1:
        print n,
        if n%2 == 0:        # n is even
            n = n/2
        else:               # n is odd
            n = n*3+1
\end{verbatim}
\afterverb
%
The condition for this loop is {\tt n != 1}, so the loop will continue
until {\tt n} is {\tt 1}, which makes the condition false.

Each time through the loop, the program outputs the value of {\tt n}
and then checks whether it is even or odd.  If it is even, {\tt n} is 
divided by 2.  If it is odd, the value of {\tt n} is replaced with
{\tt n*3+1}. For example, if the argument passed
to {\tt sequence} is 3, the resulting sequence is 3, 10, 5, 16, 8, 4, 2, 1.

Since {\tt n} sometimes increases and sometimes decreases, there is no
obvious proof that {\tt n} will ever reach 1, or that the program
terminates.  For some particular values of {\tt n}, we can prove
termination.  For example, if the starting value is a power of two,
then the value of {\tt n} will be even each time through the loop
until it reaches 1. The previous example ends with such a sequence,
starting with 16.

\index{Collatz conjecture}

The hard question is whether we can prove that this program terminates
for {\em all positive values} of {\tt n}.  So far\footnote{See
  \url{wikipedia.org/wiki/Collatz_conjecture}.}, no one has
been able to prove it {\em or} disprove it!


\section{{\tt break}}
\index{break statement}
\index{statement!break}

Sometimes you don't know it's time to end a loop until you get half
way through the body.  In that case you can use the {\tt break}
statement to jump out of the loop.

For example, suppose you want to take input from the user until they
type {\tt done}.  You could write:

\beforeverb
\begin{verbatim}
while True:
    line = raw_input('> ')
    if line == 'done':
        break
    print line

print 'Done!'
\end{verbatim}
\afterverb
%
The loop condition is {\tt True}, which is always true, so the
loop runs until it hits the break statement.

Each time through, it prompts the user with an angle bracket.
If the user types {\tt done}, the {\tt break} statement exits
the loop.  Otherwise the program echoes whatever the user types
and goes back to the top of the loop.  Here's a sample run:

\beforeverb
\begin{verbatim}
> not done
not done
> done
Done!
\end{verbatim}
\afterverb
%
This way of writing {\tt while} loops is common because you
can check the condition anywhere in the loop (not just at the
top) and you can express the stop condition affirmatively
(``stop when this happens'') rather than negatively (``keep going
until that happens.'').


\section{Square roots}

\index{square root}

Loops are often used in programs that compute
numerical results by starting with an approximate answer and
iteratively improving it.

\index{Newton's method}

For example, one way of computing square roots is Newton's method.
Suppose that you want to know the square root of $a$.  If you start
with almost any estimate, $x$, you can compute a better
estimate with the following formula:

\[ y = \frac{x + a/x}{2} \]
%
For example, if $a$ is 4 and $x$ is 3:

\beforeverb
\begin{verbatim}
>>> a = 4.0
>>> x = 3.0
>>> y = (x + a/x) / 2
>>> print y
2.16666666667
\end{verbatim}
\afterverb
%
Which is closer to the correct answer ($\sqrt{4} = 2$).  If we
repeat the process with the new estimate, it gets even closer:

\beforeverb
\begin{verbatim}
>>> x = y
>>> y = (x + a/x) / 2
>>> print y
2.00641025641
\end{verbatim}
\afterverb
%
After a few more updates, the estimate is almost exact:

\index{update}

\beforeverb
\begin{verbatim}
>>> x = y
>>> y = (x + a/x) / 2
>>> print y
2.00001024003
>>> x = y
>>> y = (x + a/x) / 2
>>> print y
2.00000000003
\end{verbatim}
\afterverb
%
In general we don't know ahead of time how many steps it takes
to get to the right answer, but we know when we get there
because the estimate
stops changing:

\beforeverb
\begin{verbatim}
>>> x = y
>>> y = (x + a/x) / 2
>>> print y
2.0
>>> x = y
>>> y = (x + a/x) / 2
>>> print y
2.0
\end{verbatim}
\afterverb
%
When {\tt y == x}, we can stop.  Here is a loop that starts
with an initial estimate, {\tt x}, and improves it until it
stops changing:

\beforeverb
\begin{verbatim}
while True:
    print x
    y = (x + a/x) / 2
    if y == x:
        break
    x = y
\end{verbatim}
\afterverb
%
For most values of {\tt a} this works fine, but in general it is
dangerous to test {\tt float} equality.
Floating-point values are only approximately right:
most rational numbers, like $1/3$, and irrational numbers, like
$\sqrt{2}$, can't be represented exactly with a {\tt float}.

\index{floating-point}
\index{epsilon}

Rather than checking whether {\tt x} and {\tt y} are exactly equal, it
is safer to use the built-in function {\tt abs} to compute the
absolute value, or magnitude, of the difference between them:

\beforeverb
\begin{verbatim}
    if abs(y-x) < epsilon:
        break
\end{verbatim}
\afterverb
%
Where \verb"epsilon" has a value like {\tt 0.0000001} that
determines how close is close enough.

\begin{ex}
\label{square_root}
\index{encapsulation}

Encapsulate this loop in a function called \verb"square_root"
that takes {\tt a} as a parameter, chooses a reasonable
value of {\tt x}, and returns an estimate of the square root
of {\tt a}.
\end{ex}


\section{Algorithms}
\index{algorithm}

Newton's method is an example of an {\bf algorithm}: it is a
mechanical process for solving a category of problems (in this
case, computing square roots).

It is not easy to define an algorithm.  It might help to start
with something that is not an algorithm.  When you learned
to multiply single-digit numbers, you probably memorized the
multiplication table.  In effect, you memorized 100 specific solutions.
That kind of knowledge is not algorithmic.

But if you were ``lazy,'' you probably cheated by learning a few
tricks.  For example, to find the product of $n$ and 9, you can
write $n-1$ as the first digit and $10-n$ as the second
digit.  This trick is a general solution for multiplying any
single-digit number by 9.  That's an algorithm!

\index{addition with carrying}
\index{carrying, addition with}
\index{subtraction!with borrowing}
\index{borrowing, subtraction with}

Similarly, the techniques you learned for addition with carrying,
subtraction with borrowing, and long division are all algorithms.  One
of the characteristics of algorithms is that they do not require any
intelligence to carry out.  They are mechanical processes in which
each step follows from the last according to a simple set of rules.

In my opinion, it is embarrassing that humans spend so much time in
school learning to execute algorithms that, quite literally, require
no intelligence.

On the other hand, the process of designing algorithms is interesting,
intellectually challenging, and a central part of what we call
programming.

Some of the things that people do naturally, without difficulty or
conscious thought, are the hardest to express algorithmically.
Understanding natural language is a good example.  We all do it, but
so far no one has been able to explain {\em how} we do it, at least
not in the form of an algorithm.


\section{Debugging}

As you start writing bigger programs, you might find yourself
spending more time debugging.  More code means more chances to
make an error and more place for bugs to hide.

\index{debugging!by bisection}
\index{bisection, debugging by}

One way to cut your debugging time is ``debugging by bisection.''
For example, if there are 100 lines in your program and you
check them one at a time, it would take 100 steps.

Instead, try to break the problem in half.  Look at the middle
of the program, or near it, for an intermediate value you
can check.  Add a {\tt print} statement (or something else
that has a verifiable effect) and run the program.

If the mid-point check is incorrect, the problem must be in the
first half of the program.  If it is correct, the problem is
in the second half.

Every time you perform a check like this, you halve the number
of lines you have to search.  After six steps (which is much
less than 100), you would be down to one or two lines of code,
at least in theory.

In practice it is not always clear what
the ``middle of the program'' is and not always possible to
check it.  It doesn't make sense to count lines and find the
exact midpoint.  Instead, think about places
in the program where there might be errors and places where it
is easy to put a check.  Then choose a spot where you
think the chances are about the same that the bug is before
or after the check.




\section{Glossary}

\begin{description}

\item[multiple assignment:] Making more than one assignment to the same
variable during the execution of a program.
\index{multiple assignment}
\index{assignment!multiple}

\item[update:] An assignment where the new value of the variable
depends on the old.
\index{update}

\item[initialize:] An assignment that gives an initial value to
a variable that will be updated.

\item[increment:] An update that increases the value of a variable
(often by one).
\index{increment}

\item[decrement:] An update that decreases the value of a variable.
\index{decrement}

\item[iteration:] Repeated execution of a set of statements using
either a recursive function call or a loop.
\index{iteration}

\item[infinite loop:] A loop in which the terminating condition is
never satisfied.
\index{infinite loop}

\end{description}


\section{Exercises}

\begin{ex}

\index{algorithm!square root}

To test the square root algorithm in this chapter, you could compare
it with {\tt math.sqrt}.  Write a function named \verb"test_square_root"
that prints a table like this:

\beforeverb
\begin{verbatim}
1.0 1.0           1.0           0.0
2.0 1.41421356237 1.41421356237 2.22044604925e-16
3.0 1.73205080757 1.73205080757 0.0
4.0 2.0           2.0           0.0
5.0 2.2360679775  2.2360679775  0.0
6.0 2.44948974278 2.44948974278 0.0
7.0 2.64575131106 2.64575131106 0.0
8.0 2.82842712475 2.82842712475 4.4408920985e-16
9.0 3.0           3.0           0.0

\end{verbatim}
\afterverb
%
The first column is a number, $a$; the second column is
the square root of $a$ computed with the function from
Exercise~\ref{square_root}; the third column is the square root computed
by {\tt math.sqrt}; the fourth column is the absolute value
of the difference between the two estimates.
\end{ex}


\begin{ex}

\index{eval function}
\index{function!eval}

The built-in function {\tt eval} takes a string and evaluates
it using the Python interpreter.  For example:

\beforeverb
\begin{verbatim}
>>> eval('1 + 2 * 3')
7
>>> import math
>>> eval('math.sqrt(5)')
2.2360679774997898
>>> eval('type(math.pi)')
<type 'float'>
\end{verbatim}
\afterverb
%
Write a function called \verb"eval_loop" that iteratively
prompts the user, takes the resulting input and evaluates
it using {\tt eval}, and prints the result.

It should continue until the user enters \verb"'done'", and then
return the value of the last expression it evaluated.

\end{ex}


\begin{ex}

\index{Ramanujan, Srinivasa}

The brilliant mathematician Srinivasa Ramanujan found an
infinite series\footnote{See \url{wikipedia.org/wiki/Pi}.}
that can be used to generate a numerical
approximation of $\pi$:

\index{pi}

\[\frac{1}{\pi} = \frac{2\sqrt{2}}{9801} 
\sum^\infty_{k=0} \frac{(4k)!(1103+26390k)}{(k!)^4 396^{4k}} \]

Write a function called \verb"estimate_pi" that uses this formula
to compute and return an estimate of $\pi$.  It should use a {\tt while}
loop to compute terms of the summation until the last term is
smaller than {\tt 1e-15} (which is Python notation for $10^{-15}$).
You can check the result by comparing it to {\tt math.pi}.

You can see my solution at \url{thinkpython.com/code/pi.py}.
\end{ex}


\chapter{Strings}
\label{strings}


\section{A string is a sequence}
\index{sequence}
\index{character}
\index{bracket operator}
\index{operator!bracket}

A string is a {\bf sequence} of characters.  
You can access the characters one at a time with the
bracket operator:

\beforeverb
\begin{verbatim}
>>> fruit = 'banana'
>>> letter = fruit[1]
\end{verbatim}
\afterverb
%
The second statement selects character number 1 from {\tt
fruit} and assigns it to {\tt letter}.  

\index{index}

The expression in brackets is called an {\bf index}.  
The index indicates which character in the sequence you
want (hence the name).

But you might not get what you expect:

\beforeverb
\begin{verbatim}
>>> print letter
a
\end{verbatim}
\afterverb
%
For most people, the first letter of \verb"'banana'" is {\tt b}, not
{\tt a}.  But for computer scientists, the index is an offset from the
beginning of the string, and the offset of the first letter is zero.

\beforeverb
\begin{verbatim}
>>> letter = fruit[0]
>>> print letter
b
\end{verbatim}
\afterverb
%
So {\tt b} is the 0th letter (``zero-eth'') of \verb"'banana'", {\tt a}
is the 1th letter (``one-eth''), and {\tt n} is the 2th (``two-eth'')
letter.

\index{index!starting at zero}
\index{zero, index starting at}

You can use any expression, including variables and operators, as an
index, but the value of the index has to be an integer.  Otherwise you
get:

\index{index}
\index{exception!TypeError}
\index{TypeError}

\beforeverb
\begin{verbatim}
>>> letter = fruit[1.5]
TypeError: string indices must be integers
\end{verbatim}
\afterverb
%

\section{{\tt len}}

\index{len function}
\index{function!len}

{\tt len} is a built-in function that returns the number of characters
in a string:

\beforeverb
\begin{verbatim}
>>> fruit = 'banana'
>>> len(fruit)
6
\end{verbatim}
\afterverb
%
To get the last letter of a string, you might be tempted to try something
like this:

\index{exception!IndexError}
\index{IndexError}

\beforeverb
\begin{verbatim}
>>> length = len(fruit)
>>> last = fruit[length]
IndexError: string index out of range
\end{verbatim}
\afterverb
%
The reason for the {\tt IndexError} is that there is no letter in {\tt
'banana'} with the index 6.  Since we started counting at zero, the
six letters are numbered 0 to 5.  To get the last character, you have
to subtract 1 from {\tt length}:

\beforeverb
\begin{verbatim}
>>> last = fruit[length-1]
>>> print last
a
\end{verbatim}
\afterverb
%
Alternatively, you can use negative indices, which count backward from
the end of the string.  The expression {\tt fruit[-1]} yields the last
letter, {\tt fruit[-2]} yields the second to last, and so on.

\index{index!negative}
\index{negative index}


\section{Traversal with a {\tt for} loop}
\label{for}

\index{traversal}
\index{loop!traversal}
\index{for loop}
\index{loop!for}
\index{statement!for}
\index{traversal}

A lot of computations involve processing a string one character at a
time.  Often they start at the beginning, select each character in
turn, do something to it, and continue until the end.  This pattern of
processing is called a {\bf traversal}.  One way to write a traversal
is with a {\tt while} loop:

\beforeverb
\begin{verbatim}
index = 0
while index < len(fruit):
    letter = fruit[index]
    print letter
    index = index + 1
\end{verbatim}
\afterverb
%
This loop traverses the string and displays each letter on a line by
itself.  The loop condition is {\tt index < len(fruit)}, so
when {\tt index} is equal to the length of the string, the
condition is false, and the body of the loop is not executed.  The
last character accessed is the one with the index {\tt len(fruit)-1},
which is the last character in the string.

\begin{ex}
Write a function that takes a string as an argument
and displays the letters backward, one per line.
\end{ex}

Another way to write a traversal is with a {\tt for} loop:

\beforeverb
\begin{verbatim}
for char in fruit:
    print char
\end{verbatim}
\afterverb
%
Each time through the loop, the next character in the string is assigned
to the variable {\tt char}.  The loop continues until no characters are
left.

\index{concatenation}
\index{abecedarian}
\index{McCloskey, Robert}

The following example shows how to use concatenation (string addition)
and a {\tt for} loop to generate an abecedarian series (that is, in
alphabetical order).  In Robert McCloskey's book {\em Make
Way for Ducklings}, the names of the ducklings are Jack, Kack, Lack,
Mack, Nack, Ouack, Pack, and Quack.  This loop outputs these names in
order:

\beforeverb
\begin{verbatim}
prefixes = 'JKLMNOPQ'
suffix = 'ack'

for letter in prefixes:
    print letter + suffix
\end{verbatim}
\afterverb
%
The output is:

\beforeverb
\begin{verbatim}
Jack
Kack
Lack
Mack
Nack
Oack
Pack
Qack
\end{verbatim}
\afterverb
%
Of course, that's not quite right because ``Ouack'' and
``Quack'' are misspelled.

\begin{ex}
Modify the program to fix this error.
\end{ex}



\section{String slices}
\label{slice}

\index{slice operator}
\index{operator!slice}
\index{index!slice}
\index{string!slice}
\index{slice!string}

A segment of a string is called a {\bf slice}.  Selecting a slice is
similar to selecting a character:

\beforeverb
\begin{verbatim}
>>> s = 'Monty Python'
>>> print s[0:5]
Monty
>>> print s[6:13]
Python
\end{verbatim}
\afterverb
%
The operator {\tt [n:m]} returns the part of the string from the 
``n-eth'' character to the ``m-eth'' character, including the first but
excluding the last.  This behavior is counterintuitive, but it might
help to imagine the indices pointing {\em between} the
characters, as in the following diagram:

\beforefig
\centerline{\includegraphics{figs/banana.eps}}
\afterfig

If you omit the first index (before the colon), the slice starts at
the beginning of the string.  If you omit the second index, the slice
goes to the end of the string:

\beforeverb
\begin{verbatim}
>>> fruit = 'banana'
>>> fruit[:3]
'ban'
>>> fruit[3:]
'ana'
\end{verbatim}
\afterverb
%
If the first index is greater than or equal to the second the result
is an {\bf empty string}, represented by two quotation marks:

\index{quotation mark}

\beforeverb
\begin{verbatim}
>>> fruit = 'banana'
>>> fruit[3:3]
''
\end{verbatim}
\afterverb
%
An empty string contains no characters and has length 0, but other
than that, it is the same as any other string.

\begin{ex}
Given that {\tt fruit} is a string, what does
{\tt fruit[:]} mean?

\index{copy!slice}
\index{slice!copy}


\end{ex}


\section{Strings are immutable}
\index{mutability}
\index{immutability}
\index{string!immutable}

It is tempting to use the {\tt []} operator on the left side of an
assignment, with the intention of changing a character in a string.
For example:

\index{TypeError}
\index{exception!TypeError}

\beforeverb
\begin{verbatim}
>>> greeting = 'Hello, world!'
>>> greeting[0] = 'J'
TypeError: object does not support item assignment
\end{verbatim}
\afterverb
%
The ``object'' in this case is the string and the ``item'' is
the character you tried to assign.  For now, an {\bf object} is
the same thing as a value, but we will refine that definition
later.  An {\bf item} is one of the values in a sequence.

\index{object}
\index{item assignment}
\index{assignment!item}
\index{immutability}

The reason for the error is that
strings are {\bf immutable}, which means you can't change an
existing string.  The best you can do is create a new string
that is a variation on the original:

\beforeverb
\begin{verbatim}
>>> greeting = 'Hello, world!'
>>> new_greeting = 'J' + greeting[1:]
>>> print new_greeting
Jello, world!
\end{verbatim}
\afterverb
%
This example concatenates a new first letter onto
a slice of {\tt greeting}.  It has no effect on
the original string.

\index{concatenation}


\section{Searching}
\label{find}

What does the following function do?

\index{find function}
\index{function!find}

\beforeverb
\begin{verbatim}
def find(word, letter):
    index = 0
    while index < len(word):
        if word[index] == letter:
            return index
        index = index + 1
    return -1
\end{verbatim}
\afterverb
%
In a sense, {\tt find} is the opposite of the {\tt []} operator.
Instead of taking an index and extracting the corresponding character,
it takes a character and finds the index where that character
appears.  If the character is not found, the function returns {\tt
-1}.

This is the first example we have seen of a {\tt return} statement
inside a loop.  If {\tt word[index] == letter}, the function breaks
out of the loop and returns immediately.

If the character doesn't appear in the string, the program
exits the loop normally and  returns {\tt -1}.

This pattern of computation---traversing a sequence and returning
when we find what we are looking for---is a called a {\bf search}.

\index{traversal}
\index{search pattern}
\index{pattern!search}

\begin{ex}
Modify {\tt find} so that it has a
third parameter, the index in {\tt word} where it should start
looking.
\end{ex}


\section{Looping and counting}
\label{counter}

\index{counter}
\index{counting and looping}
\index{looping and counting}
\index{looping!with strings}

The following program counts the number of times the letter {\tt a}
appears in a string:

\beforeverb
\begin{verbatim}
word = 'banana'
count = 0
for letter in word:
    if letter == 'a':
        count = count + 1
print count
\end{verbatim}
\afterverb
%
This program demonstrates another pattern of computation called a {\bf
counter}.  The variable {\tt count} is initialized to 0 and then
incremented each time an {\tt a} is found.
When the loop exits, {\tt count}
contains the result---the total number of {\tt a}'s.

\begin{ex}
\index{encapsulation}

Encapsulate this code in a function named {\tt
count}, and generalize it so that it accepts the string and the
letter as arguments.
\end{ex}

\begin{ex}
Rewrite this function so that instead of
traversing the string, it uses the three-parameter version of {\tt
find} from the previous section.
\end{ex}


\section{{\tt string} methods}

A {\bf method} is similar to a function---it takes arguments and
returns a value---but the syntax is different.  For example, the
method {\tt upper} takes a string and returns a new string with
all uppercase letters:

\index{method}
\index{string!method}

Instead of the function syntax {\tt upper(word)}, it uses
the method syntax {\tt word.upper()}.

\index{dot notation}

\beforeverb
\begin{verbatim}
>>> word = 'banana'
>>> new_word = word.upper()
>>> print new_word
BANANA
\end{verbatim}
\afterverb
%
This form of dot notation specifies the name of the method, {\tt
upper}, and the name of the string to apply the method to, {\tt
word}.  The empty parentheses indicate that this method takes no
argument.

\index{parentheses!empty}

A method call is called an {\bf invocation}; in this case, we would
say that we are invoking {\tt upper} on the {\tt word}.

\index{invocation}

As it turns out, there is a string method named {\tt find} that
is remarkably similar to the function we wrote:

\beforeverb
\begin{verbatim}
>>> word = 'banana'
>>> index = word.find('a')
>>> print index
1
\end{verbatim}
\afterverb
%
In this example, we invoke {\tt find} on {\tt word} and pass
the letter we are looking for as a parameter.

Actually, the {\tt find} method is more general than our function;
it can find substrings, not just characters:

\beforeverb
\begin{verbatim}
>>> word.find('na')
2
\end{verbatim}
\afterverb
%
It can take as a second argument the index where it should start:

\index{optional argument}
\index{argument!optional}

\beforeverb
\begin{verbatim}
>>> word.find('na', 3)
4
\end{verbatim}
\afterverb
%
And as a third argument the index where it should stop:

\beforeverb
\begin{verbatim}
>>> name = 'bob'
>>> name.find('b', 1, 2)
-1
\end{verbatim}
\afterverb
%
This search fails because {\tt b} does not
appear in the index range from {\tt 1} to {\tt 2} (not including {\tt
2}).


\begin{ex}
\index{count method}
\index{method!count}

There is a string method called {\tt count} that is similar
to the function in the previous exercise.  Read the documentation
of this method
and write an invocation that counts the number of {\tt a}s
in \verb"'banana'".
\end{ex}


\section{The {\tt in} operator}
\label{inboth}

\index{in operator}
\index{operator!in}
\index{boolean operator}
\index{operator!boolean}

The word {\tt in} is a boolean operator that takes two strings and
returns {\tt True} if the first appears as a substring in the second:

\beforeverb
\begin{verbatim}
>>> 'a' in 'banana'
True
>>> 'seed' in 'banana'
False
\end{verbatim}
\afterverb
%
For example, the following function prints all the
letters from {\tt word1} that also appear in {\tt word2}:

\beforeverb
\begin{verbatim}
def in_both(word1, word2):
    for letter in word1:
        if letter in word2:
            print letter
\end{verbatim}
\afterverb
%
With well-chosen variable names,
Python sometimes reads like English.  You could read
this loop, ``for (each) letter in (the first) word, if (the) letter 
(appears) in (the second) word, print (the) letter.''

Here's what you get if you compare apples and oranges:

\beforeverb
\begin{verbatim}
>>> in_both('apples', 'oranges')
a
e
s
\end{verbatim}
\afterverb
%

\section{String comparison}

\index{string!comparison}
\index{comparison!string}

The comparison operators work on strings.  To see if two strings are equal:

\beforeverb
\begin{verbatim}
if word == 'banana':
    print  'All right, bananas.'
\end{verbatim}
\afterverb
%
Other comparison operations are useful for putting words in alphabetical
order:

\beforeverb
\begin{verbatim}
if word < 'banana':
    print 'Your word,' + word + ', comes before banana.'
elif word > 'banana':
    print 'Your word,' + word + ', comes after banana.'
else:
    print 'All right, bananas.'
\end{verbatim}
\afterverb
%
Python does not handle uppercase and lowercase letters the same way
that people do.  All the uppercase letters come before all the
lowercase letters, so:

\beforeverb
\begin{verbatim}
Your word, Pineapple, comes before banana.
\end{verbatim}
\afterverb
%
A common way to address this problem is to convert strings to a
standard format, such as all lowercase, before performing the
comparison.  Keep that in mind in case you have to defend yourself
against a man armed with a Pineapple.


\section{Debugging}
\index{debugging}

\index{traversal}

When you use indices to traverse the values in a sequence,
it is tricky to get the beginning and end of the traversal
right.  Here is a function that is supposed to compare two
words and return {\tt True} if one of the words is the reverse
of the other, but it contains two errors:

\beforeverb
\begin{verbatim}
def is_reverse(word1, word2):
    if len(word1) != len(word2):
        return False
    
    i = 0
    j = len(word2)

    while j > 0:
        if word1[i] != word2[j]:
            return False
        i = i+1
        j = j-1

    return True
\end{verbatim}
\afterverb
%
The first {\tt if} statement checks whether the words are the
same length.  If not, we can return {\tt False} immediately
and then, for the rest of the function, we can assume that the words
are the same length.  This is an example of the guardian pattern
in Section~\ref{guardian}.

\index{guardian pattern}
\index{pattern!guardian}
\index{index}

{\tt i} and {\tt j} are indices: {\tt i} traverses {\tt word1}
forward while {\tt j} traverses {\tt word2} backward.  If we find
two letters that don't match, we can return {\tt False} immediately.
If we get through the whole loop and all the letters match, we
return {\tt True}.

If we test this function with the words ``pots'' and ``stop'', we
expect the return value {\tt True}, but we get an IndexError:

\index{IndexError}
\index{exception!IndexError}

\beforeverb
\begin{verbatim}
>>> is_reverse('pots', 'stop')
...
  File "reverse.py", line 15, in is_reverse
    if word1[i] != word2[j]:
IndexError: string index out of range
\end{verbatim}
\afterverb
%
For debugging this kind of error, my first move is to
print the values of the indices immediately before the line
where the error appears.

\beforeverb
\begin{verbatim}
    while j > 0:
        print i, j        # print here
        
        if word1[i] != word2[j]:
            return False
        i = i+1
        j = j-1
\end{verbatim}
\afterverb
%
Now when I run the program again, I get more information:

\beforeverb
\begin{verbatim}
>>> is_reverse('pots', 'stop')
0 4
...
IndexError: string index out of range
\end{verbatim}
\afterverb
%
The first time through the loop, the value of {\tt j} is 4,
which is out of range for the string \verb"'pots'".
The index of the last character is 3, so the
initial value for {\tt j} should be {\tt len(word2)-1}.

\index{semantic error}
\index{error!semantic}

If I fix that error and run the program again, I get:

\beforeverb
\begin{verbatim}
>>> is_reverse('pots', 'stop')
0 3
1 2
2 1
True
\end{verbatim}
\afterverb
%
This time we get the right answer, but it looks like the loop only ran
three times, which is suspicious.  To get a better idea of what is
happening, it is useful to draw a state diagram.  During the first
iteration, the frame for \verb"is_reverse" looks like this:

\index{state diagram}
\index{diagram!state}

\beforefig
\centerline{\includegraphics{figs/state4.eps}}
\afterfig

I took a little license by arranging the variables in the frame
and adding dotted lines to show that the values of {\tt i} and
{\tt j} indicate characters in {\tt word1} and {\tt word2}.

\begin{ex}
\label{is_reverse}
Starting with this diagram, execute the program on paper, changing the
values of {\tt i} and {\tt j} during each iteration.  Find and fix the
second error in this function.
\end{ex}



\section{Glossary}

\begin{description}

\item[object:] Something a variable can refer to.  For now,
you can use ``object'' and ``value'' interchangeably.
\index{object}

\item[sequence:] An ordered set; that is, a set of
values where each value is identified by an integer index.
\index{sequence}

\item[item:] One of the values in a sequence.
\index{item}

\item[index:] An integer value used to select an item in
a sequence, such as a character in a string.
\index{index}

\item[slice:] A part of a string specified by a range of indices.
\index{slice}

\item[empty string:] A string with no characters and length 0, represented
by two quotation marks.
\index{empty string}

\item[immutable:] The property of a sequence whose items cannot
be assigned.
\index{immutability}

\item[traverse:] To iterate through the items in a sequence,
performing a similar operation on each.
\index{traversal}

\item[search:] A pattern of traversal that stops
when it finds what it is looking for.
\index{search pattern}
\index{pattern!search}

\item[counter:] A variable used to count something, usually initialized
to zero and then incremented.
\index{counter}

\item[method:] A function that is associated with an object and called
using dot notation.
\index{method}

\item[invocation:] A statement that calls a method.
\index{invocation}

\end{description}


\section{Exercises}

\begin{ex}

\index{step size}
\index{slice operator}
\index{operator!slice}

A string slice can take a third index that specifies the ``step
size;'' that is, the number of spaces between successive characters.
A step size of 2 means every other character; 3 means every third,
etc.

\beforeverb
\begin{verbatim}
>>> fruit = 'banana'
>>> fruit[0:5:2]
'bnn'
\end{verbatim}
\afterverb

A step size of -1 goes through the word backwards, so
the slice \verb"[::-1]" generates a reversed string.

\index{palindrome}

Use this idiom to write a one-line version of \verb"is_palindrome"
from Exercise~\ref{palindrome}.
\end{ex}


\begin{ex}
\index{string method}
\index{method!string}

Read the documentation of the string methods at
\url{docs.python.org/lib/string-methods.html}.  You
might want to experiment with some of them to make sure
you understand how they work.  {\tt strip} and
{\tt replace} are particularly useful.

The documentation uses a syntax that might be confusing.
For example, in \verb"find(sub[, start[, end]])", the brackets
indicate optional arguments.  So {\tt sub} is required, but
{\tt start} is optional, and if you include {\tt start},
then {\tt end} is optional.
\end{ex}

\begin{ex}
The following functions are all {\em intended} to check whether a
string contains any lowercase letters, but at least some of them are
wrong.  For each function, describe what the function actually does
(assuming that the parameter is a string).

\beforeverb
\begin{verbatim}
def any_lowercase1(s):
    for c in s:
        if c.islower():
            return True
        else:
            return False

def any_lowercase2(s):
    for c in s:
        if 'c'.islower():
            return 'True'
        else:
            return 'False'

def any_lowercase3(s):
    for c in s:
        flag = c.islower()
    return flag

def any_lowercase4(s):
    flag = False
    for c in s:
        flag = flag or c.islower()
    return flag

def any_lowercase5(s):
    for c in s:
        if not c.islower():
            return False
    return True
\end{verbatim}
\afterverb

\end{ex}


\begin{ex}
\index{letter rotation}
\index{rotation, letter}

\label{exrotate}
ROT13 is a weak form of encryption that involves ``rotating'' each
letter in a word by 13 places\footnote{See
  \url{wikipedia.org/wiki/ROT13}}.  To rotate a letter means
to shift it through the alphabet, wrapping around to the beginning if
necessary, so 'A' shifted by 3 is 'D' and 'Z' shifted by 1 is 'A'.

Write a function called \verb"rotate_word"
that takes a string and an integer as parameters, and that returns
a new string that contains the letters from the original string
``rotated'' by the given amount.  

For example, ``cheer'' rotated by 7 is ``jolly'' and ``melon'' rotated
by -10 is ``cubed''.  

%For example ``sleep''
%rotated by 9 is ``bunny'' and ``latex'' rotated by 7 is ``shale''.

You might want to use the built-in functions {\tt ord}, which converts
a character to a numeric code, and {\tt chr}, which converts numeric
codes to characters.

Potentially offensive jokes on the Internet are sometimes encoded
in ROT13.  If you are not easily offended, find and decode some
of them.
\end{ex}

\chapter{Files}

\index{file}
\index{type!file}


\section{Persistence}

\index{persistence}

Most of the programs we have seen so far are transient in the
sense that they run for a short time and produce some output,
but when they end, their data disappears.  If you run the program
again, it starts with a clean slate.

Other programs are {\bf persistent}: they run for a long time
(or all the time); they keep at least some of their data
in permanent storage (a hard drive, for example); and
if they shut down and restart, they pick up where they left off.

Examples of persistent programs are operating systems, which
run pretty much whenever a computer is on, and web servers,
which run all the time, waiting for requests to come in on
the network.

One of the simplest ways for programs to maintain their data
is by reading and writing text files.  We have already seen
programs that read text files; in this chapters we will see programs
that write them.

An alternative is to store the state of the program in a database.
In this chapter I will present a simple database and a module,
{\tt pickle}, that makes it easy to store program data.

\index{pickle module}
\index{module!pickle}


\section{Reading and writing}

\index{file!reading and writing}

A text file is a sequence of characters stored on a permanent
medium like a hard drive, flash memory, or CD-ROM.  We saw how
to open and read a file in Section~\ref{wordlist}.

\index{open function}
\index{function!open}

To write a file, you have to open it with mode
\verb"'w'" as a second parameter:

\beforeverb
\begin{verbatim}
>>> fout = open('output.txt', 'w')
>>> print fout
<open file 'output.txt', mode 'w' at 0xb7eb2410>
\end{verbatim}
\afterverb
%
If the file already exists, opening it in write mode clears out
the old data and starts fresh, so be careful!
If the file doesn't exist, a new one is created.

The {\tt write} method puts data into the file.

\beforeverb
\begin{verbatim}
>>> line1 = "This here's the wattle,\n"
>>> fout.write(line1)
\end{verbatim}
\afterverb
%
Again, the file object keeps track of where it is, so if
you call {\tt write} again, it adds the new data to the end.

\beforeverb
\begin{verbatim}
>>> line2 = "the emblem of our land.\n"
>>> fout.write(line2)
\end{verbatim}
\afterverb
%
When you are done writing, you have to close the file.

\beforeverb
\begin{verbatim}
>>> fout.close()
\end{verbatim}
\afterverb
%

\index{close method}
\index{method!close}


\section{Format operator}

\index{format operator}
\index{operator!format}

The argument of {\tt write} has to be a string, so if we want
to put other values in a file, we have to convert them to
strings.  The easiest way to do that is with {\tt str}:

\beforeverb
\begin{verbatim}
>>> x = 52
>>> f.write(str(x))
\end{verbatim}
\afterverb
%
An alternative is to use the {\bf format operator}, {\tt \%}.  When
applied to integers, {\tt \%} is the modulus operator.  But
when the first operand is a string, {\tt \%} is the format operator.

\index{format string}

The first operand is the {\bf format string}, which contains
one or more {\bf format sequences}, which
specify how
the second operand is formatted.  The result is a string.

\index{format sequence}

For example, the format sequence \verb"'%d'" means that
the second operand should be formatted as an
integer ({\tt d} stands for ``decimal''):

\beforeverb
\begin{verbatim}
>>> camels = 42
>>> '%d' % camels
'42'
\end{verbatim}
\afterverb
%
The result is the string \verb"'42'", which is not to be confused
with the integer value {\tt 42}.

A format sequence can appear anywhere in the string,
so you can embed a value in a sentence:

\beforeverb
\begin{verbatim}
>>> camels = 42
>>> 'I have spotted %d camels.' % camels
'I have spotted 42 camels.'
\end{verbatim}
\afterverb
%
If there is more than one format sequence in the string,
the second argument has to be a tuple.  Each format sequence is
matched with an element of the tuple, in order.

The following example uses \verb"'%d'" to format an integer,
\verb"'%g'" to format
a floating-point number (don't ask why), and \verb"'%s'" to format
a string:

\beforeverb
\begin{verbatim}
>>> 'In %d years I have spotted %g %s.' % (3, 0.1, 'camels')
'In 3 years I have spotted 0.1 camels.'
\end{verbatim}
\afterverb
%
The number of elements in the tuple has to match the number
of format sequences in the string.  Also, the types of the
elements have to match the format sequences:

\index{exception!TypeError}
\index{TypeError}

\beforeverb
\begin{verbatim}
>>> '%d %d %d' % (1, 2)
TypeError: not enough arguments for format string
>>> '%d' % 'dollars'
TypeError: illegal argument type for built-in operation
\end{verbatim}
\afterverb
%
In the first example, there aren't enough elements; in the
second, the element is the wrong type.

The format operator is powerful, but it can be difficult to use.  You
can read more about it at
\url{docs.python.org/lib/typesseq-strings.html}.

% You can specify the number of digits as part of the format sequence.
% For example, the sequence \verb"'%8.2f'"
% formats a floating-point number to be 8 characters long, with
% 2 digits after the decimal point:

% \beforeverb
% \begin{verbatim}
% >>> '%8.2f' % 3.14159
% '    3.14'
% \end{verbatim}
% \afterverb
% %
% The result takes up eight spaces with two
% digits after the decimal point.  


\section{Filenames and paths}
\label{paths}

\index{filename}
\index{path}
\index{directory}
\index{folder}

Files are organized into {\bf directories} (also called ``folders'').
Every running program has a ``current directory,'' which is the
default directory for most operations.  
For example, when you open a file for reading, Python looks for it in the
current directory.

\index{os module}
\index{module!os}

The {\tt os} module provides functions for working with files and
directories (``os'' stands for ``operating system'').  {\tt os.getcwd}
returns the name of the current directory:

\index{getcwd function}
\index{function!getcwd}

\beforeverb
\begin{verbatim}
>>> import os
>>> cwd = os.getcwd()
>>> print cwd
/home/dinsdale
\end{verbatim}
\afterverb
%
{\tt cwd} stands for ``current working directory.''  The result in
this example is {\tt /home/dinsdale}, which is the home directory of a
user named {\tt dinsdale}.

\index{working directory}
\index{directory!working}

A string like {\tt cwd} that identifies a file is called a {\bf path}.
A {\bf relative path} starts from the current directory;
an {\bf absolute path} starts from the topmost directory in the
file system.

\index{relative path}
\index{path!relative}
\index{absolute path}
\index{path!absolute}

The paths we have seen so far are simple filenames, so they are
relative to the current directory.  To find the absolute path to
a file, you can use {\tt os.path.abspath}:

\beforeverb
\begin{verbatim}
>>> os.path.abspath('memo.txt')
'/home/dinsdale/memo.txt'
\end{verbatim}
\afterverb
%
{\tt os.path.exists} checks
whether a file or directory exists:

\index{exists function}
\index{function!exists}

\beforeverb
\begin{verbatim}
>>> os.path.exists('memo.txt')
True
\end{verbatim}
\afterverb
%
If it exists, {\tt os.path.isdir} checks whether it's a directory:

\beforeverb
\begin{verbatim}
>>> os.path.isdir('memo.txt')
False
>>> os.path.isdir('music')
True
\end{verbatim}
\afterverb
%
Similarly, {\tt os.path.isfile} checks whether it's a file.

{\tt os.listdir} returns a list of the files (and other directories)
in the given directory:

\beforeverb
\begin{verbatim}
>>> os.listdir(cwd)
['music', 'photos', 'memo.txt']
\end{verbatim}
\afterverb
%
To demonstrate these functions, the following example
``walks'' through a directory, prints
the names of all the files, and calls itself recursively on
all the directories.

\index{walk, directory}
\index{directory!walk}

\beforeverb
\begin{verbatim}
def walk(dir):
    for name in os.listdir(dir):
        path = os.path.join(dir, name)

        if os.path.isfile(path):
            print path
        else:
            walk(path)
\end{verbatim}
\afterverb
%
{\tt os.path.join} takes a directory and a file name and joins
them into a complete path.  

\begin{ex}
Modify {\tt walk} so that instead of printing the names of
the files, it returns a list of names.
\end{ex}

\begin{ex}
The {\tt os} module provides a function called {\tt walk}
that is similar to this one but more versatile.  Read
the documentation and use it to print the names of the
files in a given directory and its subdirectories.
\end{ex}


\section{Catching exceptions}
\label{catch}

A lot of things can go wrong when you try to read and write
files.  If you try to open a file that doesn't exist, you get an
{\tt IOError}:

\index{open function}
\index{function!open}
\index{exception!IOError}
\index{IOError}

\beforeverb
\begin{verbatim}
>>> fin = open('bad_file')
IOError: [Errno 2] No such file or directory: 'bad_file'
\end{verbatim}
\afterverb
%
If you don't have permission to access a file:

\index{file!permission}
\index{permission, file}

\beforeverb
\begin{verbatim}
>>> fout = open('/etc/passwd', 'w')
IOError: [Errno 13] Permission denied: '/etc/passwd'
\end{verbatim}
\afterverb
%
And if you try to open a directory for reading, you get

\beforeverb
\begin{verbatim}
>>> fin = open('/home')
IOError: [Errno 21] Is a directory
\end{verbatim}
\afterverb
%
To avoid these errors, you could use functions like {\tt os.path.exists}
and {\tt os.path.isfile}, but it would take a lot of time and code
to check all the possibilities (if ``{\tt Errno 21}'' is any
indication, there are at least 21 things that can go wrong).

\index{exception, catching}
\index{try statement}
\index{statement!try}

It is better to go ahead and try, and deal with problems if they
happen, which is exactly what the {\tt try} statement does.  The
syntax is similar to an {\tt if} statement:

\beforeverb
\begin{verbatim}
try:    
    fin = open('bad_file')
    for line in fin:
        print line
    fin.close()
except:
    print 'Something went wrong.'
\end{verbatim}
\afterverb
%
Python starts by executing the {\tt try} clause.  If all goes
well, it skips the {\tt except} clause and proceeds.  If an
exception occurs, it jumps out of the {\tt try} clause and
executes the {\tt except} clause.

Handling an exception with a {\tt try} statement is called {\bf
catching} an exception.  In this example, the {\tt except} clause
prints an error message that is not very helpful.  In general,
catching an exception gives you a chance to fix the problem, or try
again, or at least end the program gracefully.


\section{Databases}

\index{database}

A {\bf database} is a file that is organized for storing data.
Most databases are organized like a dictionary in the sense
that they map from keys to values.  The biggest difference
is that the database is on disk (or other permanent storage),
so it persists after the program ends.

\index{anydbm module}
\index{module!anydbm}

The module {\tt anydbm} provides an interface for creating
and updating database files.  As an example, I'll create a database
that contains captions for image files.

\index{open function}
\index{function!open}

Opening a database is similar
to opening other files:

\beforeverb
\begin{verbatim}
>>> import anydbm
>>> db = anydbm.open('captions.db', 'c')
\end{verbatim}
\afterverb
%
The mode \verb"'c'" means that the database should be created if
it doesn't already exist.  The result is a database object
that can be used (for most operations) like a dictionary.
If you create a new item, {\tt anydbm} updates the database file.

\index{update!database}


\beforeverb
\begin{verbatim}
>>> db['cleese.png'] = 'Photo of John Cleese.'
\end{verbatim}
\afterverb
%
When you access one of the items, {\tt anydbm} reads the file:

\beforeverb
\begin{verbatim}
>>> print db['cleese.png']
Photo of John Cleese.
\end{verbatim}
\afterverb
%
If you make another assignment to an existing key, {\tt anydbm} replaces
the old value:

\beforeverb
\begin{verbatim}
>>> db['cleese.png'] = 'Photo of John Cleese doing a silly walk.'
>>> print db['cleese.png']
Photo of John Cleese doing a silly walk.
\end{verbatim}
\afterverb
%
Many dictionary methods, like {\tt keys} and {\tt items}, also
work with database objects.  So does iteration with a {\tt for}
statement.

\index{dictionary methods!anydbm module}

\beforeverb
\begin{verbatim}
for key in db:
     print key
\end{verbatim}
\afterverb
%
As with other files, you should close the database when you are
done:

\beforeverb
\begin{verbatim}
>>> db.close()
\end{verbatim}
\afterverb
%

\index{close method}
\index{method!close}


\section{Pickling}

\index{pickling}

A limitation of {\tt anydbm} is that the keys and values have
to be strings.  If you try to use any other type, you get an
error.

\index{pickle module}
\index{module!pickle}

The {\tt pickle} module can help.  It translates
almost any type of object into a string suitable for storage in a
database, and then translates strings back into objects.

{\tt pickle.dumps} takes an object as a parameter and returns
a string representation ({\tt dumps} is short for ``dump string''):

\beforeverb
\begin{verbatim}
>>> import pickle
>>> t = [1, 2, 3]
>>> pickle.dumps(t)
'(lp0\nI1\naI2\naI3\na.'
\end{verbatim}
\afterverb
%
The format isn't obvious to human readers; it is meant to be
easy for {\tt pickle} to interpret.  {\tt pickle.loads}
(``load string'') reconstitutes the object:

\beforeverb
\begin{verbatim}
>>> t1 = [1, 2, 3]
>>> s = pickle.dumps(t1)
>>> t2 = pickle.loads(s)
>>> print t2
[1, 2, 3]
\end{verbatim}
\afterverb
%
Although the new object has the same value as the old, it is
not (in general) the same object:

\beforeverb
\begin{verbatim}
>>> t1 == t2
True
>>> t1 is t2
False
\end{verbatim}
\afterverb
%
In other words, pickling and then unpickling has the same effect
as copying the object.

You can use {\tt pickle} to store non-strings in a database.
In fact, this combination is so common that it has been
encapsulated in a module called {\tt shelve}.  

\index{shelve module}
\index{module!shelve}


\begin{ex}

\index{anagram set}
\index{set!anagram}

If you did Exercise~\ref{anagrams}, modify your solution so that
it creates a database that maps from each word in the list to
a list of words that use the same set of letters.

Write a different program that opens the database and prints
the contents in a human-readable format.
\end{ex}


\section{Pipes}

\index{shell}
\index{pipe}

Most operating systems provide a command-line interface,
also known as a {\bf shell}.  Shells usually provide commands
to navigate the file system and launch applications.  For
example, in Unix, you can change directories with {\tt cd},
display the contents of a directory with {\tt ls}, and launch
a web browser by typing (for example) {\tt firefox}.

\index{ls (Unix command)}
\index{Unix command!ls}

Any program that you can launch from the shell can also be
launched from Python using a {\bf pipe}.  A pipe is an object
that represents a running process.

For example, the Unix command {\tt ls -l} normally displays the
contents of the current directory (in long format).  You can
launch {\tt ls} with {\tt os.popen}:

\index{popen function}
\index{function!popen}

\beforeverb
\begin{verbatim}
>>> cmd = 'ls -l'
>>> fp = os.popen(cmd)
\end{verbatim}
\afterverb
%
The argument is a string that contains a shell command.  The
return value is a file pointer that behaves just like an open
file.  You can read the output from the {\tt ls} process one
line at a time with {\tt readline} or get the whole thing at
once with {\tt read}:

\index{readline method}
\index{method!readline}
\index{read method}
\index{method!read}

\beforeverb
\begin{verbatim}
>>> res = fp.read()
\end{verbatim}
\afterverb
%
When you are done, you close the pipe like a file:

\index{close method}
\index{method!close}

\beforeverb
\begin{verbatim}
>>> stat = fp.close()
>>> print stat
None
\end{verbatim}
\afterverb
%
The return value is the final status of the {\tt ls} process;
{\tt None} means that it ended normally (with no errors).

\index{file!compression}
\index{compression!file}
\index{Unix command!gunzip}
\index{gunzip (Unix command)}

A common use of pipes is to read a compressed file incrementally;
that is, without uncompressing the whole thing at once.  The
following function takes the name of a compressed file as a
parameter and returns a pipe that uses {\tt gunzip} to decompress
the contents:

\beforeverb
\begin{verbatim}
def open_gunzip(filename):
    cmd = 'gunzip -c ' + filename
    fp = os.popen(cmd)
    return fp
\end{verbatim}
\afterverb
%
If you read lines from {\tt fp} one at a time, you never have
to store the uncompressed file in memory or on disk.


\section{Writing modules}
\label{modules}

\index{module, writing}
\index{word count}

Any file that contains Python code can be imported as a module.
For example, suppose you have a file named {\tt wc.py} with the following
code:

\beforeverb
\begin{verbatim}
def linecount(filename):
    count = 0
    for line in open(filename):
        count += 1
    return count

print linecount('wc.py')
\end{verbatim}
\afterverb
%
If you run this program, it reads itself and prints the number
of lines in the file, which is 7.
You can also import it like this:

\beforeverb
\begin{verbatim}
>>> import wc
7
\end{verbatim}
\afterverb
%
Now you have a module object {\tt wc}:

\index{module object}
\index{object!module}

\beforeverb
\begin{verbatim}
>>> print wc
<module 'wc' from 'wc.py'>
\end{verbatim}
\afterverb
%
That provides a function called \verb"linecount":

\beforeverb
\begin{verbatim}
>>> wc.linecount('wc.py')
7
\end{verbatim}
\afterverb
%
So that's how you write modules in Python.

The only problem with this example is that when you import
the module it executes the test code at the bottom.  Normally
when you import a module, it defines new functions but it
doesn't execute them.

\index{import statement}
\index{statement!import}

Programs that will be imported as modules often
use the following idiom:

\beforeverb
\begin{verbatim}
if __name__ == '__main__':
    print linecount('wc.py')
\end{verbatim}
\afterverb
%
\verb"__name__" is a built-in variable that is set when the
program starts.  If the program is running as a script,
\verb"__name__" has the value \verb"__main__"; in that
case, the test code is executed.  Otherwise,
if the module is being imported, the test code is skipped.

\begin{ex}
Type this example into a file named {\tt wc.py} and run
it as a script.  Then run the Python interpreter and
{\tt import wc}.  What is the value of \verb"__name__"
when the module is being imported?

Warning: If you import a module that has already been imported,
Python does nothing.  It does not re-read the file, even if it has
changed.

\index{module!reload}
\index{reload function}
\index{function!reload}

If you want to reload a module, you can use the built-in function 
{\tt reload}, but it can be tricky, so the safest thing to do is
restart the interpreter and then import the module again.
\end{ex}



\section{Debugging}

\index{debugging}
\index{whitespace}

When you are reading and writing files, you might run into problems
with whitespace.  These errors can be hard to debug because spaces,
tabs and newlines are normally invisible:

\beforeverb
\begin{verbatim}
>>> s = '1 2\t 3\n 4'
>>> print s
1 2	 3
 4
\end{verbatim}
\afterverb

\index{repr function}
\index{function!repr}
\index{string representation}

The built-in function {\tt repr} can help.  It takes any object as an
argument and returns a string representation of the object.  For
strings, it represents whitespace
characters with backslash sequences:

\beforeverb
\begin{verbatim}
>>> print repr(s)
'1 2\t 3\n 4'
\end{verbatim}
\afterverb

This can be helpful for debugging.

One other problem you might run into is that different systems
use different characters to indicate the end of a line.  Some
systems use a newline, represented \verb"\n".  Others use
a return character, represented \verb"\r".  Some use both.
If you move files between different systems, these inconsistencies
might cause problems.

\index{end of line character}

For most systems, there are applications to convert from one
format to another.  You can find them (and read more about this
issue) at \url{wikipedia.org/wiki/Newline}.  Or, of course, you
could write one yourself.


\section{Glossary}

\begin{description}

\item[persistent:] Pertaining to a program that runs indefinitely
and keeps at least some of its data in permanent storage.
\index{persistence}

\item[format operator:] An operator, {\tt \%}, that takes a format
string and a tuple and generates a string that includes
the elements of the tuple formatted as specified by the format string.
\index{format operator}
\index{operator!format}

\item[format string:] A string, used with the format operator, that
contains format sequences.  
\index{format string}

\item[format sequence:] A sequence of characters in a format string,
like {\tt \%d}, that specifies how a value should be formatted.
\index{format sequence}

\item[text file:] A sequence of characters stored in permanent
storage like a hard drive.
\index{text file}

\item[directory:] A named collection of files, also called a folder.
\index{directory}

\item[path:] A string that identifies a file.
\index{path}

\item[relative path:] A path that starts from the current directory.
\index{relative path}

\item[absolute path:] A path that starts from the topmost directory
in the file system.
\index{absolute path}

\item[catch:] To prevent an exception from terminating
a program using the {\tt try}
and {\tt except} statements.
\index{catch}

\item[database:] A file whose contents are organized like a dictionary
with keys that correspond to values.
\index{database}

\end{description}


\section{Exercises}

\begin{ex}
\label{urllib}

\index{urllib module}
\index{module!urllib}
\index{URL}

The {\tt urllib} module provides methods for manipulating URLs
and downloading information from the web.  The following example
downloads and prints a secret message from {\tt thinkpython.com}:

\beforeverb
\begin{verbatim}
import urllib

conn = urllib.urlopen('http://thinkpython.com/secret.html')
for line in conn.fp:
    print line.strip()
\end{verbatim}
\afterverb

Run this code and follow the instructions you see there.

\index{secret exercise}
\index{exercise, secret}

\end{ex}

\begin{ex}
\label{checksum}

\index{MP3}

In a large collection of MP3 files, there may be more than one
copy of the same song, stored in different directories or with
different file names.  The goal of this exercise is to search for
these duplicates.

\begin{enumerate}

\item Write a program that searches a directory and all of its
subdirectories, recursively, and returns a list of complete paths
for all files with a given suffix (like {\tt .mp3}).
Hint: {\tt os.path} provides several useful functions for
manipulating file and path names.

\index{duplicate}
\index{MD5 algorithm}
\index{algorithm!MD5}
\index{checksum}

\item To recognize duplicates, you can use a hash function that
reads the file and generates a short summary
of the contents.  For example,
MD5 (Message-Digest algorithm 5) takes an arbitrarily-long
``message'' and returns a 128-bit ``checksum.''  The probability
is very small that two files with different contents will
return the same checksum.

You can read about MD5 at \url{wikipedia.org/wiki/Md5}.  On
a Unix system you can use the program {\tt md5sum} and a pipe to
compute checksums from Python.

\index{pipe}

\end{enumerate}

\end{ex}


\begin{ex}

\index{Internet Movie Database (IMDb)}
\index{IMDb (Internet Movie Database)}
\index{database}

The Internet Movie Database (IMDb) is an online collection of
information about movies.  Their database is available
in plain text format, so it is reasonably easy to read from
Python.  For this exercise, the files you need
are {\tt actors.list.gz} and {\tt actresses.list.gz}; you
can download them from \url{www.imdb.com/interfaces#plain}.

\index{plain text}
\index{text!plain}
\index{parse}

I have written a program that parses these files and
splits them into actor names, movie titles, etc.  You can
download it from \url{thinkpython.com/code/imdb.py}.

If you run {\tt imdb.py} as a script, it reads {\tt actors.list.gz}
and prints one actor-movie pair per line.  Or, if you {\tt import
imdb} you can use the function \verb"process_file" to, well,
process the file.  The arguments are a filename, a function
object and an optional number of lines to process.  Here is
an example:

\beforeverb
\begin{verbatim}
import imdb

def print_info(actor, date, title, role):
    print actor, date, title, role

imdb.process_file('actors.list.gz', print_info)
\end{verbatim}
\afterverb

When you call \verb"process_file", it opens {\tt filename}, reads the
contents, and calls \verb"print_info" once for each line in the file.
\verb"print_info" takes an actor, date, movie title and role as
arguments and prints them.

\begin{enumerate}

\item Write a program that reads {\tt actors.list.gz} and {\tt
  actresses.list.gz} and uses {\tt shelve} to build a database
that maps from each actor to a list of his or her films.

\index{shelve module}
\index{module!shelve}

\item Two actors are ``costars'' if they have been in at least one
  movie together.  Process the database you built in the previous step
  and build a second database that maps from each actor to a list of
  his or her costars.

\index{Bacon, Kevin}
\index{Kevin Bacon Game}

\item Write a program that can play the ``Six Degrees of Kevin
  Bacon,'' which you can read about at
  \url{wikipedia.org/wiki/Six_Degrees_of_Kevin_Bacon}.  This
problem is challenging because it requires you to find the shortest
path in a graph.  You can read about shortest path algorithms
at \url{wikipedia.org/wiki/Shortest_path_problem}.

\end{enumerate}

\end{ex}



\chapter{Functions}
\label{funcchap}

\section{Function calls}
\label{functionchap}
\index{function call}

In the context of programming, a {\bf function} is a named sequence of
statements that performs a computation.  When you define a function,
you specify the name and the sequence of statements.  Later, you can
``call'' the function by name.  
We have already seen one example of a {\bf function call}:

\beforeverb
\begin{verbatim}
>>> type(32)
<type 'int'>
\end{verbatim}
\afterverb
%
The name of the function is {\tt type}.  The expression in parentheses
is called the {\bf argument} of the function.  The result, for this
function, is the type of the argument.

\index{parentheses!argument in}

It is common to say that a function ``takes'' an argument and ``returns''
a result.  The result is called the {\bf return value}.

\index{argument}
\index{return value}


\section{Type conversion functions}
\index{conversion!type}
\index{type conversion}

% from Elkner:
% comment on whether these things are _really_ functions?
% use max as an example of a built-in?

% my reply:
% they are on the list of ``built-in functions'' so I am
% willing to call them functions.

Python provides built-in functions that convert values
from one type to another.  The {\tt int} function takes any value and
converts it to an integer, if it can, or complains otherwise:

\index{int function}
\index{function!int}

\beforeverb
\begin{verbatim}
>>> int('32')
32
>>> int('Hello')
ValueError: invalid literal for int(): Hello
\end{verbatim}
\afterverb
%
{\tt int} can convert floating-point values to integers, but it
doesn't round off; it chops off the fraction part:

\beforeverb
\begin{verbatim}
>>> int(3.99999)
3
>>> int(-2.3)
-2
\end{verbatim}
\afterverb
%
{\tt float} converts integers and strings to floating-point
numbers:

\index{float function}
\index{function!float}

\beforeverb
\begin{verbatim}
>>> float(32)
32.0
>>> float('3.14159')
3.14159
\end{verbatim}
\afterverb
%
Finally, {\tt str} converts its argument to a string:

\index{str function}
\index{function!str}

\beforeverb
\begin{verbatim}
>>> str(32)
'32'
>>> str(3.14159)
'3.14159'
\end{verbatim}
\afterverb
%



\section{Math functions}
\index{math function}
\index{function, math}

Python has a math module that provides most of the familiar
mathematical functions.  A {\bf module} is a file that contains a
collection of related functions.

\index{module}
\index{module object}

Before we can use the module, we have to import it:

\beforeverb
\begin{verbatim}
>>> import math
\end{verbatim}
\afterverb
%
This statement creates a {\bf module object} named math.  If
you print the module object, you get some information about it:

\beforeverb
\begin{verbatim}
>>> print math
<module 'math' from '/usr/lib/python2.5/lib-dynload/math.so'>
\end{verbatim}
\afterverb
%
The module object contains the functions and variables defined in the
module.  To access one of the functions, you have to specify the name
of the module and the name of the function, separated by a dot (also
known as a period).  This format is called {\bf dot notation}.

\index{dot notation}

\beforeverb
\begin{verbatim}
>>> ratio = signal_power / noise_power
>>> decibels = 10 * math.log10(ratio)

>>> radians = 0.7
>>> height = math.sin(radians)
\end{verbatim}
\afterverb
%
The first example computes the logarithm base 10 of the
signal-to-noise ratio.  The math module also provides a
function called {\tt log} that computes logarithms base {\tt e}.

\index{log function}
\index{function!log}
\index{sine function}
\index{radian}
\index{trigonometric function}
\index{function, trigonometric}

The second example finds the sine of {\tt radians}.  The name of the
variable is a hint that {\tt sin} and the other trigonometric
functions ({\tt cos}, {\tt tan}, etc.)  take arguments in radians. To
convert from degrees to radians, divide by 360 and multiply by $2
\pi$:

\beforeverb
\begin{verbatim}
>>> degrees = 45
>>> radians = degrees / 360.0 * 2 * math.pi
>>> math.sin(radians)
0.707106781187
\end{verbatim}
\afterverb
%
The expression {\tt math.pi} gets the variable {\tt pi} from the math
module.  The value of this variable is an approximation
of $\pi$, accurate to about 15 digits.

\index{pi}

If you know
your trigonometry, you can check the previous result by comparing it to
the square root of two divided by two:

\index{sqrt function}
\index{function!sqrt}

\beforeverb
\begin{verbatim}
>>> math.sqrt(2) / 2.0
0.707106781187
\end{verbatim}
\afterverb
%

\section{Composition}
\index{composition}

So far, we have looked at the elements of a program---variables,
expressions, and statements---in isolation, without talking about how to
combine them.

One of the most useful features of programming languages is their
ability to take small building blocks and {\bf compose} them.  For
example, the argument of a function can be any kind of expression,
including arithmetic operators:

\beforeverb
\begin{verbatim}
x = math.sin(degrees / 360.0 * 2 * math.pi)
\end{verbatim}
\afterverb
%
And even function calls:

\beforeverb
\begin{verbatim}
x = math.exp(math.log(x+1))
\end{verbatim}
\afterverb
%
Almost anywhere you can put a value, you can put an arbitrary
expression, with one exception: the left side of an assignment
statement has to be a variable name.  Any other expression on the left
side is a syntax error\footnote{We will see exceptions to this rule
later.}.

\beforeverb
\begin{verbatim}
>>> minutes = hours * 60                 # right
>>> hours * 60 = minutes                 # wrong!
SyntaxError: can't assign to operator
\end{verbatim}
\afterverb
%
\index{SyntaxError}
\index{exception!SyntaxError}


\section{Adding new functions}

So far, we have only been using the functions that come with Python,
but it is also possible to add new functions.
A {\bf function definition} specifies the name of a new function and
the sequence of statements that execute when the function is called.

\index{function}
\index{function definition}
\index{definition!function}

Here is an example:

\beforeverb
\begin{verbatim}
def print_lyrics():
    print "I'm a lumberjack, and I'm okay."
    print "I sleep all night and I work all day."
\end{verbatim}
\afterverb
%
{\tt def} is a keyword that indicates that this is a function
definition.  The name of the function is \verb"print_lyrics".  The
rules for function names are the same as for variable names: letters,
numbers and some punctuation marks are legal, but the first character
can't be a number.  You can't use a keyword as the name of a function,
and you should avoid having a variable and a function with the same
name.

\index{def keyword}
\index{keyword!def}
\index{argument}

The empty parentheses after the name indicate that this function
doesn't take any arguments.

\index{parentheses!empty}
\index{header}
\index{body}
\index{indentation}
\index{colon}

The first line of the function definition is called the {\bf header};
the rest is called the {\bf body}.  The header has to end with a colon
and the body has to be indented.  By convention, the indentation is
always four spaces (see Section~\ref{editor}).  The body can contain
any number of statements.

The strings in the print statements are enclosed in double
quotes.  Single quotes and double quotes do the same thing;
most people use single quotes except in cases like this where
a single quote (which is also an apostrophe) appears in the string.

\index{ellipses}

If you type a function definition in interactive mode, the interpreter
prints ellipses ({\em ...}) to let you know that the definition
isn't complete:

\beforeverb
\begin{verbatim}
>>> def print_lyrics():
...     print "I'm a lumberjack, and I'm okay."
...     print "I sleep all night and I work all day."
...
\end{verbatim}
\afterverb
%
To end the function, you have to enter an empty line (this is
not necessary in a script).

Defining a function creates a variable with the same name.

\beforeverb
\begin{verbatim}
>>> print print_lyrics
<function print_lyrics at 0xb7e99e9c>
>>> print type(print_lyrics)
<type 'function'>
\end{verbatim}
\afterverb
%
The value of \verb"print_lyrics" is a {\bf function object}, which
has type \verb"'function'".

\index{function object}
\index{object!function}

The syntax for calling the new function is the same as
for built-in functions:

\beforeverb
\begin{verbatim}
>>> print_lyrics()
I'm a lumberjack, and I'm okay.
I sleep all night and I work all day.
\end{verbatim}
\afterverb
%
Once you have defined a function, you can use it inside another
function.  For example, to repeat the previous refrain, we could write
a function called \verb"repeat_lyrics":

\beforeverb
\begin{verbatim}
def repeat_lyrics():
    print_lyrics()
    print_lyrics()
\end{verbatim}
\afterverb
%
And then call \verb"repeat_lyrics":

\beforeverb
\begin{verbatim}
>>> repeat_lyrics()
I'm a lumberjack, and I'm okay.
I sleep all night and I work all day.
I'm a lumberjack, and I'm okay.
I sleep all night and I work all day.
\end{verbatim}
\afterverb
%
But that's not really how the song goes.


\section{Definitions and uses}
\index{function definition}

Pulling together the code fragments from the previous section, the
whole program looks like this:

\beforeverb
\begin{verbatim}
def print_lyrics():
    print "I'm a lumberjack, and I'm okay."
    print "I sleep all night and I work all day."

def repeat_lyrics():
    print_lyrics()
    print_lyrics()

repeat_lyrics()
\end{verbatim}
\afterverb
%
This program contains two function definitions: \verb"print_lyrics" and
\verb"repeat_lyrics".  Function definitions get executed just like other
statements, but the effect is to create function objects.  The statements
inside the function do not get executed until the function is called, and
the function definition generates no output.

\index{use before def}

As you might expect, you have to create a function before you can
execute it.  In other words, the function definition has to be
executed before the first time it is called.

\begin{ex}
Move the last line of this program
to the top, so the function call appears before the definitions. Run 
the program and see what error
message you get.
\end{ex}

\begin{ex}
Move the function call back to the bottom
and move the definition of \verb"print_lyrics" after the definition of
\verb"repeat_lyrics".  What happens when you run this program?
\end{ex}


\section{Flow of execution}
\index{flow of execution}

In order to ensure that a function is defined before its first use,
you have to know the order in which statements are executed, which is
called the {\bf flow of execution}.

Execution always begins at the first statement of the program.
Statements are executed one at a time, in order from top to bottom.

Function definitions do not alter the flow of execution of the
program, but remember that statements inside the function are not
executed until the function is called.

A function call is like a detour in the flow of execution. Instead of
going to the next statement, the flow jumps to the body of
the function, executes all the statements there, and then comes back
to pick up where it left off.

That sounds simple enough, until you remember that one function can
call another.  While in the middle of one function, the program might
have to execute the statements in another function. But while
executing that new function, the program might have to execute yet
another function!

Fortunately, Python is good at keeping track of where it is, so each
time a function completes, the program picks up where it left off in
the function that called it.  When it gets to the end of the program,
it terminates.

What's the moral of this sordid tale?  When you read a program, you
don't always want to read from top to bottom.  Sometimes it makes
more sense if you follow the flow of execution.


\section{Parameters and arguments}
\label{parameters}
\index{parameter}
\index{function parameter}
\index{argument}
\index{function argument}

Some of the built-in functions we have seen require arguments.  For
example, when you call {\tt math.sin} you pass a number
as an argument.  Some functions take more than one argument:
{\tt math.pow} takes two, the base and the exponent.

Inside the function, the arguments are assigned to
variables called {\bf parameters}.  Here is an example of a
user-defined function that takes an argument:

\index{parentheses!parameters in}

\beforeverb
\begin{verbatim}
def print_twice(bruce):
    print bruce
    print bruce
\end{verbatim}
\afterverb
%
This function assigns the argument to a parameter
named {\tt bruce}.  When the function is called, it prints the value of
the parameter (whatever it is) twice.

This function works with any value that can be printed.

\beforeverb
\begin{verbatim}
>>> print_twice('Spam')
Spam
Spam
>>> print_twice(17)
17
17
>>> print_twice(math.pi)
3.14159265359
3.14159265359
\end{verbatim}
\afterverb
%
The same rules of composition that apply to built-in functions also
apply to user-defined functions, so we can use any kind of expression
as an argument for \verb"print_twice":

\index{composition}

\beforeverb
\begin{verbatim}
>>> print_twice('Spam '*4)
Spam Spam Spam Spam
Spam Spam Spam Spam
>>> print_twice(math.cos(math.pi))
-1.0
-1.0
\end{verbatim}
\afterverb
%
The argument is evaluated before the function is called, so
in the examples the expressions \verb"'Spam '*4" and
{\tt math.cos(math.pi)} are only evaluated once.

\index{argument}

You can also use a variable as an argument:

\beforeverb
\begin{verbatim}
>>> michael = 'Eric, the half a bee.'
>>> print_twice(michael)
Eric, the half a bee.
Eric, the half a bee.
\end{verbatim}
\afterverb
%
The name of the variable we pass as an argument ({\tt michael}) has
nothing to do with the name of the parameter ({\tt bruce}).  It
doesn't matter what the value was called back home (in the caller);
here in \verb"print_twice", we call everybody {\tt bruce}.


\section{Variables and parameters are local}
\index{local variable}
\index{variable!local}

When you create a variable inside a function, it is {\bf local},
which means that it only
exists inside the function.  For example:

\index{parentheses!parameters in}

\beforeverb
\begin{verbatim}
def cat_twice(part1, part2):
    cat = part1 + part2
    print_twice(cat)
\end{verbatim}
\afterverb
%
This function takes two arguments, concatenates them, and prints
the result twice.  Here is an example that uses it:

\index{concatenation}

\beforeverb
\begin{verbatim}
>>> line1 = 'Bing tiddle '
>>> line2 = 'tiddle bang.'
>>> cat_twice(line1, line2)
Bing tiddle tiddle bang.
Bing tiddle tiddle bang.
\end{verbatim}
\afterverb
%
When \verb"cat_twice" terminates, the variable {\tt cat}
is destroyed.  If we try to print it, we get an exception:

\index{NameError}
\index{exception!NameError}

\beforeverb
\begin{verbatim}
>>> print cat
NameError: name 'cat' is not defined
\end{verbatim}
\afterverb
%
Parameters are also local.
For example, outside \verb"print_twice", there is no
such thing as {\tt bruce}.

\index{parameter}


\section{Stack diagrams}
\label{stackdiagram}
\index{stack diagram}
\index{function frame}
\index{frame}

To keep track of which variables can be used where, it is sometimes
useful to draw a {\bf stack diagram}.  Like state diagrams, stack
diagrams show the value of each variable, but they also show the
function each variable belongs to.

\index{stack diagram}
\index{diagram!stack}

Each function is represented by a {\bf frame}.  A frame is a box
with the name of a function
beside it and the parameters and variables of the function inside it.
The stack diagram for the
previous example looks like this:

\beforefig
\centerline{\includegraphics{figs/stack.eps}}
\afterfig

The frames are arranged in a stack that indicates which function
called which, and so on.  In this example, \verb"print_twice"
was called by \verb"cat_twice", and \verb"cat_twice" was called by 
\verb"__main__", which is a special name for the topmost frame.  When
you create a variable outside of any function, it belongs to 
\verb"__main__".

Each parameter refers to the same value as its corresponding
argument.  So, {\tt part1} has the same value as
{\tt line1}, {\tt part2} has the same value as {\tt line2},
and {\tt bruce} has the same value as {\tt cat}.

If an error occurs during a function call, Python prints the
name of the function, and the name of the function that called
it, and the name of the function that called {\em that}, all the
way back to \verb"__main__".

For example, if you try to access {\tt cat} from within 
\verb"print_twice", you get a {\tt NameError}:

\beforeverb
\begin{verbatim}
Traceback (innermost last):
  File "test.py", line 13, in __main__
    cat_twice(line1, line2)
  File "test.py", line 5, in cat_twice
    print_twice(cat)
  File "test.py", line 9, in print_twice
    print cat
NameError: name 'cat' is not defined
\end{verbatim}
\afterverb
%
This list of functions is called a {\bf traceback}.  It tells you what
program file the error occurred in, and what line, and what functions
were executing at the time.  It also shows the line of code that
caused the error.

\index{traceback}

The order of the functions in the traceback is the same as the
order of the frames in the stack diagram.  The function that is
currently running is at the bottom.


\section{Fruitful functions and void functions}

\index{fruitful function}
\index{void function}
\index{function, fruitful}
\index{function, void} 

Some of the functions we are using, such as the math functions, yield
results; for lack of a better name, I call them {\bf fruitful
  functions}.  Other functions, like \verb"print_twice", perform an
action but don't return a value.  They are called {\bf void
  functions}.

When you call a fruitful function, you almost always
want to do something with the result; for example, you might
assign it to a variable or use it as part of an expression:

\beforeverb
\begin{verbatim}
x = math.cos(radians)
golden = (math.sqrt(5) + 1) / 2
\end{verbatim}
\afterverb
%
When you call a function in interactive mode, Python displays
the result:

\beforeverb
\begin{verbatim}
>>> math.sqrt(5)
2.2360679774997898
\end{verbatim}
\afterverb
%
But in a script, if you call a fruitful function all by itself,
the return value is lost forever!

\beforeverb
\begin{verbatim}
math.sqrt(5)
\end{verbatim}
\afterverb
%
This script computes the square root of 5, but since it doesn't store
or display the result, it is not very useful.

\index{interactive mode}
\index{script mode}

Void functions might display something on the screen or have some
other effect, but they don't have a return value.  If you try to
assign the result to a variable, you get a special value called
{\tt None}.

\index{None special value}
\index{special value!None}

\beforeverb
\begin{verbatim}
>>> result = print_twice('Bing')
Bing
Bing
>>> print result
None
\end{verbatim}
\afterverb
%
The value {\tt None} is not the same as the string \verb"'None'". 
It is a special value that has its own type:

\beforeverb
\begin{verbatim}
>>> print type(None)
<type 'NoneType'>
\end{verbatim}
\afterverb
%
The functions we have written so far are all void.  We will start
writing fruitful functions in a few chapters.


\section{Why functions?}
\index{function, reasons for}

It may not be clear why it is worth the trouble to divide
a program into functions.  There are several reasons:

\begin{itemize}

\item Creating a new function gives you an opportunity to name a group
of statements, which makes your program easier to read and debug.

\item Functions can make a program smaller by eliminating repetitive
code.  Later, if you make a change, you only have
to make it in one place.

\item Dividing a long program into functions allows you to debug the
parts one at a time and then assemble them into a working whole.

\item Well-designed functions are often useful for many programs.
Once you write and debug one, you can reuse it.

\end{itemize}


\section{Debugging}
\label{editor}
\index{debugging}

If you are using a text editor to write your scripts, you might
run into problems with spaces and tabs.  The best way to avoid
these problems is to use spaces exclusively (no tabs).  Most text
editors that know about Python do this by default, but some
don't.

\index{whitespace}

Tabs and spaces are usually invisible, which makes them
hard to debug, so try to find an editor that manages indentation
for you.

Also, don't forget to save your program before you run it.  Some
development environments do this automatically, but some don't.
In that case the program you are looking at in the text editor
is not the same as the program you are running.

Debugging can take a long time if you keep running the same,
incorrect, program over and over!

Make sure that the code you are looking at is the code you are running.
If you're not sure, put something like \verb"print 'hello'" at the
beginning of the program and run it again.  If you don't see
\verb"hello", you're not running the right program!




\section{Glossary}

\begin{description}

\item[function:] A named sequence of statements that performs some
useful operation.  Functions may or may not take arguments and may or
may not produce a result.
\index{function}

\item[function definition:]  A statement that creates a new function,
specifying its name, parameters, and the statements it executes.
\index{function definition}

\item[function object:]  A value created by a function definition.
The name of the function is a variable that refers to a function
object.
\index{function definition}

\item[header:] The first line of a function definition.
\index{header}

\item[body:] The sequence of statements inside a function definition.
\index{body}

\item[parameter:] A name used inside a function to refer to the value
passed as an argument.
\index{parameter}

\item[function call:] A statement that executes a function. It
consists of the function name followed by an argument list.
\index{function call}

\item[argument:]  A value provided to a function when the function is called.
This value is assigned to the corresponding parameter in the function.
\index{argument}

\item[local variable:]  A variable defined inside a function.  A local
variable can only be used inside its function.
\index{local variable}

\item[return value:]  The result of a function.  If a function call
is used as an expression, the return value is the value of
the expression.
\index{return value}

\item[fruitful function:] A function that returns a value.
\index{fruitful function}

\item[void function:] A function that doesn't return a value.
\index{void function}

\item[module:] A file that contains a
collection of related functions and other definitions.
\index{module}

\item[import statement:] A statement that reads a module file and creates
a module object.
\index{import statement}
\index{statement!import}

\item[module object:] A value created by an {\tt import} statement
that provides access to the values defined in a module.
\index{module}

\item[dot notation:]  The syntax for calling a function in another
module by specifying the module name followed by a dot (period) and
the function name.
\index{dot notation}

\item[composition:] Using an expression as part of a larger expression,
or a statement as part of a larger statement.
\index{composition}

\item[flow of execution:]  The order in which statements are executed during
a program run.
\index{flow of execution}

\item[stack diagram:]  A graphical representation of a stack of functions,
their variables, and the values they refer to.
\index{stack diagram}

\item[frame:]  A box in a stack diagram that represents a function call.
It contains the local variables and parameters of the function.
\index{function frame}
\index{frame}

\item[traceback:]  A list of the functions that are executing,
printed when an exception occurs.
\index{traceback}


\end{description}


\section{Exercises}

\begin{ex}

\index{len function}
\index{function!len}

Python provides a built-in function called {\tt len} that
returns the length of a string, so the value of \verb"len('allen')" is 5.

Write a function named \verb"right_justify" that takes a string
named {\tt s} as a parameter and prints the string with enough
leading spaces so that the last letter of the string is in column 70
of the display.

\beforeverb
\begin{verbatim}
>>> right_justify('allen')
                                                                 allen
\end{verbatim}
\afterverb

\end{ex}


\begin{ex}
\index{function object}
\index{object!function}

A function object is a value you can assign to a variable
or pass as an argument.  For example, \verb"do_twice" is a function
that takes a function object as an argument and calls it twice:

\beforeverb
\begin{verbatim}
def do_twice(f):
    f()
    f()
\end{verbatim}
\afterverb

Here's an example that uses \verb"do_twice" to call a function
named \verb"print_spam" twice.

\beforeverb
\begin{verbatim}
def print_spam():
    print 'spam'

do_twice(print_spam)
\end{verbatim}
\afterverb

\begin{enumerate}

\item Type this example into a script and test it.

\item Modify \verb"do_twice" so that it takes two arguments, a
function object and a value, and calls the function twice,
passing the value as an argument.

\item Write a more general version of \verb"print_spam", called
\verb"print_twice", that takes a string as a parameter and prints
it twice.

\item Use the modified version of \verb"do_twice" to call
\verb"print_twice" twice, passing \verb"'spam'" as an argument.

\item Define a new function called 
\verb"do_four" that takes a function object and a value
and calls the function four times, passing the value
as a parameter.  There should be only
two statements in the body of this function, not four.

\end{enumerate}

You can see my solution at \url{thinkpython.com/code/do_four.py}.

\end{ex}



\begin{ex}
This exercise\footnote{Based on an exercise in Oualline, {\em
    Practical C Programming, Third Edition}, O'Reilly (1997)} can be
done using only the statements and other features we have learned so
far.  

\index{grid}

\begin{enumerate}

\item Write a function that draws a grid like the
  following:

\beforeverb
\begin{verbatim}
+ - - - - + - - - - +
|         |         |
|         |         |
|         |         |
|         |         |
+ - - - - + - - - - +
|         |         |
|         |         |
|         |         |
|         |         |
+ - - - - + - - - - +
\end{verbatim}
\afterverb
%
Hint: to print more than one value on a line, you can print
a comma-separated sequence:

\beforeverb
\begin{verbatim}
print '+', '-'
\end{verbatim}
\afterverb
%
If the sequence ends with a comma, Python leaves the line unfinished,
so the value printed next appears on the same line.

\beforeverb
\begin{verbatim}
print '+', 
print '-'
\end{verbatim}
\afterverb
%
The output of these statements is \verb"'+ -'".

A {\tt print} statement all by itself ends the current line and
goes to the next line.

\item Use the previous function to draw a similar grid
with four rows and four columns.

\end{enumerate}

You can see my solution at \url{thinkpython.com/code/grid.py}.

\end{ex}



\chapter{Fruitful functions}
\label{fruitchap}

\section{Return values}
\index{return value}

Some of the built-in functions we have used, such as the math
functions, produce results.  Calling the function generates a
value, which we usually assign to a variable or use as part of an
expression.

\beforeverb
\begin{verbatim}
e = math.exp(1.0)
height = radius * math.sin(radians)
\end{verbatim}
\afterverb
%
All of the functions we have written so far are void; they print
something or move turtles around, but their return value is {\tt
None}.

In this chapter, we are (finally) going to write fruitful functions.
The first example is {\tt area}, which returns the area of a circle
with the given radius:

\beforeverb
\begin{verbatim}
def area(radius):
    temp = math.pi * radius**2
    return temp
\end{verbatim}
\afterverb
%
We have seen the {\tt return} statement before, but in a fruitful
function the {\tt return} statement includes
an expression.  This statement means: ``Return immediately from
this function and use the following expression as a return value.''
The expression can be arbitrarily complicated, so we could
have written this function more concisely:

\index{return statement}
\index{statement!return}

\beforeverb
\begin{verbatim}
def area(radius):
    return math.pi * radius**2
\end{verbatim}
\afterverb
%
On the other hand, {\bf temporary variables} like {\tt temp} often make
debugging easier.

\index{temporary variable}
\index{variable!temporary}

Sometimes it is useful to have multiple return statements, one in each
branch of a conditional:

\beforeverb
\begin{verbatim}
def absolute_value(x):
    if x < 0:
        return -x
    else:
        return x
\end{verbatim}
\afterverb
%
Since these {\tt return} statements are in an alternative conditional,
only one will be executed.

As soon as a return statement executes, the function
terminates without executing any subsequent statements.
Code that appears after a {\tt return} statement, or any other place
the flow of execution can never reach, is called {\bf dead code}.

\index{dead code}

In a fruitful function, it is a good idea to ensure
that every possible path through the program hits a
{\tt return} statement.  For example:

\beforeverb
\begin{verbatim}
def absolute_value(x):
    if x < 0:
        return -x
    if x > 0:
        return x
\end{verbatim}
\afterverb
%
This function is incorrect because if {\tt x} happens to be 0,
neither condition is true, and the function ends without hitting a
{\tt return} statement.  If the flow of execution gets to the end
of a function, the return value is {\tt None}, which is not
the absolute value of 0.

\index{None special value}
\index{special value!None}

\beforeverb
\begin{verbatim}
>>> print absolute_value(0)
None
\end{verbatim}
\afterverb
%
By the way, Python provides a built-in function called 
{\tt abs} that computes absolute values.

\index{abs function}
\index{function!abs}

\begin{ex}

\index{compare function}
\index{function!compare}

Write a {\tt compare} function
that returns {\tt 1} if {\tt x > y},
{\tt 0} if {\tt x == y}, and {\tt -1} if {\tt x < y}.
\end{ex}


\section{Incremental development}
\label{incremental development}
\index{development plan!incremental}

As you write larger functions, you might find yourself
spending more time debugging.

To deal with increasingly complex programs,
you might want to try a process called
{\bf incremental development}.  The goal of incremental development
is to avoid long debugging sessions by adding and testing only
a small amount of code at a time.

\index{testing!incremental development}
\index{Pythagorean theorem}

As an example, suppose you want to find the distance between two
points, given by the coordinates $(x_1, y_1)$ and $(x_2, y_2)$.
By the Pythagorean theorem, the distance is:

\begin{displaymath}
\mathrm{distance} = \sqrt{(x_2 - x_1)^2 + (y_2 - y_1)^2}
\end{displaymath}
%
The first step is to consider what a {\tt distance} function should
look like in Python.  In other words, what are the inputs (parameters)
and what is the output (return value)?

In this case, the inputs are two points, which you can represent
using four numbers.  The return value is the distance, which is
a floating-point value.

Already you can write an outline of the function:

\beforeverb
\begin{verbatim}
def distance(x1, y1, x2, y2):
    return 0.0
\end{verbatim}
\afterverb
%
Obviously, this version doesn't compute distances; it always returns
zero.  But it is syntactically correct, and it runs, which means that
you can test it before you make it more complicated.

To test the new function, call it with sample arguments:

\beforeverb
\begin{verbatim}
>>> distance(1, 2, 4, 6)
0.0
\end{verbatim}
\afterverb
%
I chose these values so that the horizontal distance is 3 and the
vertical distance is 4; that way, the result is 5
(the hypotenuse of a 3-4-5 triangle). When testing a function, it is
useful to know the right answer.

\index{testing!knowing the answer}

At this point we have confirmed that the function is syntactically
correct, and we can start adding code to the body.
A reasonable next step is to find the differences
$x_2 - x_1$ and $y_2 - y_1$.  The next version stores those values in
temporary variables and prints them.

\beforeverb
\begin{verbatim}
def distance(x1, y1, x2, y2):
    dx = x2 - x1
    dy = y2 - y1
    print 'dx is', dx
    print 'dy is', dy
    return 0.0
\end{verbatim}
\afterverb
%
If the function is working, it should display \verb"'dx is 3'" and {\tt
'dy is 4'}.  If so, we know that the function is getting the right
arguments and performing the first computation correctly.  If not,
there are only a few lines to check.

Next we compute the sum of squares of {\tt dx} and {\tt dy}:

\beforeverb
\begin{verbatim}
def distance(x1, y1, x2, y2):
    dx = x2 - x1
    dy = y2 - y1
    dsquared = dx**2 + dy**2
    print 'dsquared is: ', dsquared
    return 0.0
\end{verbatim}
\afterverb
%
Again, you would run the program at this stage and check the output
(which should be 25).
Finally, you can use {\tt math.sqrt} to compute and return the result:

\index{sqrt}
\index{function!sqrt}

\beforeverb
\begin{verbatim}
def distance(x1, y1, x2, y2):
    dx = x2 - x1
    dy = y2 - y1
    dsquared = dx**2 + dy**2
    result = math.sqrt(dsquared)
    return result
\end{verbatim}
\afterverb
%
If that works correctly, you are done.  Otherwise, you might
want to print the value of {\tt result} before the return
statement.

The final version of the function doesn't display anything when it
runs; it only returns a value.  The {\tt print} statements we wrote
are useful for debugging, but once you get the function working, you
should remove them.  Code like that is called {\bf scaffolding}
because it is helpful for building the program but is not part of the
final product.

\index{scaffolding}

When you start out, you should add only a line or two of code at a
time.  As you gain more experience, you might find yourself writing
and debugging bigger chunks.  Either way, incremental development
can save you a lot of debugging time.

The key aspects of the process are:

\begin{enumerate}

\item Start with a working program and make small incremental changes. 
At any point, if there is an error, you should have a good idea
where it is.

\item Use temporary variables to hold intermediate values so you can
display and check them.

\item Once the program is working, you might want to remove some of
the scaffolding or consolidate multiple statements into compound
expressions, but only if it does not make the program difficult to
read.

\end{enumerate}

\begin{ex}

\index{hypotenuse}

Use incremental development to write a function
called {\tt hypotenuse} that returns the length of the hypotenuse of a
right triangle given the lengths of the two legs as arguments.
Record each stage of the development process as you go.
\end{ex}


\section{Composition}

\index{composition}
\index{function composition}

As you should expect by now, you can call one function from
within another.  This ability is called {\bf composition}.

As an example, we'll write a function that takes two points,
the center of the circle and a point on the perimeter, and computes
the area of the circle.

Assume that the center point is stored in the variables {\tt xc} and
{\tt yc}, and the perimeter point is in {\tt xp} and {\tt yp}. The
first step is to find the radius of the circle, which is the distance
between the two points.  We just wrote a function, {\tt
distance}, that does that:

\beforeverb
\begin{verbatim}
radius = distance(xc, yc, xp, yp)
\end{verbatim}
\afterverb
%
The next step is to find the area of a circle with that radius;
we just wrote that, too:

\beforeverb
\begin{verbatim}
result = area(radius)
\end{verbatim}
\afterverb
%
Encapsulating these steps in a function, we get:

\index{encapsulation}

\beforeverb
\begin{verbatim}
def circle_area(xc, yc, xp, yp):
    radius = distance(xc, yc, xp, yp)
    result = area(radius)
    return result
\end{verbatim}
\afterverb
%
The temporary variables {\tt radius} and {\tt result} are useful for
development and debugging, but once the program is working, we can
make it more concise by composing the function calls:

\beforeverb
\begin{verbatim}
def circle_area(xc, yc, xp, yp):
    return area(distance(xc, yc, xp, yp))
\end{verbatim}
\afterverb
%

\section{Boolean functions}
\label{boolean}

\index{boolean function}

Functions can return booleans, which is often convenient for hiding
complicated tests inside functions.  For example:

\beforeverb
\begin{verbatim}
def is_divisible(x, y):
    if x % y == 0:
        return True
    else:
        return False
\end{verbatim}
\afterverb
%
It is common to give boolean functions names that sound like yes/no
questions; \verb"is_divisible" returns either {\tt True} or {\tt False}
to indicate whether {\tt x} is divisible by {\tt y}.

Here is an example:

\beforeverb
\begin{verbatim}
>>>   is_divisible(6, 4)
False
>>>   is_divisible(6, 3)
True
\end{verbatim}
\afterverb
%
The result of the {\tt ==} operator is a boolean, so we can write the
function more concisely by returning it directly:

\beforeverb
\begin{verbatim}
def is_divisible(x, y):
    return x % y == 0
\end{verbatim}
\afterverb
%
Boolean functions are often used in conditional statements:

\index{conditional statement}
\index{statement!conditional}

\beforeverb
\begin{verbatim}
if is_divisible(x, y):
    print 'x is divisible by y'
\end{verbatim}
\afterverb
%
It might be tempting to write something like:

\beforeverb
\begin{verbatim}
if is_divisible(x, y) == True:
    print 'x is divisible by y'
\end{verbatim}
\afterverb
%
But the extra comparison is unnecessary.

\begin{ex}
Write a function \verb"is_between(x, y, z)" that
returns {\tt True} if $x \le y \le z$ or {\tt False} otherwise.
\end{ex}

% TBD - May want ot talk about "guarding" code here - lots removed

\section{Debugging}
\label{factdebug}

\index{debugging}

Breaking a large program into smaller functions creates natural
checkpoints for debugging.  If a function is not working, there are
three possibilities to consider:

\begin{itemize}

\item There is something wrong with the arguments the function
is getting; a precondition is violated.

\item There is something wrong with the function; a postcondition
is violated.

\item There is something wrong with the return value or the
way it is being used.

\end{itemize}

To rule out the first possibility, you can add a {\tt print} statement
at the beginning of the function and display the values of the
parameters (and maybe their types).  Or you can write code
that checks the preconditions explicitly.

\index{precondition}
\index{postcondition}

If the parameters look good, add a {\tt print} statement before each
{\tt return} statement that displays the return value.  If
possible, check the result by hand.  Consider calling the
function with values that make it easy to check the result
(as in Section~\ref{incremental development}).

If the function seems to be working, look at the function call
to make sure the return value is being used correctly (or used
at all!).

\index{flow of execution}

Adding print statements at the beginning and end of a function
can help make the flow of execution more visible.

If you are confused about the flow of execution, this kind of
output can be helpful.  It takes some time to develop effective
scaffolding, but a little bit of scaffolding can save a lot of debugging.


\section{Glossary}

\begin{description}

\item[temporary variable:]  A variable used to store an intermediate value in
a complex calculation.
\index{temporary variable}
\index{variable!temporary}

\item[dead code:]  Part of a program that can never be executed, often because
it appears after a {\tt return} statement.
\index{dead code}

\item[{\tt None}:]  A special value returned by functions that
have no return statement or a return statement without an argument.
\index{None special value}
\index{special value!None}

\item[incremental development:]  A program development plan intended to
avoid debugging by adding and testing only
a small amount of code at a time.
\index{incremental development}

\item[scaffolding:]  Code that is used during program development but is
not part of the final version.
\index{scaffolding}

\item[guardian:]  A programming pattern that uses a conditional
statement to check for and handle circumstances that
might cause an error.
\index{guardian pattern}
\index{pattern!guardian}

\end{description}


\section{Exercises}

\begin{ex}
\index{stack diagram}

Draw a stack diagram for the following
program.  What does the program print?

\beforeverb
\begin{verbatim}
def b(z):
    prod = a(z, z)
    print z, prod
    return prod

def a(x, y):
    x = x + 1
    return x * y

def c(x, y, z):
    sum = x + y + z
    pow = b(sum)**2
    return pow

x = 1
y = x + 1
print c(x, y+3, x+y)
\end{verbatim}
\afterverb

\end{ex}


\begin{ex}
\index{Ackerman function}
\index{function!ack}

The Ackermann function, $A(m, n)$, is defined\footnote{See
  \url{wikipedia.org/wiki/Ackermann_function}}:

\begin{eqnarray}
A(m, n) = \begin{cases} 
              n+1 & \mbox{if } m = 0 \\ 
        A(m-1, 1) & \mbox{if } m > 0 \mbox{ and } n = 0 \\ 
A(m-1, A(m, n-1)) & \mbox{if } m > 0 \mbox{ and } n > 0.
\end{cases} 
\end{eqnarray}
%
Write a function named {\tt ack} that evaluates Ackerman's function.
Use your function to evaluate {\tt ack(3, 4)}, which should be 125.
What happens for larger values of {\tt m} and {\tt n}?

\end{ex}


\begin{ex}
\label{palindrome}

\index{palindrome}

A palindrome is a word that is spelled the same backward and
forward, like ``noon'' and ``redivider''.  Recursively, a word
is a palindrome if the first and last letters are the same
and the middle is a palindrome.

The following are functions that take a string argument and
return the first, last, and middle letters:

\beforeverb
\begin{verbatim}
def first(word):
    return word[0]

def last(word):
    return word[-1]

def middle(word):
    return word[1:-1]
\end{verbatim}
\afterverb
%
We'll see how they work in Chapter~\ref{strings}.

\begin{enumerate}

\item Type these functions into a file named {\tt palindrome.py}
and test them out.  What happens if you call {\tt middle} with
a string with two letters?  One letter?  What about the empty
string, which is written \verb"''" and contains no letters?

\item Write a function called \verb"is_palindrome" that takes
a string argument and returns {\tt True} if it is a palindrome
and {\tt False} otherwise.  Remember that you can use the
built-in function {\tt len} to check the length of a string.

\end{enumerate}

\end{ex}

\begin{ex}
A number, $a$, is a power of $b$ if it is divisible by $b$
and $a/b$ is a power of $b$.  Write a function called
\verb"is_power" that takes parameters {\tt a} and {\tt b}
and returns {\tt True} if {\tt a} is a power of {\tt b}.
\end{ex}


\begin{ex}

\index{greatest common divisor (GCD)}
\index{GCD (greatest common divisor)}

The greatest common divisor (GCD) of $a$ and $b$ is the largest number
that divides both of them with no remainder\footnote{This exercise is
  based on an example from Abelson and Sussman's {\em Structure and
    Interpretation of Computer Programs}.}.

One way to find the GCD of two numbers is Euclid's algorithm,
which is based on the observation that if $r$ is the remainder
when $a$ is divided by $b$, then $gcd(a, b) = gcd(b, r)$.
As a base case, we can consider $gcd(a, 0) = a$.

\index{Euclid's algorithm}
\index{algorithm!Euclid}

Write a function called
\verb"gcd" that takes parameters {\tt a} and {\tt b}
and returns their greatest common divisor.  If you need
help, see \url{wikipedia.org/wiki/Euclidean_algorithm}.

\end{ex}





\chapter{Case study: word play}

\section{Reading word lists}
\label{wordlist}

For the exercises in this chapter we need a list of English words.
There are lots of word lists available on the Web, but the one most
suitable for our purpose is one of the word lists collected and
contributed to the public domain by Grady Ward as part of the Moby
lexicon project\footnote{\url{wikipedia.org/wiki/Moby_Project}}.  It
is a list of 113,809 official crosswords; that is, words that are
considered valid in crossword puzzles and other word games.  In the
Moby collection, the filename is {\tt 113809of.fic}; I include a copy
of this file, with the simpler name {\tt words.txt}, along with
Swampy.

\index{Swampy}
\index{crosswords}

This file is in plain text, so you can open it with a text
editor, but you can also read it from Python.  The built-in
function {\tt open} takes the name of the file as a parameter
and returns a {\bf file object} you can use to read the file.

\index{open function}
\index{function!open}
\index{plain text}
\index{text!plain}
\index{object!file}
\index{file object}

\beforeverb
\begin{verbatim}
>>> fin = open('words.txt')
>>> print fin
<open file 'words.txt', mode 'r' at 0xb7f4b380>
\end{verbatim}
\afterverb
%
{\tt fin} is a common name for a file object used for
input.  Mode \verb"'r'" indicates that this file is open for
reading (as opposed to \verb"'w'" for writing).

\index{readline method}
\index{method!readline}

The file object provides several methods for reading, including
{\tt readline}, which reads characters from the file
until it gets to a newline and returns the result as a
string:

\beforeverb
\begin{verbatim}
>>> fin.readline()
'aa\r\n'
\end{verbatim}
\afterverb
%
The first word in this particular list is ``aa,'' which is a kind of
lava.  The sequence \verb"\r\n" represents two whitespace characters,
a carriage return and a newline, that separate this word from the
next.

The file object keeps track of where it is in the file, so
if you call {\tt readline} again, you get the next word:

\beforeverb
\begin{verbatim}
>>> fin.readline()
'aah\r\n'
\end{verbatim}
\afterverb
%
The next word is ``aah,'' which is a perfectly legitimate
word, so stop looking at me like that.
Or, if it's the whitespace that's bothering you,
we can get rid of it with the string method {\tt strip}:

\index{strip method}
\index{method!strip}

\beforeverb
\begin{verbatim}
>>> line = fin.readline()
>>> word = line.strip()
>>> print word
aahed
\end{verbatim}
\afterverb
%
You can also use a file object as part of a {\tt for} loop.
This program reads {\tt words.txt} and prints each word, one
per line:

\index{open function}
\index{function!open}

\beforeverb
\begin{verbatim}
fin = open('words.txt')
for line in fin:
    word = line.strip()
    print word
\end{verbatim}
\afterverb
%

\begin{ex}
Write a program that reads {\tt words.txt} and prints only the
words with more than 20 characters (not counting whitespace).

\index{whitespace}

\end{ex}


\section{Exercises}

There are solutions to these exercises in the next section.
You should at least attempt each one before you read the solutions.

\begin{ex}
In 1939 Ernest Vincent Wright published a 50,000 word novel called
{\em Gadsby} that does not contain the letter ``e.''  Since ``e'' is
the most common letter in English, that's not easy to do.

In fact, it is difficult to construct a solitary thought without using
that most common symbol.  It is slow going at first, but with caution
and hours of training you can gradually gain facility.

All right, I'll stop now.

Write a function called \verb"has_no_e" that returns {\tt True} if
the given word doesn't have the letter ``e'' in it.

Modify your program from the previous section to print only the words
that have no ``e'' and compute the percentage of the words in the list
have no ``e.''

\index{lipogram}

\end{ex}


\begin{ex} 
Write a function named {\tt avoids}
that takes a word and a string of forbidden letters, and
that returns {\tt True} if the word doesn't use any of the forbidden
letters.

Modify your program to prompt the user to enter a string
of forbidden letters and then print the number of words that
don't contain any of them.
Can you find a combination of 5 forbidden letters that
excludes the smallest number of words?
\end{ex}



\begin{ex}
Write a function named \verb"uses_only" that takes a word and a
string of letters, and that returns {\tt True} if the word contains
only letters in the list.  Can you make a sentence using only the
letters {\tt acefhlo}?  Other than ``Hoe alfalfa?''
\end{ex}


\begin{ex} 
Write a function named \verb"uses_all" that takes a word and a
string of required letters, and that returns {\tt True} if the word
uses all the required letters at least once.  How many words are there
that use all the vowels {\tt aeiou}?  How about {\tt aeiouy}?
\end{ex}


\begin{ex}
Write a function called \verb"is_abecedarian" that returns
{\tt True} if the letters in a word appear in alphabetical order
(double letters are ok).  
How many abecedarian words are there?
\end{ex}

\index{abecedarian}


%\begin{ex}
%\label{palindrome}
%A palindrome is a word that reads the same
%forward and backward, like ``rotator'' and ``noon.''
%Write a boolean function named \verb"is_palindrome" that
%takes a string as a parameter and returns {\tt True} if it is
%a palindrome.

%Modify your program from the previous section to print all
%of the palindromes in the word list and then print the total
%number of palindromes.
%\end{ex}



\section{Search}

\index{search pattern}
\index{pattern!search}

All of the exercises in the previous section have something
in common; they can be solved with the search pattern we saw
in Section~\ref{find}.  The simplest example is:

\beforeverb
\begin{verbatim}
def has_no_e(word):
    for letter in word:
        if letter == 'e':
            return False
    return True
\end{verbatim}
\afterverb
%
The {\tt for} loop traverses the characters in {\tt word}.  If we find
the letter ``e'', we can immediately return {\tt False}; otherwise we
have to go to the next letter.  If we exit the loop normally, that
means we didn't find an ``e'', so we return {\tt True}.

\index{traversal}
\index{in operator}
\index{operator!in}

You can write this function more concisely using the {\tt in}
operator, but I started with this version because it 
demonstrates the logic of the search pattern.

\index{generalization}

{\tt avoids} is a more general version of \verb"has_no_e" but it
has the same structure:

\beforeverb
\begin{verbatim}
def avoids(word, forbidden):
    for letter in word:
        if letter in forbidden:
            return False
    return True
\end{verbatim}
\afterverb
%
We can return {\tt False} as soon as we find a forbidden letter;
if we get to the end of the loop, we return {\tt True}.

\verb"uses_only" is similar except that the sense of the condition
is reversed:

\beforeverb
\begin{verbatim}
def uses_only(word, available):
    for letter in word: 
        if letter not in available:
            return False
    return True
\end{verbatim}
\afterverb
%
Instead of a list of forbidden letters, we have a list of available
letters.  If we find a letter in {\tt word} that is not in
{\tt available}, we can return {\tt False}.

\verb"uses_all" is similar except that we reverse the role
of the word and the string of letters:

\beforeverb
\begin{verbatim}
def uses_all(word, required):
    for letter in required: 
        if letter not in word:
            return False
    return True
\end{verbatim}
\afterverb
%
Instead of traversing the letters in {\tt word}, the loop
traverses the required letters.  If any of the required letters
do not appear in the word, we can return {\tt False}.

\index{traversal}

If you were really thinking like a computer scientist, you would
have recognized that \verb"uses_all" was an instance of a
previously-solved problem, and you would have written:

\beforeverb
\begin{verbatim}
def uses_all(word, required):
    return uses_only(required, word)
\end{verbatim}
\afterverb
%
This is an example of a program development method called {\bf problem
recognition}, which means that you recognize the problem you are
working on as an instance of a previously-solved problem, and apply a
previously-developed solution.

\index{problem recognition}
\index{development plan!problem recognition}


\section{Looping with indices}

\index{looping!with indices}
\index{index!looping with}

I wrote the functions in the previous section with {\tt for}
loops because I only needed the characters in the strings; I didn't
have to do anything with the indices.

For \verb"is_abecedarian" we have to compare adjacent letters,
which is a little tricky with a {\tt for} loop:

\beforeverb
\begin{verbatim}
def is_abecedarian(word):
    previous = word[0]
    for c in word:
        if c < previous:
            return False
        previous = c
    return True
\end{verbatim}
\afterverb


Another option is to use a {\tt while} loop:

\beforeverb
\begin{verbatim}
def is_abecedarian(word):
    i = 0
    while i < len(word)-1:
        if word[i+1] < word[i]:
            return False
        i = i+1
    return True
\end{verbatim}
\afterverb
%
The loop starts at {\tt i=0} and ends when {\tt i=len(word)-1}.  Each
time through the loop, it compares the $i$th character (which you can
think of as the current character) to the $i+1$th character (which you
can think of as the next).

If the next character is less than (alphabetically before) the current
one, then we have discovered a break in the abecedarian trend, and
we return {\tt False}.

If we get to the end of the loop without finding a fault, then the
word passes the test.  To convince yourself that the loop ends
correctly, consider an example like \verb"'flossy'".  The
length of the word is 6, so
the last time the loop runs is when {\tt i} is 4, which is the
index of the second-to-last character.  On the last iteration,
it compares the second-to-last character to the last, which is
what we want.

\index{palindrome}

Here is a version of \verb"is_palindrome" (see
Exercise~\ref{palindrome}) that uses two indices; one starts at the
beginning and goes up; the other starts at the end and goes down.

\beforeverb
\begin{verbatim}
def is_palindrome(word):
    i = 0
    j = len(word)-1

    while i<j:
        if word[i] != word[j]:
            return False
        i = i+1
        j = j-1

    return True
\end{verbatim}
\afterverb

Or, if you noticed that this is an instance of a previously-solved
problem, you might have written:

\beforeverb
\begin{verbatim}
def is_palindrome(word):
    return is_reverse(word, word)
\end{verbatim}
\afterverb

\index{problem recognition}
\index{development plan!problem recognition}

Assuming you did Exercise~\ref{is_reverse}.


\section{Debugging}

\index{debugging}
\index{testing!is hard}
\index{program testing}

Testing programs is hard.  The functions in this chapter are
relatively easy to test because you can check the results by hand.
Even so, it is somewhere between difficult and impossible to choose a
set of words that test for all possible errors.

Taking \verb"has_no_e" as an example, there are two obvious
cases to check: words that have an 'e' should return {\tt False};
words that don't should return {\tt True}.  You should have no
trouble coming up with one of each.

Within each case, there are some less obvious subcases.  Among the
words that have an ``e,'' you should test words with an ``e'' at the
beginning, the end, and somewhere in the middle.  You should test long
words, short words, and very short words, like the empty string.  The
empty string is an example of a {\bf special case}, which is one of
the non-obvious cases where errors often lurk.

\index{special case}

In addition to the test cases you generate, you can also test
your program with a word list like {\tt words.txt}.  By scanning
the output, you might be able to catch errors, but be careful:
you might catch one kind of error (words that should not be
included, but are) and not another (words that should be included,
but aren't).

In general, testing can help you find bugs, but it is not easy to
generate a good set of test cases, and even if you do, you can't
be sure your program is correct.

\index{testing!and absence of bugs}

According to a legendary computer scientist:

\begin{quote}
Program testing can be used to show the presence of bugs, but never to
show their absence!

--- Edsger W. Dijkstra
\end{quote}

\index{Dijkstra, Edsger}


\section{Glossary}

\begin{description}

\item[file object:] A value that represents an open file.
\index{file object}
\index{object!file}

\item[problem recognition:] A way of solving a problem by
expressing it as an instance of a previously-solved problem.
\index{problem recognition}

\item[special case:] A test case that is atypical or non-obvious
(and less likely to be handled correctly).
\index{special case}

\end{description}


\section{Exercises}

\begin{ex}

\index{Car Talk}
\index{Puzzler}
\index{double letters}

This question is based on a Puzzler that was broadcast on the radio
program {\em Car
  Talk}\footnote{\url{www.cartalk.com/content/puzzler/transcripts/200725}}:

\begin{quote}
Give me a word with three consecutive double letters. I'll give you a
couple of words that almost qualify, but don't. For example, the word
committee, c-o-m-m-i-t-t-e-e. It would be great except for the `i' that
sneaks in there. Or Mississippi: M-i-s-s-i-s-s-i-p-p-i. If you could
take out those i's it would work. But there is a word that has three
consecutive pairs of letters and to the best of my knowledge this may
be the only word. Of course there are probably 500 more but I can only
think of one. What is the word?
\end{quote}

Write a program to find it.  You can see my solution at
\url{thinkpython.com/code/cartalk.py}.

\end{ex}


\begin{ex}
Here's another {\em Car Talk}
Puzzler\footnote{\url{www.cartalk.com/content/puzzler/transcripts/200803}}:

\index{Car Talk}
\index{Puzzler}
\index{odometer}
\index{palindrome}

\begin{quote}
``I was driving on the highway the other day and I happened to
notice my odometer. Like most odometers, it shows six digits,
in whole miles only. So, if my car had 300,000
miles, for example, I'd see 3-0-0-0-0-0.

``Now, what I saw that day was very interesting. I noticed that the
last 4 digits were palindromic; that is, they read the same forward as
backward. For example, 5-4-4-5 is a palindrome, so my odometer
could have read 3-1-5-4-4-5.

``One mile later, the last 5 numbers were palindromic. For example, it
could have read 3-6-5-4-5-6.  One mile after that, the middle 4 out of
6 numbers were palindromic.  And you ready for this? One mile later,
all 6 were palindromic!

``The question is, what was on the odometer when I first looked?''
\end{quote}

Write a Python program that tests all the six-digit numbers and prints
any numbers that satisfy these requirements.  You can see my solution
at \url{thinkpython.com/code/cartalk.py}.

\end{ex}


\begin{ex}
Here's another {\em Car Talk} Puzzler you can solve with a
search\footnote{\url{www.cartalk.com/content/puzzler/transcripts/200813}}:

\index{Car Talk}
\index{Puzzler}
\index{palindrome}

\begin{quote}
``Recently I had a visit with my mom and we realized that
the two digits that make up my age when reversed resulted in her
age. For example, if she's 73, I'm 37. We wondered how often this has
happened over the years but we got sidetracked with other topics and
we never came up with an answer.

``When I got home I figured out that the digits of our ages have been
reversible six times so far. I also figured out that if we're lucky it
would happen again in a few years, and if we're really lucky it would
happen one more time after that. In other words, it would have
happened 8 times over all. So the question is, how old am I now?''

\end{quote}

Write a Python program that searches for solutions to this Puzzler.
Hint: you might find the string method {\tt zfill} useful.

You can see my solution at \url{thinkpython.com/code/cartalk.py}.

\end{ex}



\chapter{Lists}

\index{list}
\index{type!list}


\section{A list is a sequence}

Like a string, a {\bf list} is a sequence of values.  In a string, the
values are characters; in a list, they can be any type.  The values in
list are called {\bf elements} or sometimes {\bf items}.

\index{element}
\index{sequence}
\index{item}

There are several ways to create a new list; the simplest is to
enclose the elements in square brackets (\verb"[" and \verb"]"):

\beforeverb
\begin{verbatim}
[10, 20, 30, 40]
['crunchy frog', 'ram bladder', 'lark vomit']
\end{verbatim}
\afterverb
%
The first example is a list of four integers.  The second is a list of
three strings.  The elements of a list don't have to be the same type.
The following list contains a string, a float, an integer, and
(lo!) another list:

\beforeverb
\begin{verbatim}
['spam', 2.0, 5, [10, 20]]
\end{verbatim}
\afterverb
%
A list within another list is {\bf nested}.

\index{nested list}
\index{list!nested}

A list that contains no elements is
called an empty list; you can create one with empty
brackets, \verb"[]".

\index{empty list}
\index{list!empty}

As you might expect, you can assign list values to variables:

\beforeverb
\begin{verbatim}
>>> cheeses = ['Cheddar', 'Edam', 'Gouda']
>>> numbers = [17, 123]
>>> empty = []
>>> print cheeses, numbers, empty
['Cheddar', 'Edam', 'Gouda'] [17, 123] []
\end{verbatim}
\afterverb
%

\index{assignment}

% From Jeff: write sum for a nested list?


\section{Lists are mutable}

\index{list!element}
\index{access}
\index{index}
\index{bracket operator}
\index{operator!bracket}

The syntax for accessing the elements of a list is the same as for
accessing the characters of a string---the bracket operator.  The
expression inside the brackets specifies the index.  Remember that the
indices start at 0:

\beforeverb
\begin{verbatim}
>>> print cheeses[0]
Cheddar
\end{verbatim}
\afterverb
%
Unlike strings, lists are mutable.  When the bracket operator appears
on the left side of an assignment, it identifies the element of the
list that will be assigned.

\index{mutability}

\beforeverb
\begin{verbatim}
>>> numbers = [17, 123]
>>> numbers[1] = 5
>>> print numbers
[17, 5]
\end{verbatim}
\afterverb
%
The one-eth element of {\tt numbers}, which
used to be 123, is now 5.

\index{index!starting at zero}
\index{zero, index starting at}

You can think of a list as a relationship between indices and
elements.  This relationship is called a {\bf mapping}; each index
``maps to'' one of the elements.  Here is a state diagram showing {\tt
cheeses}, {\tt numbers} and {\tt empty}:

\index{state diagram}
\index{diagram!state}
\index{mapping}

\beforefig
\centerline{\includegraphics{figs/list_state.eps}}
\afterfig

Lists are represented by boxes with the word ``list'' outside
and the elements of the list inside.  {\tt cheeses} refers to
a list with three elements indexed 0, 1 and 2.
{\tt numbers} contains two elements; the diagram shows that the
value of the second element has been reassigned from 123 to 5.
{\tt empty} refers to a list with no elements.

\index{item assignment}
\index{assignment!item}

List indices work the same way as string indices:

\begin{itemize}

\item Any integer expression can be used as an index.

\item If you try to read or write an element that does not exist, you
get an {\tt IndexError}.

\index{exception!IndexError}
\index{IndexError}

\item If an index has a negative value, it counts backward from the
end of the list.

\end{itemize}

\index{list!index}


\index{list!membership}
\index{membership!list}
\index{in operator}
\index{operator!in}

The {\tt in} operator also works on lists.

\beforeverb
\begin{verbatim}
>>> cheeses = ['Cheddar', 'Edam', 'Gouda']
>>> 'Edam' in cheeses
True
>>> 'Brie' in cheeses
False
\end{verbatim}
\afterverb


\section{Traversing a list}
\index{list!traversal}
\index{traversal!list}
\index{for loop}
\index{loop!for}
\index{statement!for}

The most common way to traverse the elements of a list is
with a {\tt for} loop.  The syntax is the same as for strings:

\beforeverb
\begin{verbatim}
for cheese in cheeses:
    print cheese
\end{verbatim}
\afterverb
%
This works well if you only need to read the elements of the
list.  But if you want to write or update the elements, you
need the indices.  A common way to do that is to combine
the functions {\tt range} and {\tt len}:

\index{looping!with indices}
\index{index!looping with}

\beforeverb
\begin{verbatim}
for i in range(len(numbers)):
    numbers[i] = numbers[i] * 2
\end{verbatim}
\afterverb
%
This loop traverses the list and updates each element.  {\tt len}
returns the number of elements in the list.  {\tt range} returns
a list of indices from 0 to $n-1$, where $n$ is the length of
the list.  Each time through the loop {\tt i} gets the index
of the next element.  The assignment statement in the body uses
{\tt i} to read the old value of the element and to assign the
new value.

\index{item update}
\index{update!item}

A {\tt for} loop over an empty list never executes the body:

\beforeverb
\begin{verbatim}
for x in empty:
    print 'This never happens.'
\end{verbatim}
\afterverb
%
Although a list can contain another list, the nested
list still counts as a single element.  The length of this list is
four:

\index{nested list}
\index{list!nested}

\beforeverb
\begin{verbatim}
['spam', 1, ['Brie', 'Roquefort', 'Pol le Veq'], [1, 2, 3]]
\end{verbatim}
\afterverb



\section{List operations}
\index{list!operation}

The {\tt +} operator concatenates lists:

\index{concatenation!list}
\index{list!concatenation}

\beforeverb
\begin{verbatim}
>>> a = [1, 2, 3]
>>> b = [4, 5, 6]
>>> c = a + b
>>> print c
[1, 2, 3, 4, 5, 6]
\end{verbatim}
\afterverb
%
Similarly, the {\tt *} operator repeats a list a given number of times:

\index{repetition!list}
\index{list!repetition}

\beforeverb
\begin{verbatim}
>>> [0] * 4
[0, 0, 0, 0]
>>> [1, 2, 3] * 3
[1, 2, 3, 1, 2, 3, 1, 2, 3]
\end{verbatim}
\afterverb
%
The first example repeats {\tt [0]} four times.  The second example
repeats the list {\tt [1, 2, 3]} three times.


\section{List slices}

\index{slice operator}
\index{operator!slice}
\index{index!slice}
\index{list!slice}
\index{slice!list}

The slice operator also works on lists:

\beforeverb
\begin{verbatim}
>>> t = ['a', 'b', 'c', 'd', 'e', 'f']
>>> t[1:3]
['b', 'c']
>>> t[:4]
['a', 'b', 'c', 'd']
>>> t[3:]
['d', 'e', 'f']
\end{verbatim}
\afterverb
%
If you omit the first index, the slice starts at the beginning.
If you omit the second, the slice goes to the end.  So if you
omit both, the slice is a copy of the whole list.

\index{list!copy}
\index{slice!copy}
\index{copy!slice}

\beforeverb
\begin{verbatim}
>>> t[:]
['a', 'b', 'c', 'd', 'e', 'f']
\end{verbatim}
\afterverb
%
Since lists are mutable, it is often useful to make a copy
before performing operations that fold, spindle or mutilate
lists.

\index{mutability}

A slice operator on the left side of an assignment
can update multiple elements:

\index{slice!update}
\index{update!slice}

\beforeverb
\begin{verbatim}
>>> t = ['a', 'b', 'c', 'd', 'e', 'f']
>>> t[1:3] = ['x', 'y']
>>> print t
['a', 'x', 'y', 'd', 'e', 'f']
\end{verbatim}
\afterverb
%

% You can add elements to a list by squeezing them into an empty
% slice:

% \beforeverb
% \begin{verbatim}
% >>> t = ['a', 'd', 'e', 'f']
% >>> t[1:1] = ['b', 'c']
% >>> print t
% ['a', 'b', 'c', 'd', 'e', 'f']
% \end{verbatim}
% \afterverb
%
% And you can remove elements from a list by assigning the empty list to
% them:

% \beforeverb
% \begin{verbatim}
% >>> t = ['a', 'b', 'c', 'd', 'e', 'f']
% >>> t[1:3] = []
% >>> print t
% ['a', 'd', 'e', 'f']
% \end{verbatim}
% \afterverb
%
% But both of those operations can be expressed more clearly
% with list methods.


\section{List methods}

\index{list!method}
\index{method, list}

Python provides methods that operate on lists.  For example,
{\tt append} adds a new element to the end of a list:

\index{append method}
\index{method!append}

\beforeverb
\begin{verbatim}
>>> t = ['a', 'b', 'c']
>>> t.append('d')
>>> print t
['a', 'b', 'c', 'd']
\end{verbatim}
\afterverb
%
{\tt extend} takes a list as an argument and appends all of
the elements:

\index{extend method}
\index{method!extend}

\beforeverb
\begin{verbatim}
>>> t1 = ['a', 'b', 'c']
>>> t2 = ['d', 'e']
>>> t1.extend(t2)
>>> print t1
['a', 'b', 'c', 'd', 'e']
\end{verbatim}
\afterverb
%
This example leaves {\tt t2} unmodified.

{\tt sort} arranges the elements of the list from low to high:

\index{sort method}
\index{method!sort}

\beforeverb
\begin{verbatim}
>>> t = ['d', 'c', 'e', 'b', 'a']
>>> t.sort()
>>> print t
['a', 'b', 'c', 'd', 'e']
\end{verbatim}
\afterverb
%
List methods are all void; they modify the list and return {\tt None}.
If you accidentally write {\tt t = t.sort()}, you will be disappointed
with the result.

\index{void method}
\index{method!void}
\index{None special value}
\index{special value!None}


\section{Map, filter and reduce}

To add up all the numbers in a list, you can use a loop like this:

% see add.py

\beforeverb
\begin{verbatim}
def add_all(t):
    total = 0
    for x in t:
        total += x
    return total
\end{verbatim}
\afterverb
%
{\tt total} is initialized to 0.  Each time through the loop,
{\tt x} gets one element from the list.  The {\tt +=} operator
provides a short way to update a variable:

\index{update operator}
\index{operator!update}

\beforeverb
\begin{verbatim}
    total += x
\end{verbatim}
\afterverb
%
is equivalent to:

\beforeverb
\begin{verbatim}
    total = total + x
\end{verbatim}
\afterverb
%
As the loop executes, {\tt total} accumulates the sum of the
elements; a variable used this way is sometimes called an
{\bf accumulator}.

\index{accumulator!sum}

Adding up the elements of a list is such a common operation
that Python provides it as a built-in function, {\tt sum}:

\beforeverb
\begin{verbatim}
>>> t = [1, 2, 3]
>>> sum(t)
6
\end{verbatim}
\afterverb
%
An operation like this that combines a sequence of elements into
a single value is sometimes called {\bf reduce}.

\index{reduce pattern}
\index{pattern!reduce}
\index{traversal}

Sometimes you want to traverse one list while building
another.  For example, the following function takes a list of strings
and returns a new list that contains capitalized strings:

\beforeverb
\begin{verbatim}
def capitalize_all(t):
    res = []
    for s in t:
        res.append(s.capitalize())
    return res
\end{verbatim}
\afterverb
%
{\tt res} is initialized with an empty list; each time through
the loop, we append the next element.  So {\tt res} is another
kind of accumulator.

\index{accumulator!list}

An operation like \verb"capitalize_all" is sometimes called a {\bf
map} because it ``maps'' a function (in this case the method {\tt
capitalize}) onto each of the elements in a sequence.

\index{map pattern}
\index{pattern!map}
\index{filter pattern}
\index{pattern!filter}

Another common operation is to select some of the elements from
a list and return a sublist.  For example, the following
function takes a list of strings and returns a list that contains
only the uppercase strings:

\beforeverb
\begin{verbatim}
def only_upper(t):
    res = []
    for s in t:
        if s.isupper():
            res.append(s)
    return res
\end{verbatim}
\afterverb
%
{\tt isupper} is a string method that returns {\tt True} if
the string contains only upper case letters.

An operation like \verb"only_upper" is called a {\bf filter} because
it selects some of the elements and filters out the others.

Most common list operations can be expressed as a combination
of map, filter and reduce.  Because these operations are
so common, Python provides language features to support them,
including the built-in function {\tt map} and an operator
called a ``list comprehension.''

\index{list!comprehension}

\begin{ex}
\label{cumulative}
\index{cumulative sum}

Write a function that takes a list of numbers and returns the
cumulative sum; that is, a new list where the $i$th element
is the sum of the first $i+1$ elements from the original list.
For example, the cumulative sum of {\tt [1, 2, 3]} is
{\tt [1, 3, 6]}. 
\end{ex}


\section{Deleting elements}

\index{element deletion}
\index{deletion, element of list}

There are several ways to delete elements from a list.  If you
know the index of the element you want, you can use
{\tt pop}:

\index{pop method}
\index{method!pop}

\beforeverb
\begin{verbatim}
>>> t = ['a', 'b', 'c']
>>> x = t.pop(1)
>>> print t
['a', 'c']
>>> print x
b
\end{verbatim}
\afterverb
%
{\tt pop} modifies the list and returns the element that was removed.
If you don't provide an index, it deletes and returns the
last element.

If you don't need the removed value, you can use the {\tt del}
operator:

\index{del operator}
\index{operator!del}

\beforeverb
\begin{verbatim}
>>> t = ['a', 'b', 'c']
>>> del t[1]
>>> print t
['a', 'c']
\end{verbatim}
\afterverb
%

If you know the element you want to remove (but not the index), you
can use {\tt remove}:

\index{remove method}
\index{method!remove}

\beforeverb
\begin{verbatim}
>>> t = ['a', 'b', 'c']
>>> t.remove('b')
>>> print t
['a', 'c']
\end{verbatim}
\afterverb
%
The return value from {\tt remove} is {\tt None}.

\index{None special value}
\index{special value!None}

To remove more than one element, you can use {\tt del} with
a slice index:

\beforeverb
\begin{verbatim}
>>> t = ['a', 'b', 'c', 'd', 'e', 'f']
>>> del t[1:5]
>>> print t
['a', 'f']
\end{verbatim}
\afterverb
%
As usual, the slice selects all the elements up to, but not
including, the second index.


\section{Lists and strings}

\index{list}
\index{string}
\index{sequence}

A string is a sequence of characters and a list is a sequence
of values, but a list of characters is not the same as a
string.  To convert from a string to a list of characters,
you can use {\tt list}:

\index{list!function}
\index{function!list}

\beforeverb
\begin{verbatim}
>>> s = 'spam'
>>> t = list(s)
>>> print t
['s', 'p', 'a', 'm']
\end{verbatim}
\afterverb
%
Because {\tt list} is the name of a built-in function, you should
avoid using it as a variable name.  I also avoid {\tt l} because
it looks too much like {\tt 1}.  So that's why I use {\tt t}.

The {\tt list} function breaks a string into individual letters.  If
you want to break a string into words, you can use the {\tt split}
method:

\index{split method}
\index{method!split}

\beforeverb
\begin{verbatim}
>>> s = 'pining for the fjords'
>>> t = s.split()
>>> print t
['pining', 'for', 'the', 'fjords']
\end{verbatim}
\afterverb
%
An optional argument called a {\bf delimiter} specifies which
characters to use as word boundaries.
The following example
uses a hyphen as a delimiter:

\index{optional argument}
\index{argument!optional}
\index{delimiter}

\beforeverb
\begin{verbatim}
>>> s = 'spam-spam-spam'
>>> delimiter = '-'
>>> s.split(delimiter)
['spam', 'spam', 'spam']
\end{verbatim}
\afterverb
%
{\tt join} is the inverse of {\tt split}.  It
takes a list of strings and
concatenates the elements.  {\tt join} is a string method,
so you have to invoke it on the delimiter and pass the
list as a parameter:

\index{join method}
\index{method!join}
\index{concatenation}

\beforeverb
\begin{verbatim}
>>> t = ['pining', 'for', 'the', 'fjords']
>>> delimiter = ' '
>>> delimiter.join(t)
'pining for the fjords'
\end{verbatim}
\afterverb
%
In this case the delimiter is a space character, so
{\tt join} puts a space between words.  To concatenate
strings without spaces, you can use the empty string,
\verb"''", as a delimiter. 

\index{empty string}
\index{string!empty}


\section{Objects and values}

\index{object}
\index{value}

If we execute these assignment statements:

\beforeverb
\begin{verbatim}
a = 'banana'
b = 'banana'
\end{verbatim}
\afterverb
%
We know that {\tt a} and {\tt b} both refer to a
string, but we don't
know whether they refer to the {\em same} string.
There are two possible states:

\index{aliasing}

\beforefig
\centerline{\includegraphics{figs/list1.eps}}
\afterfig

In one case, {\tt a} and {\tt b} refer to two different objects that
have the same value.  In the second case, they refer to the same
object.

\index{is operator}
\index{operator!is}

To check whether two variables refer to the same object, you can
use the {\tt is} operator.

\beforeverb
\begin{verbatim}
>>> a = 'banana'
>>> b = 'banana'
>>> a is b
True
\end{verbatim}
\afterverb
%
In this example, Python only created one string object,
and both {\tt a} and {\tt b} refer to it.

But when you create two lists, you get two objects:

\beforeverb
\begin{verbatim}
>>> a = [1, 2, 3]
>>> b = [1, 2, 3]
>>> a is b
False
\end{verbatim}
\afterverb
%
So the state diagram looks like this:

\index{state diagram}
\index{diagram!state}

\beforefig
\centerline{\includegraphics{figs/list2.eps}}
\afterfig

In this case we would say that the two lists are {\bf equivalent},
because they have the same elements, but not {\bf identical}, because
they are not the same object.  If two objects are identical, they are
also equivalent, but if they are equivalent, they are not necessarily
identical.

\index{equivalence}
\index{identity}

Until now, we have been using ``object'' and ``value''
interchangeably, but it is more precise to say that an object has a
value.  If you execute {\tt a = [1,2,3]}, {\tt a} refers to a list
object whose value is a particular sequence of elements.  If another
list has the same elements, we would say it has the same value.

\index{object}
\index{value}


\section{Aliasing}

\index{aliasing}
\index{reference!aliasing}

If {\tt a} refers to an object and you assign {\tt b = a},
then both variables refer to the same object:

\beforeverb
\begin{verbatim}
>>> a = [1, 2, 3]
>>> b = a
>>> b is a
True
\end{verbatim}
\afterverb
%
The state diagram looks like this:

\index{state diagram}
\index{diagram!state}

\beforefig
\centerline{\includegraphics{figs/list3.eps}}
\afterfig

The association of a variable with an object is called a {\bf
reference}.  In this example, there are two references to the same
object.

\index{reference}

An object with more than one reference has more
than one name, so we say that the object is {\bf aliased}.

\index{mutability}

If the aliased object is mutable, 
changes made with one alias affect
the other:

\beforeverb
\begin{verbatim}
>>> b[0] = 17
>>> print a
[17, 2, 3]
\end{verbatim}
\afterverb
%
Although this behavior can be useful, it is error-prone.  In general,
it is safer to avoid aliasing when you are working with mutable
objects.

\index{immutability}

For immutable objects like strings, aliasing is not as much of a
problem.  In this example:

\beforeverb
\begin{verbatim}
a = 'banana'
b = 'banana'
\end{verbatim}
\afterverb
%
It almost never makes a difference whether {\tt a} and {\tt b} refer
to the same string or not.


\section{List arguments}

\index{list!as argument}
\index{argument}
\index{argument!list}
\index{reference}
\index{parameter}

When you pass a list to a function, the function gets a reference
to the list.
If the function modifies a list parameter, the caller sees the change.
For example, \verb"delete_head" removes the first element from a list:

\beforeverb
\begin{verbatim}
def delete_head(t):
    del t[0]
\end{verbatim}
\afterverb
%
Here's how it is used:

\beforeverb
\begin{verbatim}
>>> letters = ['a', 'b', 'c']
>>> delete_head(letters)
>>> print letters
['b', 'c']
\end{verbatim}
\afterverb
%
The parameter {\tt t} and the variable {\tt letters} are
aliases for the same object.  The stack diagram looks like
this:

\index{stack diagram}
\index{diagram!stack}

\beforefig
\centerline{\includegraphics{figs/stack5.eps}}
\afterfig

Since the list is shared by two frames, I drew
it between them.

It is important to distinguish between operations that
modify lists and operations that create new lists.  For
example, the {\tt append} method modifies a list, but the
{\tt +} operator creates a new list:

\index{append method}
\index{method!append}
\index{list!concatenation}
\index{concatenation!list}

\beforeverb
\begin{verbatim}
>>> t1 = [1, 2]
>>> t2 = t1.append(3)
>>> print t1
[1, 2, 3]
>>> print t2
None

>>> t3 = t1 + [3]
>>> print t3
[1, 2, 3]
>>> t2 is t3
False
\end{verbatim}
\afterverb

This difference is important when you write functions that
are supposed to modify lists.  For example, this function
{\em does not} delete the head of a list:

\beforeverb
\begin{verbatim}
def bad_delete_head(t):
    t = t[1:]              # WRONG!
\end{verbatim}
\afterverb

The slice operator creates a new list and the assignment
makes {\tt t} refer to it, but none of that has any effect
on the list that was passed as an argument.

\index{slice operator}
\index{operator!slice}

An alternative is to write a function that creates and
returns a new list.  For
example, {\tt tail} returns all but the first
element of a list:

\beforeverb
\begin{verbatim}
def tail(t):
    return t[1:]
\end{verbatim}
\afterverb
%
This function leaves the original list unmodified.
Here's how it is used:

\beforeverb
\begin{verbatim}
>>> letters = ['a', 'b', 'c']
>>> rest = tail(letters)
>>> print rest
['b', 'c']
\end{verbatim}
\afterverb


\begin{ex}

Write a function called {\tt chop} that takes a list and modifies
it, removing the first and last elements, and returns {\tt None}.

Then write a function called {\tt middle} that takes a list and
returns a new list that contains all but the first and last
elements.

\end{ex}


\section{Debugging}
\index{debugging}

Careless use of lists (and other mutable objects)
can lead to long hours of debugging.  Here are some common
pitfalls and ways to avoid them:

\begin{enumerate}

\item Don't forget that most list methods modify the argument and
  return {\tt None}.  This is the opposite of the string methods,
  which return a new string and leave the original alone.

If you are used to writing string code like this:

\beforeverb
\begin{verbatim}
word = word.strip()
\end{verbatim}
\afterverb

It is tempting to write list code like this:

\beforeverb
\begin{verbatim}
t = t.sort()           # WRONG!
\end{verbatim}
\afterverb

\index{sort method}
\index{method!sort}

Because {\tt sort} returns {\tt None}, the
next operation you perform with {\tt t} is likely to fail.

Before using list methods and operators, you should read the
documentation carefully and then test them in interactive mode.  The
methods and operators that lists share with other sequences (like
strings) are documented at
\url{docs.python.org/lib/typesseq.html}.  The
methods and operators that only apply to mutable sequences
are documented at \url{docs.python.org/lib/typesseq-mutable.html}.


\item Pick an idiom and stick with it.

Part of the problem with lists is that there are too many
ways to do things.  For example, to remove an element from
a list, you can use {\tt pop}, {\tt remove}, {\tt del},
or even a slice assignment.

To add an element, you can use the {\tt append} method or
the {\tt +} operator.  But don't forget that these are right: 

\beforeverb
\begin{verbatim}
t.append(x)
t = t + [x]
\end{verbatim}
\afterverb

And these are wrong:

\beforeverb
\begin{verbatim}
t.append([x])          # WRONG!
t = t.append(x)        # WRONG!
t + [x]                # WRONG!
t = t + x              # WRONG!
\end{verbatim}
\afterverb

Try out each of these examples in interactive mode to make sure
you understand what they do.  Notice that only the last
one causes a runtime error; the other three are legal, but they
do the wrong thing.


\item Make copies to avoid aliasing.

\index{aliasing!copying to avoid}
\index{copy!to avoid aliasing}

If you want to use a method like {\tt sort} that modifies
the argument, but you need to keep the original list as
well, you can make a copy.

\beforeverb
\begin{verbatim}
orig = t[:]
t.sort()
\end{verbatim}
\afterverb

In this example you could also use the built-in function {\tt sorted},
which returns a new, sorted list and leaves the original alone.
But in that case you should avoid using {\tt sorted} as a variable
name!

\end{enumerate}



\section{Glossary}

\begin{description}

\item[list:] A sequence of values.
\index{list}

\item[element:] One of the values in a list (or other sequence),
also called items.
\index{element}

\item[index:] An integer value that indicates an element in a list.
\index{index}

\item[nested list:] A list that is an element of another list.
\index{nested list}

\item[list traversal:] The sequential accessing of each element in a list.
\index{list!traversal}

\item[mapping:] A relationship in which each element of one set
corresponds to an element of another set.  For example, a list is
a mapping from indices to elements.
\index{mapping}

\item[accumulator:] A variable used in a loop to add up or
accumulate a result.
\index{accumulator}

\index{traversal}

\item[reduce:] A processing pattern that traverses a sequence 
and accumulates the elements into a single result.
\index{reduce pattern}
\index{pattern!reduce}

\item[map:] A processing pattern that traverses a sequence and
performs an operation on each element.
\index{map pattern}
\index{pattern!map}

\item[filter:] A processing pattern that traverses a list and
selects the elements that satisfy some criterion.
\index{filter pattern}
\index{pattern!filter}

\item[object:] Something a variable can refer to.  An object
has a type and a value.
\index{object}

\item[equivalent:] Having the same value.
\index{equivalent}

\item[identical:] Being the same object (which implies equivalence).
\index{identical}

\item[reference:] The association between a variable and its value.
\index{reference}

\item[aliasing:] A circumstance where two or more variables refer to the same
object.
\index{aliasing}

\item[delimiter:] A character or string used to indicate where a
string should be split.
\index{delimiter}

\end{description}


\section{Exercises}

\begin{ex}
Write a function called \verb"is_sorted" that takes a list as a
parameter and returns {\tt True} if the list is sorted in ascending
order and {\tt False} otherwise.  You can assume (as a precondition)
that the elements of the list can be compared with the comparison
operators {\tt <}, {\tt >}, etc.

\index{precondition}

For example, \verb"is_sorted([1,2,2])" should return {\tt True}
and \verb"is_sorted(['b','a'])" should return {\tt False}.
\end{ex}


\begin{ex}
\label{anagram}

\index{anagram}

Two words are anagrams if you can rearrange the letters from one
to spell the other.  Write a function called \verb"is_anagram"
that takes two strings and returns {\tt True} if they are anagrams.
\end{ex}


\begin{ex}
\label{duplicate}

The (so-called) Birthday Paradox:

\begin{enumerate}

\index{birthday paradox}
\index{duplicate}

\item Write a function called \verb"has_duplicates" that takes
a list and returns {\tt True} if there is any element that
appears more than once.  It should not modify the original
list.

\item If there are 23 students in your class, what are the chances
that two of you have the same birthday?  You can estimate this
probability by generating random samples of 23 birthdays
and checking for matches.  Hint: you can generate random birthdays
with the {\tt randint} function in the {\tt random} module.

\index{random module}
\index{module!random}
\index{randint function}
\index{function!randint}

\end{enumerate}

You can read about this problem at
\url{wikipedia.org/wiki/Birthday_paradox}, and you can see my solution
at \url{thinkpython.com/code/birthday.py}.

\end{ex}


\begin{ex}

\index{duplicate}
\index{uniqueness}

Write a function called \verb"remove_duplicates" that takes
a list and returns a new list with only the unique elements from
the original.  Hint: they don't have to be in the same order.
\end{ex}


\begin{ex}
\index{append method}
\index{method append}
\index{list!concatenation}
\index{concatenation!list}

Write a function that reads the file {\tt words.txt} and builds
a list with one element per word.  Write two versions of
this function, one using the {\tt append} method and the
other using the idiom {\tt t = t + [x]}.  Which one takes
longer to run?  Why?

You can see my solution at \url{thinkpython.com/code/wordlist.py}.
\end{ex}


\begin{ex}
\label{wordlist1}
\label{bisection}

\index{membership!bisection search}
\index{bisection search}
\index{search, bisection}

To check whether a word is in the word list, you could use
the {\tt in} operator, but it would be slow because it searches
through the words in order.

Because the words are in alphabetical order, we can speed things up
with a bisection search, which is similar to what you do when you look
a word up in the dictionary.  You start in the middle and check to see
whether the word you are looking for comes before the word in the
middle of the list.  If so, then you search the first half of the list
the same way.  Otherwise you search the second half.

Either way, you cut the remaining search space in half.  If the
word list has 113,809 words, it will take about 17 steps to
find the word or conclude that it's not there.

Write a function called {\tt bisect} that takes a sorted list
and a target value and returns the index of the value
in the list, if it's there, or {\tt None} if it's not.

\index{bisect module}
\index{module!bisect}

Or you could read the documentation of the {\tt bisect} module
and use that!
\end{ex}

\begin{ex}
\index{reverse word pair}

Two words are a ``reverse pair'' if each is the reverse of the
other.  Write a program that finds all the reverse pairs in the
word list. 
\end{ex}

\begin{ex}
\index{interlocking words}

Two words ``interlock'' if taking alternating letters from each forms
a new word\footnote{This exercise is inspired by an example at
  \url{puzzlers.org}.}.  For example, ``shoe'' and ``cold''
interlock to form ``schooled.''

\begin{enumerate}

\item Write a program that finds all pairs of words that interlock.
  Hint: don't enumerate all pairs!

\item Can you find any words that are three-way interlocked; that is,
  every third letter forms a word, starting from the first, second or
  third?

\end{enumerate}
\end{ex}


\chapter{Dictionaries}
\index{dictionary}

\index{dictionary}
\index{type!dict}
\index{key}
\index{key-value pair}
\index{index}

A {\bf dictionary} is like a list, but more general.  In a list,
the indices have to be integers; in a dictionary they can
be (almost) any type.

You can think of a dictionary as a mapping between a set of indices
(which are called {\bf keys}) and a set of values.  Each key maps to a
value.  The association of a key and a value is called a {\bf
  key-value pair} or sometimes an {\bf item}.

As an example, we'll build a dictionary that maps from English
to Spanish words, so the keys and the values are all strings.

The function {\tt dict} creates a new dictionary with no items.
Because {\tt dict} is the name of a built-in function, you
should avoid using it as a variable name.

\index{dict function}
\index{function!dict}

\beforeverb
\begin{verbatim}
>>> eng2sp = dict()
>>> print eng2sp
{}
\end{verbatim}
\afterverb

The squiggly-brackets, \verb"{}", represent an empty dictionary.
To add items to the dictionary, you can use square brackets:

\index{squiggly bracket}
\index{bracket!squiggly}

\beforeverb
\begin{verbatim}
>>> eng2sp['one'] = 'uno'
\end{verbatim}
\afterverb
%
This line creates an item that maps from the key
{\tt 'one'} to the value \verb"'uno'".  If we print the
dictionary again, we see a key-value pair with a colon
between the key and value:

\beforeverb
\begin{verbatim}
>>> print eng2sp
{'one': 'uno'}
\end{verbatim}
\afterverb
%
This output format is also an input format.  For example,
you can create a new dictionary with three items:

\beforeverb
\begin{verbatim}
>>> eng2sp = {'one': 'uno', 'two': 'dos', 'three': 'tres'}
\end{verbatim}
\afterverb
%
But if you print {\tt eng2sp}, you might be surprised:

\beforeverb
\begin{verbatim}
>>> print eng2sp
{'one': 'uno', 'three': 'tres', 'two': 'dos'}
\end{verbatim}
\afterverb
%
The order of the key-value pairs is not the same.  In fact, if
you type the same example on your computer, you might get a
different result.  In general, the order of items in
a dictionary is unpredictable.

But that's not a problem because
the elements of a dictionary are never indexed with integer indices.
Instead, you use the keys to look up the corresponding values:

\beforeverb
\begin{verbatim}
>>> print eng2sp['two']
'dos'
\end{verbatim}
\afterverb
%
The key {\tt 'two'} always maps to the value \verb"'dos'" so the order
of the items doesn't matter.

If the key isn't in the dictionary, you get an exception:

\index{exception!KeyError}
\index{KeyError}

\beforeverb
\begin{verbatim}
>>> print eng2sp['four']
KeyError: 'four'
\end{verbatim}
\afterverb
%
The {\tt len} function works on dictionaries; it returns the
number of key-value pairs:

\index{len function}
\index{function!len}

\beforeverb
\begin{verbatim}
>>> len(eng2sp)
3
\end{verbatim}
\afterverb
%
The {\tt in} operator works on dictionaries; it tells you whether
something appears as a {\em key} in the dictionary (appearing
as a value is not good enough).

\index{membership!dictionary}
\index{in operator}
\index{operator!in}

\beforeverb
\begin{verbatim}
>>> 'one' in eng2sp
True
>>> 'uno' in eng2sp
False
\end{verbatim}
\afterverb
%
To see whether something appears as a value in a dictionary, you
can use the method {\tt values}, which returns the values as
a list, and then use the {\tt in} operator:

\index{values method}
\index{method!values}

\beforeverb
\begin{verbatim}
>>> vals = eng2sp.values()
>>> 'uno' in vals
True
\end{verbatim}
\afterverb
%
The {\tt in} operator uses different algorithms for lists and
dictionaries.  For lists, it uses a search algorithm, as in
Section~\ref{find}.  As the list gets longer, the search time gets
longer in direct proportion.  For dictionaries, Python uses an
algorithm called a {\bf hashtable} that has a remarkable property: the
{\tt in} operator takes about the same amount of time no matter how
many items there are in a dictionary.  I won't explain how that's
possible, but you can read more about it at
\url{wikipedia.org/wiki/Hash_table}.

\index{hashtable}

\begin{ex}
\label{wordlist2}

\index{set membership}
\index{membership!set}

Write a function that reads the words in {\tt words.txt} and
stores them as keys in a dictionary.  It doesn't matter what the
values are.  Then you can use the {\tt in} operator
as a fast way to check whether a string is in
the dictionary.

If you did Exercise~\ref{wordlist1}, you can compare the speed
of this implementation with the list {\tt in} operator and the
bisection search.

\end{ex}


\section{Dictionary as a set of counters}
\label{histogram}

\index{counter}

Suppose you are given a string and you want to count how many
times each letter appears.  There are several ways you could do it:

\begin{enumerate}

\item You could create 26 variables, one for each letter of the
alphabet.  Then you could traverse the string and, for each
character, increment the corresponding counter, probably using
a chained conditional.

\item You could create a list with 26 elements.  Then you could
convert each character to a number (using the built-in function
{\tt ord}), use the number as an index into the list, and increment
the appropriate counter.

\item You could create a dictionary with characters as keys
and counters as the corresponding values.  The first time you
see a character, you would add an item to the dictionary.  After
that you would increment the value of an existing item.

\end{enumerate}

Each of these options performs the same computation, but each
of them implements that computation in a different way.

\index{implementation}

An {\bf implementation} is a way of performing a computation;
some implementations are better than others.  For example,
an advantage of the dictionary implementation is that we don't
have to know ahead of time which letters appear in the string
and we only have to make room for the letters that do appear.

Here is what the code might look like:

\beforeverb
\begin{verbatim}
def histogram(s):
    d = dict()
    for c in s:
        if c not in d:
            d[c] = 1
        else:
            d[c] += 1
    return d
\end{verbatim}
\afterverb
%
The name of the function is {\bf histogram}, which is a statistical
term for a set of counters (or frequencies).

\index{histogram}
\index{frequency}
\index{traversal}

The first line of the
function creates an empty dictionary.  The {\tt for} loop traverses
the string.  Each time through the loop, if the character {\tt c} is
not in the dictionary, we create a new item with key {\tt c} and the
initial value 1 (since we have seen this letter once).  If {\tt c} is
already in the dictionary we increment {\tt d[c]}.

\index{histogram}

Here's how it works:

\beforeverb
\begin{verbatim}
>>> h = histogram('brontosaurus')
>>> print h
{'a': 1, 'b': 1, 'o': 2, 'n': 1, 's': 2, 'r': 2, 'u': 2, 't': 1}
\end{verbatim}
\afterverb
%
The histogram indicates that the letters {\tt 'a'} and \verb"'b'"
appear once; \verb"'o'" appears twice, and so on.

\begin{ex}

\index{get method}
\index{method!get}

Dictionaries have a method called {\tt get} that takes a key
and a default value.  If the key appears in the dictionary,
{\tt get} returns the corresponding value; otherwise it returns
the default value.  For example:

\beforeverb
\begin{verbatim}
>>> h = histogram('a')
>>> print h
{'a': 1}
>>> h.get('a', 0)
1
>>> h.get('b', 0)
0
\end{verbatim}
\afterverb
%
Use {\tt get} to write {\tt histogram} more concisely.  You
should be able to eliminate the {\tt if} statement.
\end{ex}


\section{Looping and dictionaries}

\index{dictionary!looping with}
\index{looping!with dictionaries}
\index{traversal}

If you use a dictionary in a {\tt for} statement, it traverses
the keys of the dictionary.  For example, \verb"print_hist"
prints each key and the corresponding value:

\beforeverb
\begin{verbatim}
def print_hist(h):
    for c in h:
        print c, h[c]
\end{verbatim}
\afterverb
%
Here's what the output looks like:

\beforeverb
\begin{verbatim}
>>> h = histogram('parrot')
>>> print_hist(h)
a 1
p 1
r 2
t 1
o 1
\end{verbatim}
\afterverb
%
Again, the keys are in no particular order.

\begin{ex}

\index{keys method}
\index{method!keys}

Dictionaries have a method called {\tt keys} that returns
the keys of the dictionary, in no particular order, as a list.

Modify \verb"print_hist" to print the keys and their values
in alphabetical order.
\end{ex}



\section{Reverse lookup}

\index{dictionary!lookup}
\index{dictionary!reverse lookup}
\index{lookup, dictionary}
\index{reverse lookup, dictionary}

Given a dictionary {\tt d} and a key {\tt k}, it is easy to
find the corresponding value {\tt v = d[k]}.  This operation
is called a {\bf lookup}.

But what if you have {\tt v} and you want to find {\tt k}?
You have two problems: first, there might be more than one
key that maps to the value {\tt v}.  Depending on the application,
you might be able to pick one, or you might have to make
a list that contains all of them.  Second, there is no
simple syntax to do a {\bf reverse lookup}; you have to search.

Here is a function that takes a value and returns the first
key that maps to that value:

\beforeverb
\begin{verbatim}
def reverse_lookup(d, v):
    for k in d:
        if d[k] == v:
            return k
    raise ValueError
\end{verbatim}
\afterverb
%
This function is yet another example of the search pattern, but it
uses a feature we haven't seen before, {\tt raise}.  The {\tt raise}
statement causes an exception; in this case it causes a {\tt
  ValueError}, which generally indicates that there is something wrong
with the value of a parameter.

\index{search}
\index{pattern!search}
\index{raise statement}
\index{statement!raise}
\index{exception!ValueError}
\index{ValueError}

If we get to the end of the loop, that means {\tt v}
doesn't appear in the dictionary as a value, so we raise an
exception.

Here is an example of a successful reverse lookup:

\beforeverb
\begin{verbatim}
>>> h = histogram('parrot')
>>> k = reverse_lookup(h, 2)
>>> print k
r
\end{verbatim}
\afterverb
%
And an unsuccessful one:

\beforeverb
\begin{verbatim}
>>> k = reverse_lookup(h, 3)
Traceback (most recent call last):
  File "<stdin>", line 1, in ?
  File "<stdin>", line 5, in reverse_lookup
ValueError
\end{verbatim}
\afterverb
%
The result when you raise an exception is the same as when
Python raises one: it prints a traceback and an error message.

\index{traceback}
\index{optional argument}
\index{argument!optional}

The {\tt raise} statement takes a detailed error message as an
optional argument.  For example:

\beforeverb
\begin{verbatim}
>>> raise ValueError, 'value does not appear in the dictionary'
Traceback (most recent call last):
  File "<stdin>", line 1, in ?
ValueError: value does not appear in the dictionary
\end{verbatim}
\afterverb
%
A reverse lookup is much slower than a forward lookup; if you
have to do it often, or if the dictionary gets big, the performance
of your program will suffer.

\begin{ex}
Modify \verb"reverse_lookup" so that it builds and returns a list
of {\em all} keys that map to {\tt v}, or an empty list if there
are none.
\end{ex}


\section{Dictionaries and lists}

Lists can appear as values in a dictionary.  For example, if you
were given a dictionary that maps from letters to frequencies, you
might want to invert it; that is, create a dictionary that maps
from frequencies to letters.  Since there might be several letters
with the same frequency, each value in the inverted dictionary
should be a list of letters.

\index{invert dictionary}
\index{dictionary!invert}

Here is a function that inverts a dictionary:

\beforeverb
\begin{verbatim}
def invert_dict(d):
    inv = dict()
    for key in d:
        val = d[key]
        if val not in inv:
            inv[val] = [key]
        else:
            inv[val].append(key)
    return inv
\end{verbatim}
\afterverb
%
Each time through the loop, {\tt key} gets a key from {\tt d} and 
{\tt val} gets the corresponding value.  If {\tt val} is not in {\tt inv},
that means we haven't seen it before, so we create a new item and
initialize it with a {\bf singleton} (a list that contains a
single element).  Otherwise we have seen this value before, so we
append the corresponding key to the list.

\index{singleton}

Here is an example:

\beforeverb
\begin{verbatim}
>>> hist = histogram('parrot')
>>> print hist
{'a': 1, 'p': 1, 'r': 2, 't': 1, 'o': 1}
>>> inv = invert_dict(hist)
>>> print inv
{1: ['a', 'p', 't', 'o'], 2: ['r']}
\end{verbatim}
\afterverb
%
And here is a diagram showing {\tt hist} and {\tt inv}:

\index{state diagram}
\index{diagram!state}

\beforefig
\centerline{\includegraphics{figs/dict1.eps}}
\afterfig

A dictionary is represented as a box with the type {\tt dict} above it
and the key-value pairs inside.  If the values are integers, floats or
strings, I usually draw them inside the box, but I usually draw lists
outside the box, just to keep the diagram simple.

Lists can be values in a dictionary, as this example shows, but they
cannot be keys.  Here's what happens if you try:

\index{TypeError}
\index{exception!TypeError}


\beforeverb
\begin{verbatim}
>>> t = [1, 2, 3]
>>> d = dict()
>>> d[t] = 'oops'
Traceback (most recent call last):
  File "<stdin>", line 1, in ?
TypeError: list objects are unhashable
\end{verbatim}
\afterverb
%
I mentioned earlier that a dictionary is implemented using
a hashtable and that means that the keys have to be {\bf hashable}.

\index{hash function}
\index{hashable}

A {\bf hash} is a function that takes a value (of any kind)
and returns an integer.  Dictionaries use these integers,
called hash values, to store and look up key-value pairs.

\index{immutability}

This system works fine if the keys are immutable.  But if the
keys are mutable, like lists, bad things happen.  For example,
when you create a key-value pair, Python hashes the key and 
stores it in the corresponding location.  If you modify the
key and then hash it again, it would go to a different location.
In that case you might have two entries for the same key,
or you might not be able to find a key.  Either way, the
dictionary wouldn't work correctly.

That's why the keys have to be hashable, and why mutable types like
lists aren't.  The simplest way to get around this limitation is to
use tuples, which we will see in the next chapter.

Since dictionaries are mutable, they can't be used as keys,
but they {\em can} be used as values.

\begin{ex}
Read the documentation of the dictionary method {\tt setdefault}
and use it to write a more concise version of \verb"invert_dict".

\index{setdefault method}
\index{method!setdefault}

\end{ex}


\section{Memos}

If you played with the {\tt fibonacci} function from
Section~\ref{one more example}, you might have noticed that the bigger
the argument you provide, the longer the function takes to run.
Furthermore, the run time increases very quickly.

\index{fibonacci function}
\index{function!fibonacci}

To understand why, consider this {\bf call graph} for
{\tt fibonacci} with {\tt n=4}:

\beforefig
\centerline{\includegraphics[height=2in]{figs/fibonacci.eps}}
\afterfig

A call graph shows a set of function frames, with lines connecting each
frame to the frames of the functions it calls.  At the top of the
graph, {\tt fibonacci} with {\tt n=4} calls {\tt fibonacci} with {\tt
n=3} and {\tt n=2}.  In turn, {\tt fibonacci} with {\tt n=3} calls
{\tt fibonacci} with {\tt n=2} and {\tt n=1}.  And so on.

\index{function frame}
\index{frame}
\index{call graph}

Count how many times {\tt fibonacci(0)} and {\tt fibonacci(1)} are
called.  This is an inefficient solution to the problem, and it gets
worse as the argument gets bigger.

\index{memo}

One solution is to keep track of values that have already been
computed by storing them in a dictionary.  A previously computed value
that is stored for later use is called a {\bf memo}\footnote{See
  \url{wikipedia.org/wiki/Memoization}}.  Here is an
implementation of {\tt fibonacci} using memos:

\beforeverb
\begin{verbatim}
known = {0:0, 1:1}

def fibonacci(n):
    if n in known:
        return known[n]

    res = fibonacci(n-1) + fibonacci(n-2)
    known[n] = res
    return res
\end{verbatim}
\afterverb
%
{\tt known} is a dictionary that keeps track of the Fibonacci
numbers we already know.  It starts with
two items: 0 maps to 0 and 1 maps to 1.

Whenever {\tt fibonacci} is called, it checks {\tt known}.
If the result is already there, it can return
immediately.  Otherwise it has to 
compute the new value, add it to the dictionary, and return it.

\begin{ex}
Run this version of {\tt fibonacci} and the original with
a range of parameters and compare their run times.
\end{ex}


\section{Global variables}

\index{global variable}
\index{variable!global}

In the previous example, {\tt known} is created outside the function,
so it belongs to the special frame called \verb"__main__".
Variables in \verb"__main__" are sometimes called {\bf global}
because they can be accessed from any function.  Unlike local
variables, which disappear when their function ends, global variables
persist from one function call to the next.

\index{flag}

It is common to use global variables for {\bf flags}; that is, 
boolean variables that indicate (``flag'') whether a condition
is true.  For example, some programs use
a flag named {\tt verbose} to control the level of detail in the
output:

\beforeverb
\begin{verbatim}
verbose = True

def example1():
    if verbose:
        print 'Running example1'
\end{verbatim}
\afterverb
%
If you try to reassign a global variable, you might be surprised.
The following example is supposed to keep track of whether the
function has been called:

\index{multiple assignment}
\index{assignment!multiple}

\beforeverb
\begin{verbatim}
been_called = False

def example2():
    been_called = True         # WRONG
\end{verbatim}
\afterverb
%
But if you run it you will see that the value of \verb"been_called"
doesn't change.  The problem is that {\tt example2} creates a new local
variable named \verb"been_called".  The local variable goes away when
the function ends, and has no effect on the global variable.

\index{global statement}
\index{statement!global}
\index{declaration}

To reassign a global variable inside a function you have to
{\bf declare} the global variable before you use it:

\beforeverb
\begin{verbatim}
been_called = False

def example2():
    global been_called 
    been_called = True
\end{verbatim}
\afterverb
%
The {\tt global} statement tells the interpreter
something like, ``In this function, when I say \verb"been_called", I
mean the global variable; don't create a local one.''

\index{update!global variable}
\index{global variable!update}

Here's an example that tries to update a global variable:

\beforeverb
\begin{verbatim}
count = 0

def example3():
    count = count + 1          # WRONG
\end{verbatim}
\afterverb
%
If you run it you get:

\index{UnboundLocalError}
\index{exception!UnboundLocalError}

\beforeverb
\begin{verbatim}
UnboundLocalError: local variable 'count' referenced before assignment
\end{verbatim}
\afterverb
%
Python assumes that {\tt count} is local, which means
that you are reading it before writing it.  The solution, again,
is to declare {\tt count} global.

\index{counter}

\beforeverb
\begin{verbatim}
def example3():
    global count
    count += 1
\end{verbatim}
\afterverb
%
If the global value is mutable, you can modify it without
declaring it:

\index{mutability}

\beforeverb
\begin{verbatim}
known = {0:0, 1:1}

def example4():
    known[2] = 1
\end{verbatim}
\afterverb
%
So you can add, remove and replace elements of a global list or
dictionary, but if you want to reassign the variable, you
have to declare it:

\beforeverb
\begin{verbatim}
def example5():
    global known
    known = dict()
\end{verbatim}
\afterverb
%

\section{Long integers}

\index{long integer}
\index{integer!long}
\index{type!long}

If you compute {\tt fibonacci(50)}, you get:

\beforeverb
\begin{verbatim}
>>> fibonacci(50)
12586269025L
\end{verbatim}
\afterverb
%
The {\tt L} at the end indicates that the result is a long
integer\footnote{In Python 3.0, type {\tt long} is gone; all integers,
  even really big ones, are type {\tt int}.}, or type {\tt long}.

\index{Python 3.0}

Values with type {\tt int} have a limited range;
long integers can be arbitrarily big, but as they get bigger
they consume more space and time.

The mathematical operators work on long integers, and the functions
in the {\tt math} module, too, so in general any code that
works with {\tt int} will also work with {\tt long}.

Any time the result of a computation is too big to be represented with
an integer, Python converts the result as a long integer:

\beforeverb
\begin{verbatim}
>>> 1000 * 1000
1000000
>>> 100000 * 100000
10000000000L
\end{verbatim}
\afterverb
%
In the first case the result has type {\tt int}; in the
second case it is {\tt long}.

\begin{ex}

\index{encryption}
\index{RSA algorithm}
\index{algorithm!RSA}

Exponentiation of large integers is the basis of common
algorithms for public-key encryption.  Read the Wikipedia
page on the RSA algorithm\footnote{\url{wikipedia.org/wiki/RSA}}
and write functions to encode and decode messages.

\end{ex}


\section{Debugging}
\index{debugging}

As you work with bigger datasets it can become unwieldy to
debug by printing and checking data by hand.  Here are some
suggestions for debugging large datasets:

\begin{description}

\item[Scale down the input:] If possible, reduce the size of the
dataset.  For example if the program reads a text file, start with
just the first 10 lines, or with the smallest example you can find.
You can either edit the files themselves, or (better) modify the
program so it reads only the first {\tt n} lines.

If there is an error, you can reduce {\tt n} to the smallest
value that manifests the error, and then increase it gradually
as you find and correct errors.

\item[Check summaries and types:] Instead of printing and checking the
entire dataset, consider printing summaries of the data: for example,
the number of items in a dictionary or the total of a list of numbers.

A common cause of runtime errors is a value that is not the right
type.  For debugging this kind of error, it is often enough to print
the type of a value.

\item[Write self-checks:]  Sometimes you can write code to check
for errors automatically.  For example, if you are computing the
average of a list of numbers, you could check that the result is
not greater than the largest element in the list or less than
the smallest.  This is called a ``sanity check'' because it detects
results that are ``insane.''

\index{sanity check}
\index{consistency check}

Another kind of check compares the results of two different
computations to see if they are consistent.  This is called a
``consistency check.''

\item[Pretty print the output:] Formatting debugging output
can make it easier to spot an error.  We saw an example in
Section~\ref{factdebug}.  The {\tt pprint} module provides
a {\tt pprint} function that displays built-in types in
a more human-readable format.

\index{pretty print}
\index{pprint module}
\index{module!pprint}

\end{description}

Again, time you spend building scaffolding can reduce
the time you spend debugging.

\index{scaffolding}

\section{Glossary}

\begin{description}

\item[dictionary:] A mapping from a set of keys to their
corresponding values.
\index{dictionary}

\item[key-value pair:] The representation of the mapping from
a key to a value.
\index{key-value pair}

\item[item:] Another name for a key-value pair.
\index{item!dictionary}

\item[key:] An object that appears in a dictionary as the
first part of a key-value pair.
\index{key}

\item[value:] An object that appears in a dictionary as the
second part of a key-value pair.  This is more specific than
our previous use of the word ``value.''
\index{value}

\item[implementation:] A way of performing a computation.
\index{implementation}

\item[hashtable:] The algorithm used to implement Python
dictionaries.
\index{hashtable}

\item[hash function:] A function used by a hashtable to compute the
location for a key.
\index{hash function}

\item[hashable:] A type that has a hash function.  Immutable
types like integers,
floats and strings are hashable; mutable types like lists and
dictionaries are not.
\index{hashable}

\item[lookup:] A dictionary operation that takes a key and finds
the corresponding value.
\index{lookup}

\item[reverse lookup:] A dictionary operation that takes a value and finds
one or more keys that map to it.
\index{reverse lookup, dictionary}

\item[singleton:] A list (or other sequence) with a single element.
\index{singleton}

\item[call graph:] A diagram that shows every frame created during
the execution of a program, with an arrow from each caller to
each callee. 
\index{call graph}
\index{diagram!call graph}

\item[histogram:] A set of counters.
\index{histogram}

\item[memo:] A computed value stored to avoid unnecessary future 
computation.
\index{memo}

\item[global variable:]  A variable defined outside a function.  Global
variables can be accessed from any function.
\index{global variable}

\item[flag:] A boolean variable used to indicate whether a condition
is true.
\index{flag}

\item[declaration:] A statement like {\tt global} that tells the
interpreter something about a variable.
\index{declaration}

\end{description}

\section{Exercises}

\begin{ex}
\index{duplicate}

If you did Exercise~\ref{duplicate}, you already have
a function named \verb"has_duplicates" that takes a list
as a parameter and returns {\tt True} if there is any object
that appears more than once in the list.

Use a dictionary to write a faster, simpler version of
\verb"has_duplicates".
\end{ex}


\begin{ex}
\label{exrotatepairs}

\index{letter rotation}
\index{rotation!letters}

Two words are ``rotate pairs'' if you can rotate one of them
and get the other (see \verb"rotate_word" in Exercise~\ref{exrotate}).

Write a program that reads a wordlist and finds all the rotate
pairs.
\end{ex}


\begin{ex}
\index{Car Talk}
\index{Puzzler}

Here's another Puzzler from {\em Car
Talk}\footnote{\url{www.cartalk.com/content/puzzler/transcripts/200717}}:

\begin{quote}
This was sent in by a fellow named Dan O'Leary. He came upon a common
one-syllable, five-letter word recently that has the following unique
property. When you remove the first letter, the remaining letters form
a homophone of the original word, that is a word that sounds exactly
the same. Replace the first letter, that is, put it back and remove
the second letter and the result is yet another homophone of the
original word. And the question is, what's the word?

Now I'm going to give you an example that doesn't work. Let's look at
the five-letter word, `wrack.' W-R-A-C-K, you know like to `wrack with
pain.' If I remove the first letter, I am left with a four-letter
word, 'R-A-C-K.' As in, `Holy cow, did you see the rack on that buck!
It must have been a nine-pointer!' It's a perfect homophone. If you
put the `w' back, and remove the `r,' instead, you're left with the
word, `wack,' which is a real word, it's just not a homophone of the
other two words.

But there is, however, at least one word that Dan and we know of,
which will yield two homophones if you remove either of the first two
letters to make two, new four-letter words. The question is, what's
the word?
\end{quote}

\index{homophone}
\index{reducible word}
\index{word, reducible}

You can use the dictionary from Exercise~\ref{wordlist2} to check
whether a string is in the word list.

To check whether two words are homophones, you can use the CMU
Pronouncing Dictionary.  You can download it from
\url{www.speech.cs.cmu.edu/cgi-bin/cmudict} or from
\url{thinkpython.com/code/c06d} and you can also download
\url{thinkpython.com/code/pronounce.py}, which provides a function
named \verb"read_dictionary" that reads the pronouncing dictionary and
returns a Python dictionary that maps from each word to a string that
describes its primary pronunciation.

Write a program that lists all the words that solve the Puzzler.
You can see my solution at \url{thinkpython.com/code/homophone.py}.

\end{ex}



\chapter{Tuples}
\label{tuplechap}

\section{Tuples are immutable}

\index{tuple}
\index{type!tuple}
\index{sequence}

A tuple is a sequence of values.  The values can be any type, and
they are indexed by integers, so in that respect tuples are a lot
like lists.  The important difference is that tuples are immutable.

\index{mutability}
\index{immutability}

Syntactically, a tuple is a comma-separated list of values:

\beforeverb
\begin{verbatim}
>>> t = 'a', 'b', 'c', 'd', 'e'
\end{verbatim}
\afterverb
%
Although it is not necessary, it is common to enclose tuples in
parentheses:

\index{parentheses!tuples in}

\beforeverb
\begin{verbatim}
>>> t = ('a', 'b', 'c', 'd', 'e')
\end{verbatim}
\afterverb
%
To create a tuple with a single element, you have to include the final
comma:

\index{singleton}
\index{tuple!singleton}

\beforeverb
\begin{verbatim}
>>> t1 = ('a',)
>>> type(t1)
<type 'tuple'>
\end{verbatim}
\afterverb
%
Without the comma, Python treats \verb"('a')" as a string in
parentheses:

\beforeverb
\begin{verbatim}
>>> t2 = ('a')
>>> type(t2)
<type 'str'>
\end{verbatim}
\afterverb
%
Another way to create a tuple is the built-in function {\tt tuple}.
With no argument, it creates an empty tuple:

\index{tuple function}
\index{function!tuple}

\beforeverb
\begin{verbatim}
>>> t = tuple()
>>> print t
()
\end{verbatim}
\afterverb
%
If the argument is a sequence (string, list or tuple), the result
is a tuple with the elements of the sequence:

\beforeverb
\begin{verbatim}
>>> t = tuple('lupins')
>>> print t
('l', 'u', 'p', 'i', 'n', 's')
\end{verbatim}
\afterverb
%
Because {\tt tuple} is the name of a built-in function, you should
avoid using it as a variable name.

Most list operators also work on tuples.  The bracket operator
indexes an element:

\index{bracket operator}
\index{operator!bracket}

\beforeverb
\begin{verbatim}
>>> t = ('a', 'b', 'c', 'd', 'e')
>>> print t[0]
'a'
\end{verbatim}
\afterverb
%
And the slice operator selects a range of elements.

\index{slice operator}
\index{operator!slice}
\index{tuple!slice}
\index{slice!tuple}

\beforeverb
\begin{verbatim}
>>> print t[1:3]
('b', 'c')
\end{verbatim}
\afterverb
%
But if you try to modify one of the elements of the tuple, you get
an error:

\index{exception!TypeError}
\index{TypeError}
\index{item assignment}
\index{assignment!item}

\beforeverb
\begin{verbatim}
>>> t[0] = 'A'
TypeError: object doesn't support item assignment
\end{verbatim}
\afterverb
%
You can't modify the elements of a tuple, but you can replace
one tuple with another:

\beforeverb
\begin{verbatim}
>>> t = ('A',) + t[1:]
>>> print t
('A', 'b', 'c', 'd', 'e')
\end{verbatim}
\afterverb
%

\section{Tuple assignment}
\label{tuple assignment}

\index{tuple!assignment}
\index{assignment!tuple}
\index{swap pattern}
\index{pattern!swap}

It is often useful to swap the values of two variables.
With conventional assignments, you have to use a temporary
variable.  For example, to swap {\tt a} and {\tt b}:

\beforeverb
\begin{verbatim}
>>> temp = a
>>> a = b
>>> b = temp
\end{verbatim}
\afterverb
%
This solution is cumbersome; {\bf tuple assignment} is more elegant:

\beforeverb
\begin{verbatim}
>>> a, b = b, a
\end{verbatim}
\afterverb
%
The left side is a tuple of variables; the right side is a tuple of
expressions.  Each value is assigned to its respective variable.  
All the expressions on the right side are evaluated before any
of the assignments.

The number of variables on the left and the number of
values on the right have to be the same:

\index{exception!ValueError}
\index{ValueError}

\beforeverb
\begin{verbatim}
>>> a, b = 1, 2, 3
ValueError: too many values to unpack
\end{verbatim}
\afterverb
%
More generally, the right side can be any kind of sequence
(string, list or tuple).  For example, to split an email address
into a user name and a domain, you could write:

\index{split method}
\index{method!split}
\index{email address}

\beforeverb
\begin{verbatim}
>>> addr = 'monty@python.org'
>>> uname, domain = addr.split('@')
\end{verbatim}
\afterverb
%
The return value from {\tt split} is a list with two elements;
the first element is assigned to {\tt uname}, the second to
{\tt domain}.

\beforeverb
\begin{verbatim}
>>> print uname
monty
>>> print domain
python.org
\end{verbatim}
\afterverb
%

\section{Tuples as return values}

\index{tuple}
\index{value!tuple}
\index{return value!tuple}
\index{function, tuple as return value}

Strictly speaking, a function can only return one value, but
if the value is a tuple, the effect is the same as returning
multiple values.  For example, if you want to divide two integers
and compute the quotient and remainder, it is inefficient to
compute {\tt x/y} and then {\tt x\%y}.  It is better to compute
them both at the same time.

\index{divmod}

The built-in function {\tt divmod} takes two arguments and
returns a tuple of two values, the quotient and remainder.
You can store the result as a tuple:

\beforeverb
\begin{verbatim}
>>> t = divmod(7, 3)
>>> print t
(2, 1)
\end{verbatim}
\afterverb
%
Or use tuple assignment to store the elements separately:

\index{tuple assignment}
\index{assignment!tuple}

\beforeverb
\begin{verbatim}
>>> quot, rem = divmod(7, 3)
>>> print quot
2
>>> print rem
1
\end{verbatim}
\afterverb
%
Here is an example of a function that returns a tuple:

\beforeverb
\begin{verbatim}
def min_max(t):
    return min(t), max(t)
\end{verbatim}
\afterverb
%
{\tt max} and {\tt min} are built-in functions that find
the largest and smallest elements of a sequence.  \verb"min_max"
computes both and returns a tuple of two values.

\index{max function}
\index{function!max}
\index{min function}
\index{function!min}


\section{Variable-length argument tuples}

\index{variable-length argument tuple}
\index{argument!variable-length tuple}
\index{gather}
\index{parameter!gather}
\index{argument!gather}

Functions can take a variable number of arguments.  A parameter
name that begins with {\tt *} {\bf gathers} arguments into
a tuple.  For example, {\tt printall}
takes any number of arguments and prints them:

\beforeverb
\begin{verbatim}
def printall(*args):
    print args
\end{verbatim}
\afterverb
%
The gather parameter can have any name you like, but {\tt args} is
conventional.  Here's how the function works:

\beforeverb
\begin{verbatim}
>>> printall(1, 2.0, '3')
(1, 2.0, '3')
\end{verbatim}
\afterverb
%
You can combine the gather operator with required and positional
arguments:

\beforeverb
\begin{verbatim}
def pointless(required, optional=0, *args):
    print required, optional, args
\end{verbatim}
\afterverb
%
Run this function with 1, 2, 3 and 4 or more arguments and
make sure you understand what it does.

\index{scatter}
\index{argument scatter}

The complement of gather is {\bf scatter}.  If you have a
sequence of values and you want to pass it to a function
as multiple arguments, you can use the {\tt *} operator.
For example, {\tt divmod} takes exactly two arguments; it
doesn't work with a tuple:

\index{TypeError}
\index{exception!TypeError}

\beforeverb
\begin{verbatim}
>>> t = (7, 3)
>>> divmod(t)
TypeError: divmod expected 2 arguments, got 1
\end{verbatim}
\afterverb
%
But if you scatter the tuple, it works:

\beforeverb
\begin{verbatim}
>>> divmod(*t)
(2, 1)
\end{verbatim}
\afterverb
%

\begin{ex}
Many of the built-in functions use
variable-length argument tuples.  For example, {\tt max}
and {\tt min} can take any number of arguments:

\index{max function}
\index{function!max}
\index{min function}
\index{function!min}

\beforeverb
\begin{verbatim}
>>> max(1,2,3)
3
\end{verbatim}
\afterverb
%
But {\tt sum} does not.

\index{sum function}
\index{function!sum}

\beforeverb
\begin{verbatim}
>>> sum(1,2,3)
TypeError: sum expected at most 2 arguments, got 3
\end{verbatim}
\afterverb
%
Write a function called {\tt sumall} that takes any number
of arguments and returns their sum.

\end{ex}


\section{Lists and tuples}

\index{zip function}
\index{function!zip}

{\tt zip} is a built-in function that takes two or more sequences and
``zips'' them into a list\footnote{In Python 3.0, {\tt zip} returns an
  iterator of tuples, but for most purposes, an iterator behaves like
  a list.} of tuples where each tuple contains one element from each
sequence.

\index{Python 3.0}

This example zips a string and a list:

\beforeverb
\begin{verbatim}
>>> s = 'abc'
>>> t = [0, 1, 2]
>>> zip(s, t)
[('a', 0), ('b', 1), ('c', 2)]
\end{verbatim}
\afterverb
%
The result is a list of tuples where each tuple contains
a character from the string and the corresponding element from
the list.

\index{list!of tuples}

If the sequences are not the same length, the result has the
length of the shorter one.

\beforeverb
\begin{verbatim}
>>> zip('Anne', 'Elk')
[('A', 'E'), ('n', 'l'), ('n', 'k')]
\end{verbatim}
\afterverb
%
You can use tuple assignment in a {\tt for} loop to traverse a list of
tuples:

\index{traversal}
\index{tuple assignment}
\index{assignment!tuple}

\beforeverb
\begin{verbatim}
t = [('a', 0), ('b', 1), ('c', 2)]
for letter, number in t:
    print number, letter
\end{verbatim}
\afterverb
%
Each time through the loop, Python selects the next tuple in
the list and assigns the elements to {\tt letter} and 
{\tt number}.  The output of this loop is:

\index{loop}

\beforeverb
\begin{verbatim}
0 a
1 b
2 c
\end{verbatim}
\afterverb
%
If you combine {\tt zip}, {\tt for} and tuple assignment, you get a
useful idiom for traversing two (or more) sequences at the same
time.  For example, \verb"has_match" takes two sequences, {\tt t1} and
{\tt t2}, and returns {\tt True} if there is an index {\tt i}
such that {\tt t1[i] == t2[i]}:

\index{for loop}

\beforeverb
\begin{verbatim}
def has_match(t1, t2):
    for x, y in zip(t1, t2):
        if x == y:
            return True
    return False
\end{verbatim}
\afterverb
%
If you need to traverse the elements of a sequence and their
indices, you can use the built-in function {\tt enumerate}:

\index{traversal}
\index{enumerate function}
\index{function!enumerate}

\beforeverb
\begin{verbatim}
for index, element in enumerate('abc'):
    print index, element
\end{verbatim}
\afterverb
%
The output of this loop is:

\beforeverb
\begin{verbatim}
0 a
1 b
2 c
\end{verbatim}
\afterverb
%
Again.


\section{Dictionaries and tuples}

\index{dictionary}
\index{items method}
\index{method!items}
\index{key-value pair}

Dictionaries have a method called {\tt items} that returns a list of
tuples, where each tuple is a key-value pair\footnote{This behavior is
  slightly different in Python 3.0.}.

\beforeverb
\begin{verbatim}
>>> d = {'a':0, 'b':1, 'c':2}
>>> t = d.items()
>>> print t
[('a', 0), ('c', 2), ('b', 1)]
\end{verbatim}
\afterverb
%
As you should expect from a dictionary, the items are in no
particular order.

\index{dictionary!initialize}

Conversely, you can use a list of tuples to initialize
a new dictionary:

\beforeverb
\begin{verbatim}
>>> t = [('a', 0), ('c', 2), ('b', 1)]
>>> d = dict(t)
>>> print d
{'a': 0, 'c': 2, 'b': 1}
\end{verbatim}
\afterverb

Combining {\tt dict} with {\tt zip} yields a concise way
to create a dictionary:

\index{zip function!use with dict}

\beforeverb
\begin{verbatim}
>>> d = dict(zip('abc', range(3)))
>>> print d
{'a': 0, 'c': 2, 'b': 1}
\end{verbatim}
\afterverb
%
The dictionary method {\tt update} also takes a list of tuples
and adds them, as key-value pairs, to an existing dictionary.

\index{update method}
\index{method!update}

\index{traverse!dictionary}
\index{dictionary!traversal}

Combining {\tt items}, tuple assignment and {\tt for}, you
get the idiom for traversing the keys and values of a dictionary:

\beforeverb
\begin{verbatim}
for key, val in d.items():
    print val, key
\end{verbatim}
\afterverb
%
The output of this loop is:

\beforeverb
\begin{verbatim}
0 a
2 c
1 b
\end{verbatim}
\afterverb
%
Again.

\index{tuple!as key in dictionary}
\index{hashable}

It is common to use tuples as keys in dictionaries (primarily because
you can't use lists).  For example, a telephone directory might map
from last-name, first-name pairs to telephone numbers.  Assuming
that we have defined {\tt last}, {\tt first} and {\tt number}, we
could write:

\beforeverb
\begin{verbatim}
directory[last,first] = number
\end{verbatim}
\afterverb
%
The expression in brackets is a tuple.  We could use tuple
assignment to traverse this dictionary.

\index{tuple!in brackets}

\beforeverb
\begin{verbatim}
for last, first in directory:
    print first, last, directory[last,first]
\end{verbatim}
\afterverb
%
This loop traverses the keys in {\tt directory}, which are tuples.  It
assigns the elements of each tuple to {\tt last} and {\tt first}, then
prints the name and corresponding telephone number.

There are two ways to represent tuples in a state diagram.  The more
detailed version shows the indices and elements just as they appear in
a list.  For example, the tuple \verb"('Cleese', 'John')" would appear:

\index{state diagram}
\index{diagram!state}

\beforefig
\centerline{\includegraphics{figs/tuple1.eps}}
\afterfig

But in a larger diagram you might want to leave out the
details.  For example, a diagram of the telephone directory might
appear:

\beforefig
\centerline{\includegraphics{figs/dict2.eps}}
\afterfig

Here the tuples are shown using Python syntax as a graphical
shorthand.

The telephone number in the diagram is the complaints line for the
BBC, so please don't call it.



\section{Comparing tuples}

\index{comparison!tuple}
\index{tuple!comparison}
\index{sort method}
\index{method!sort}

The comparison operators work with tuples and other sequences;
Python starts by comparing the first element from each
sequence.  If they are equal, it goes on to the next elements,
and so on, until it finds elements that differ.  Subsequent
elements are not considered (even if they are really big).

\beforeverb
\begin{verbatim}
>>> (0, 1, 2) < (0, 3, 4)
True
>>> (0, 1, 2000000) < (0, 3, 4)
True
\end{verbatim}
\afterverb
%
The {\tt sort} function works the same way.  It sorts 
primarily by first element, but in the case of a tie, it sorts
by second element, and so on.  

This feature lends itself to a pattern called {\bf DSU} for 

\begin{description}

\item[Decorate] a sequence by building a list of tuples
with one or more sort keys preceding the elements from the sequence,

\item[Sort] the list of tuples, and

\item[Undecorate] by extracting the sorted elements of the sequence.

\end{description}

\label{DSU}
\index{DSU pattern}
\index{pattern!DSU}
\index{decorate-sort-undecorate pattern}
\index{pattern!decorate-sort-undecorate}

For example, suppose you have a list of words and you want to
sort them from longest to shortest:

\beforeverb
\begin{verbatim}
def sort_by_length(words):
    t = []
    for word in words:
       t.append((len(word), word))

    t.sort(reverse=True)

    res = []
    for length, word in t:
        res.append(word)
    return res
\end{verbatim}
\afterverb
%
The first loop builds a list of tuples, where each
tuple is a word preceded by its length.

{\tt sort} compares the first element, length, first, and
only considers the second element to break ties.  The keyword argument
{\tt reverse=True} tells {\tt sort} to go in decreasing order.

\index{keyword argument}
\index{argument!keyword}
\index{traversal}

The second loop traverses the list of tuples and builds a list of
words in descending order of length.

\begin{ex}
In this example, ties are broken by comparing words, so words
with the same length appear in alphabetical order.  For other
applications you might want to break ties at random.  Modify
this example so that words with the same length appear in
random order.  Hint: see the {\tt random} function in the
{\tt random} module.

\index{random module}
\index{module!random}
\index{random function}
\index{function!random}

\end{ex}


\section{Sequences of sequences}
\index{sequence}

I have focused on lists of tuples, but almost all of the examples in
this chapter also work with lists of lists, tuples of tuples, and
tuples of lists.  To avoid enumerating the possible combinations, it
is sometimes easier to talk about sequences of sequences.

In many contexts, the different kinds of sequences (strings, lists and
tuples) can be used interchangeably.  So how and why do you choose one
over the others?

\index{string}
\index{list}
\index{tuple}
\index{mutability}
\index{immutability}

To start with the obvious, strings are more limited than other
sequences because the elements have to be characters.  They are
also immutable.  If you need the ability to change the characters
in a string (as opposed to creating a new string), you might
want to use a list of characters instead.

Lists are more common than tuples, mostly because they are mutable.
But there are a few cases where you might prefer tuples:

\begin{enumerate}

\item In some contexts, like a {\tt return} statement, it is
syntactically simpler to create a tuple than a list.  In other
contexts, you might prefer a list.

\item If you want to use a sequence as a dictionary key, you
have to use an immutable type like a tuple or string.

\item If you are passing a sequence as an argument to a function,
using tuples reduces the potential for unexpected behavior
due to aliasing.

\end{enumerate}

Because tuples are immutable, they don't provide methods
like {\tt sort} and {\tt reverse}, which modify existing lists.
But Python provides the built-in functions {\tt sorted}
and {\tt reversed}, which take any sequence as a parameter
and return a new list with the same elements in a different
order.

\index{sorted function}
\index{function!sorted}
\index{reversed function}
\index{function!reversed}


\section{Debugging}

\index{debugging}
\index{data structure}
\index{shape error}
\index{error!shape}

Lists, dictionaries and tuples are known generically as {\bf data
  structures}; in this chapter we are starting to see compound data
structures, like lists of tuples, and dictionaries that contain tuples
as keys and lists as values.  Compound data structures are useful, but
they are prone to what I call {\bf shape errors}; that is, errors
caused when a data structure has the wrong type, size or composition.
For example, if you are expecting a list with one integer and I
give you a plain old integer (not in a list), it won't work.

\index{structshape module}
\index{module!structshape}

To help debug these kinds of errors, I have written a module
called {\tt structshape} that provides a function, also called
{\tt structshape}, that takes any kind of data structure as
an argument and returns a string that summarizes its shape.
You can download it from \url{thinkpython.com/code/structshape.py}

Here's the result for a simple list:

\beforeverb
\begin{verbatim}
>>> from structshape import structshape
>>> t = [1,2,3]
>>> print structshape(t)
list of 3 int
\end{verbatim}
\afterverb
%
A fancier program might write ``list of 3 int{\em s},'' but it
was easier not to deal with plurals.  Here's a list of lists:

\beforeverb
\begin{verbatim}
>>> t2 = [[1,2], [3,4], [5,6]]
>>> print structshape(t2)
list of 3 list of 2 int
\end{verbatim}
\afterverb
%
If the elements of the list are not the same type,
{\tt structshape} groups them, in order, by type:

\beforeverb
\begin{verbatim}
>>> t3 = [1, 2, 3, 4.0, '5', '6', [7], [8], 9]
>>> print structshape(t3)
list of (3 int, float, 2 str, 2 list of int, int)
\end{verbatim}
\afterverb
%
Here's a list of tuples:

\beforeverb
\begin{verbatim}
>>> s = 'abc'
>>> lt = zip(t, s)
>>> print structshape(lt)
list of 3 tuple of (int, str)
\end{verbatim}
\afterverb
%
And here's a dictionary with 3 items that map integers to strings.

\beforeverb
\begin{verbatim}
>>> d = dict(lt) 
>>> print structshape(d)
dict of 3 int->str
\end{verbatim}
\afterverb
%
If you are having trouble keeping track of your data structures,
{\tt structshape} can help.


\section{Glossary}

\begin{description}

\item[tuple:] An immutable sequence of elements.
\index{tuple}

\item[tuple assignment:] An assignment with a sequence on the
right side and a tuple of variables on the left.  The right
side is evaluated and then its elements are assigned to the
variables on the left.
\index{tuple assignment}
\index{assignment!tuple}

\item[gather:] The operation of assembling a variable-length
argument tuple.
\index{gather}

\item[scatter:] The operation of treating a sequence as a list of
arguments.
\index{scatter}

\item[DSU:] Abbreviation of ``decorate-sort-undecorate,'' a
pattern that involves building a list of tuples, sorting, and
extracting part of the result.
\index{DSU pattern}

\item[data structure:] A collection of related values, often
organized in lists, dictionaries, tuples, etc.
\index{data structure}

\item[shape (of a data structure):] A summary of the type,
size and composition of a data structure.
\index{shape}

\end{description}


\section{Exercises}

\begin{ex}
Write a function called \verb"most_frequent" that takes a string and
prints the letters in decreasing order of frequency.  Find text
samples from several different languages and see how letter frequency
varies between languages.  Compare your results with the tables at
\url{wikipedia.org/wiki/Letter_frequencies}.

\index{letter frequency}
\index{frequency!letter}

\end{ex}


\begin{ex}
\label{anagrams}

\index{anagram set}
\index{set!anagram}

More anagrams!

\begin{enumerate}

\item Write a program
that reads a word list from a file (see Section~\ref{wordlist}) and
prints all the sets of words that are anagrams.

Here is an example of what the output might look like:

\beforeverb
\begin{verbatim}
['deltas', 'desalt', 'lasted', 'salted', 'slated', 'staled']
['retainers', 'ternaries']
['generating', 'greatening']
['resmelts', 'smelters', 'termless']
\end{verbatim}
\afterverb
%
Hint: you might want to build a dictionary that maps from a
set of letters to a list of words that can be spelled with those
letters.  The question is, how can you represent the set of
letters in a way that can be used as a key?

\item Modify the previous program so that it prints the largest set
of anagrams first, followed by the second largest set, and so on.

\index{Scrabble}
\index{bingo}

\item In Scrabble a ``bingo'' is when you play all seven tiles in
your rack, along with a letter on the board, to form an eight-letter
word.  What set of 8 letters forms the most possible bingos?
Hint: there are seven.

% (7, ['angriest', 'astringe', 'ganister', 'gantries', 'granites',
% 'ingrates', 'rangiest'])

\index{metathesis}

\item Two words form a ``metathesis pair'' if you can transform one
  into the other by swapping two letters\footnote{This exercise is
    inspired by an example at \url{puzzlers.org}.}; for example,
  ``converse'' and ``conserve.''  Write a program that finds all of
  the metathesis pairs in the dictionary.  Hint: don't test all pairs
  of words, and don't test all possible swaps.

You can download a solution from \url{thinkpython.com/code/anagram_sets.py}.

\end{enumerate}
\end{ex}



\begin{ex}

\index{Car Talk}
\index{Puzzler}

Here's another Car Talk Puzzler\footnote{
\url{www.cartalk.com/content/puzzler/transcripts/200651}}:

\begin{quote}
What is the longest English word, that remains a valid English word,
as you remove its letters one at a time?

Now, letters can be removed from either end, or the middle, but you
can't rearrange any of the letters. Every time you drop a letter, you
wind up with another English word. If you do that, you're eventually
going to wind up with one letter and that too is going to be an
English word---one that's found in the dictionary. I want to know
what's the longest word and how many letters does it
have?

I'm going to give you a little modest example: Sprite. Ok? You start
off with sprite, you take a letter off, one from the interior of the
word, take the r away, and we're left with the word spite, then we
take the e off the end, we're left with spit, we take the s off, we're
left with pit, it, and I.
\end{quote}

\index{reducible word}
\index{word, reducible}

Write a program to find all words that can be reduced in this way,
and then find the longest one.

This exercise is a little more challenging than most, so here are
some suggestions:

\begin{enumerate}

\item You might want to write a function that takes a word and
  computes a list of all the words that can be formed by removing one
  letter.  These are the ``children'' of the word.

\index{recursive definition}
\index{definition!recursive}

\item Recursively, a word is reducible if any of its children
are reducible.  As a base case, you can consider the empty
string reducible.

\item The wordlist I provided, {\tt words.txt}, doesn't
contain single letter words.  So you might want to add
``I'', ``a'', and the empty string.

\item To improve the performance of your program, you might want
to memoize the words that are known to be reducible.

\end{enumerate}

You can see my solution at \url{thinkpython.com/code/reducible.py}.

\end{ex}




%\begin{ex}
%\url{wikipedia.org/wiki/Word_Ladder}
%\end{ex}




\chapter{Case study: data structure selection}

\section{Word frequency analysis}
\label{analysis}

As usual, you should at least attempt the following exercises
before you read my solutions.

\begin{ex}
Write a program that reads a file, breaks each line into
words, strips whitespace and punctuation from the words, and
converts them to lowercase.

\index{string module}
\index{module!string}

Hint: The {\tt string} module provides strings named {\tt whitespace},
which contains space, tab, newline, etc., and {\tt
  punctuation} which contains the punctuation characters.  Let's see
if we can make Python swear:

\beforeverb
\begin{verbatim}
>>> import string
>>> print string.punctuation
!"#$%&'()*+,-./:;<=>?@[\]^_`{|}~
\end{verbatim}
\afterverb
%
Also, you might consider using the string methods {\tt strip},
{\tt replace} and {\tt translate}.

\index{strip method}
\index{method!strip}
\index{replace method}
\index{method!replace}
\index{translate method}
\index{method!translate}

\end{ex}


\begin{ex}

\index{Project Gutenberg}

Go to Project Gutenberg (\url{gutenberg.net}) and download 
your favorite out-of-copyright book in plain text format.

\index{plain text}
\index{text!plain}

Modify your program from the previous exercise to read the book
you downloaded, skip over the header information at the beginning
of the file, and process the rest of the words as before.

Then modify the program to count the total number of words in
the book, and the number of times each word is used.

\index{word frequency}
\index{frequency!word}

Print the number of different words used in the book.  Compare
different books by different authors, written in different eras.
Which author uses the most extensive vocabulary?
\end{ex}


\begin{ex}
Modify the program from the previous exercise to print the
20 most frequently-used words in the book.
\end{ex}


\begin{ex}
Modify the previous program to read a word list (see
Section~\ref{wordlist}) and then print all the words in the book that
are not in the word list.  How many of them are typos?  How many of
them are common words that {\em should} be in the word list, and how
many of them are really obscure?
\end{ex}


\section{Random numbers}

\index{random number}
\index{number, random}
\index{deterministic}
\index{pseudorandom}

Given the same inputs, most computer programs generate the same
outputs every time, so they are said to be {\bf deterministic}.
Determinism is usually a good thing, since we expect the same
calculation to yield the same result.  For some applications, though,
we want the computer to be unpredictable.  Games are an obvious
example, but there are more.

Making a program truly nondeterministic turns out to be not so easy,
but there are ways to make it at least seem nondeterministic.  One of
them is to use algorithms that generate {\bf pseudorandom} numbers.
Pseudorandom numbers are not truly random because they are generated
by a deterministic computation, but just by looking at the numbers it
is all but impossible to distinguish them from random.

\index{random module}
\index{module!random}

The {\tt random} module provides functions that generate
pseudorandom numbers (which I will simply call ``random'' from
here on).

\index{random function}
\index{function!random}

The function {\tt random} returns a random float
between 0.0 and 1.0 (including 0.0 but not 1.0).  Each time you
call {\tt random}, you get the next number in a long series.  To see a
sample, run this loop:

\beforeverb
\begin{verbatim}
import random

for i in range(10):
    x = random.random()
    print x
\end{verbatim}
\afterverb
%
The function {\tt randint} takes parameters {\tt low} and
{\tt high} and returns an integer between {\tt low} and
{\tt high} (including both).

\index{randint function}
\index{function!randint}

\beforeverb
\begin{verbatim}
>>> random.randint(5, 10)
5
>>> random.randint(5, 10)
9
\end{verbatim}
\afterverb
%
To choose an element from a sequence at random, you can use
{\tt choice}:

\index{choice function}
\index{function!choice}

\beforeverb
\begin{verbatim}
>>> t = [1, 2, 3]
>>> random.choice(t)
2
>>> random.choice(t)
3
\end{verbatim}
\afterverb
%
The {\tt random} module also provides functions to generate
random values from continuous distributions including
Gaussian, exponential, gamma, and a few more.

\begin{ex}

\index{histogram!random choice}

Write a function named \verb"choose_from_hist" that takes
a histogram as defined in Section~\ref{histogram} and returns a 
random value from the histogram, chosen with probability
in proportion to frequency.  For example, for this histogram:

\beforeverb
\begin{verbatim}
>>> t = ['a', 'a', 'b']
>>> h = histogram(t)
>>> print h
{'a': 2, 'b': 1}
\end{verbatim}
\afterverb
%
your function should {\tt 'a'} with probability $2/3$ and \verb"'b'"
with probability $1/3$.
\end{ex}


\section{Word histogram}

Here is a program that reads a file and builds a histogram of the
words in the file:

\index{histogram!word frequencies}

\beforeverb
\begin{verbatim}
import string

def process_file(filename):
    h = dict()
    fp = open(filename)
    for line in fp:
        process_line(line, h)
    return h

def process_line(line, h):
    line = line.replace('-', ' ')
    
    for word in line.split():
        word = word.strip(string.punctuation + string.whitespace)
        word = word.lower()

        h[word] = h.get(word, 0) + 1

hist = process_file('emma.txt')
\end{verbatim}
\afterverb
%
This program reads {\tt emma.txt}, which contains the text of {\em
  Emma} by Jane Austen.

\index{Austin, Jane}

\verb"process_file" loops through the lines of the file,
passing them one at a time to \verb"process_line".  The histogram
{\tt h} is being used as an accumulator.

\index{accumulator!histogram}
\index{traversal}

\verb"process_line" uses the string method {\tt replace} to replace
hyphens with spaces before using {\tt split} to break the line into a
list of strings.  It traverses the list of words and uses {\tt strip}
and {\tt lower} to remove punctuation and convert to lower case.  (It
is a shorthand to say that strings are ``converted;'' remember that
string are immutable, so methods like {\tt strip} and {\tt lower}
return new strings.)

Finally, \verb"process_line" updates the histogram by creating a new
item or incrementing an existing one.

\index{update!histogram}

To count the total number of words in the file, we can add up
the frequencies in the histogram:

\beforeverb
\begin{verbatim}
def total_words(h):
    return sum(h.values())
\end{verbatim}
\afterverb
%
The number of different words is just the number of items in
the dictionary:

\beforeverb
\begin{verbatim}
def different_words(h):
    return len(h)
\end{verbatim}
\afterverb
%
Here is some code to print the results:

\beforeverb
\begin{verbatim}
print 'Total number of words:', total_words(hist)
print 'Number of different words:', different_words(hist)
\end{verbatim}
\afterverb
%
And the results:

\beforeverb
\begin{verbatim}
Total number of words: 161073
Number of different words: 7212
\end{verbatim}
\afterverb
%

\section{Most common words}

\index{DSU pattern}
\index{pattern!DSU}

To find the most common words, we can apply the DSU pattern;
\verb"most_common" takes a histogram and returns a list of
word-frequency tuples, sorted in reverse order by frequency:

\beforeverb
\begin{verbatim}
def most_common(h):
    t = []
    for key, value in h.items():
        t.append((value, key))

    t.sort(reverse=True)
    return t
\end{verbatim}
\afterverb
%
Here is a loop that prints the ten most common words:

\beforeverb
\begin{verbatim}
t = most_common(hist)
print 'The most common words are:'
for freq, word in t[0:10]:
    print word, '\t', freq
\end{verbatim}
\afterverb
%
And here are the results from {\em Emma}:

\beforeverb
\begin{verbatim}
The most common words are:
to      5242
the     5204
and     4897
of      4293
i       3191
a       3130
it      2529
her     2483
was     2400
she     2364
\end{verbatim}
\afterverb
%

\section{Optional parameters}

\index{optional parameter}
\index{parameter!optional}

We have seen built-in functions and methods that take a variable
number of arguments.  It is possible to write user-defined functions
with optional arguments, too.  For example, here is a function that
prints the most common words in a histogram

\beforeverb
\begin{verbatim}
def print_most_common(hist, num=10)
    t = most_common(hist)
    print 'The most common words are:'
    for freq, word in t[0:num]:
        print word, '\t', freq
\end{verbatim}
\afterverb

The first parameter is required; the second is optional.
The {\bf default value} of {\tt num} is 10.

\index{default value}
\index{value!default}

If you only provide one argument:

\beforeverb
\begin{verbatim}
print_most_common(hist)
\end{verbatim}
\afterverb

{\tt num} gets the default value.  If you provide two arguments:

\beforeverb
\begin{verbatim}
print_most_common(hist, 20)
\end{verbatim}
\afterverb

{\tt num} gets the value of the argument instead.  In other
words, the optional argument {\bf overrides} the default value.

\index{override}

If a function has both required and optional parameters, all
the required parameters have to come first, followed by the
optional ones.


\section{Dictionary subtraction}

\index{dictionary!subtraction}
\index{subtraction!dictionary}

Finding the words from the book that are not in the word list
from {\tt words.txt} is a problem you might recognize as set
subtraction; that is, we want to find all the words from one
set (the words in the book) that are not in another set (the
words in the list).

{\tt subtract} takes dictionaries {\tt d1} and {\tt d2} and returns a
new dictionary that contains all the keys from {\tt d1} that are not
in {\tt d2}.  Since we don't really care about the values, we
set them all to None.

\beforeverb
\begin{verbatim}
def subtract(d1, d2):
    res = dict()
    for key in d1:
        if key not in d2:
            res[key] = None
    return res
\end{verbatim}
\afterverb
%
To find the words in the book that are not in {\tt words.txt},
we can use \verb"process_file" to build a histogram for
{\tt words.txt}, and then subtract:

\beforeverb
\begin{verbatim}
words = process_file('words.txt')
diff = subtract(hist, words)

print "The words in the book that aren't in the word list are:"
for word in diff.keys():
    print word,
\end{verbatim}
\afterverb
%
Here are some of the results from {\em Emma}:

\beforeverb
\begin{verbatim}
The words in the book that aren't in the word list are:
 rencontre jane's blanche woodhouses disingenuousness 
friend's venice apartment ...
\end{verbatim}
\afterverb
%
Some of these words are names and possessives.  Others, like
``rencontre,'' are no longer in common use.  But a few are common
words that should really be in the list!

\begin{ex}

\index{set}
\index{type!set}

Python provides a data structure called {\tt set} that provides many
common set operations.  Read the documentation at
\url{docs.python.org/lib/types-set.html} and write a program
that uses set subtraction to find words in the book that are not in
the word list.
\end{ex}


\section{Random words}
\label{randomwords}

\index{histogram!random choice}

To choose a random word from the histogram, the simplest algorithm
is to build a list with multiple copies of each word, according
to the observed frequency, and then choose from the list:

\beforeverb
\begin{verbatim}
def random_word(h):
    t = []
    for word, freq in h.items():
        t.extend([word] * freq)

    return random.choice(t)
\end{verbatim}
\afterverb
%
The expression {\tt [word] * freq} creates a list with {\tt freq}
copies of the string {\tt word}.  The {\tt extend}
method is similar to {\tt append} except that the argument is
a sequence.

\begin{ex}
\label{randhist}

\index{algorithm}

This algorithm works, but it is not very efficient; each time you
choose a random word, it rebuilds the list, which is as big as
the original book.  An obvious improvement is to build the list
once and then make multiple selections, but the list is still big.

An alternative is:

\begin{enumerate}

\item Use {\tt keys} to get a list of the words in the book.

\item Build a list that contains the cumulative sum of the word
  frequencies (see Exercise~\ref{cumulative}).  The last item
  in this list is the total number of words in the book, $n$.
  
\item Choose a random number from 1 to $n$.  Use a bisection search
  (See Exercise~\ref{bisection}) to find the index where the random
  number would be inserted in the cumulative sum.

\item Use the index to find the corresponding word in the word list.

\end{enumerate}

Write a program that uses this algorithm to choose a random
word from the book.
\end{ex}



\section{Markov analysis}

\index{Markov analysis}

If you choose words from the book at random, you can get a
sense of the vocabulary, you probably won't get a sentence:

\beforeverb
\begin{verbatim}
this the small regard harriet which knightley's it most things
\end{verbatim}
\afterverb
%
A series of random words seldom makes sense because there
is no relationship between successive words.  For example, in
a real sentence you would expect an article like ``the'' to
be followed by an adjective or a noun, and probably not a verb
or adverb.

One way to measure these kinds of relationships is Markov
analysis\footnote{This case study is based on an example from
  Kernighan and Pike, {\em The Practice of Programming}, 1999.}, which
characterizes, for a given sequence of words, the probability of the
word that comes next.  For example, the song {\em Eric, the Half a
  Bee} begins:

\begin{quote}
Half a bee, philosophically, \\
Must, ipso facto, half not be. \\
But half the bee has got to be \\
Vis a vis, its entity. D'you see? \\
\\
But can a bee be said to be \\
Or not to be an entire bee \\
When half the bee is not a bee \\
Due to some ancient injury? \\
\end{quote}
%
In this text,
the phrase ``half the'' is always followed by the word ``bee,''
but the phrase ``the bee'' might be followed by either
``has'' or ``is''.

\index{prefix}
\index{suffix}
\index{mapping}

The result of Markov analysis is a mapping from each prefix
(like ``half the'' and ``the bee'') to all possible suffixes
(like ``has'' and ``is'').

\index{random text}
\index{text!random}

Given this mapping, you can generate a random text by
starting with any prefix and choosing at random from the
possible suffixes.  Next, you can combine the end of the
prefix and the new suffix to form the next prefix, and repeat.

For example, if you start with the prefix ``Half a,'' then the
next word has to be ``bee,'' because the prefix only appears
once in the text.  The next prefix is ``a bee,'' so the
next suffix might be ``philosophically,'' ``be'' or ``due.''

In this example the length of the prefix is always two, but
you can do Markov analysis with any prefix length.  The length
of the prefix is called the ``order'' of the analysis.

\begin{ex}
Markov analysis:

\begin{enumerate}

\item Write a program to read a text from a file and perform Markov
analysis.  The result should be a dictionary that maps from
prefixes to a collection of possible suffixes.  The collection
might be a list, tuple, or dictionary; it is up to you to make
an appropriate choice.  You can test your program with prefix
length two, but you should write the program in a way that makes
it easy to try other lengths.

\item Add a function to the previous program to generate random text
based on the Markov analysis.  Here is an example from {\em Emma}
with prefix length 2:

\begin{quote}
He was very clever, be it sweetness or be angry, ashamed or only
amused, at such a stroke. She had never thought of Hannah till you
were never meant for me?" "I cannot make speeches, Emma:" he soon cut
it all himself.
\end{quote}

For this example, I left the punctuation attached to the words.
The result is almost syntactically correct, but not quite.
Semantically, it almost makes sense, but not quite.

What happens if you increase the prefix length?  Does the random
text make more sense?

\index{mash-up}

\item Once your program is working, you might want to try a mash-up:
if you analyze text from two or more books, the random
text you generate will blend the vocabulary and phrases from
the sources in interesting ways.

\end{enumerate}
\end{ex}


\section{Data structures}

\index{data structure}

Using Markov analysis to generate random text is fun, but there is
also a point to this exercise: data structure selection.  In your
solution to the previous exercises, you had to choose:

\begin{itemize}

\item How to represent the prefixes.

\item How to represent the collection of possible suffixes.

\item How to represent the mapping from each prefix to
the collection of possible suffixes.

\end{itemize}

Ok, the last one is the easy; the only mapping type we have
seen is a dictionary, so it is the natural choice.

For the prefixes, the most obvious options are string,
list of strings, or tuple of strings.  For the suffixes,
one option is a list; another is a histogram (dictionary).

\index{implementation}

How should you choose?  The first step is to think about
the operations you will need to implement for each data structure.
For the prefixes, we need to be able to remove words from
the beginning and add to the end.  For example, if the current
prefix is ``Half a,'' and the next word is ``bee,'' you need
to be able to form the next prefix, ``a bee.''

\index{tuple!as key in dictionary}

Your first choice might be a list, since it is easy to add
and remove elements, but we also need to be able to use the
prefixes as keys in a dictionary, so that rules out lists.
With tuples, you can't append or remove, but you can use
the addition operator to form a new tuple:

\beforeverb
\begin{verbatim}
def shift(prefix, word):
    return prefix[1:] + (word,)
\end{verbatim}
\afterverb
%
{\tt shift} takes a tuple of words, {\tt prefix}, and a string, 
{\tt word}, and forms a new tuple that has all the words
in {\tt prefix} except the first, and {\tt word} added to
the end.

For the collection of suffixes, the operations we need to
perform include adding a new suffix (or increasing the frequency
of an existing one), and choosing a random suffix.

Adding a new suffix is equally easy for the list implementation
or the histogram.  Choosing a random element from a list
is easy; choosing from a histogram is harder to do
efficiently (see Exercise~\ref{randhist}).

So far we have been talking mostly about ease of implementation,
but there are other factors to consider in choosing data structures.
One is run time.  Sometimes there is a theoretical reason to expect
one data structure to be faster than other; for example, I mentioned
that the {\tt in} operator is faster for dictionaries than for lists,
at least when the number of elements is large.

But often you don't know ahead of time which implementation will
be faster.  One option is to implement both of them and see which
is better.  This approach is called {\bf benchmarking}.  A practical
alternative is to choose the data structure that is
easiest to implement, and then see if it is fast enough for the
intended application.  If so, there is no need to go on.  If not,
there are tools, like the {\tt profile} module, that can identify
the places in a program that take the most time.

\index{benchmarking}
\index{profile module}
\index{module!profile}

The other factor to consider is storage space.  For example, using a
histogram for the collection of suffixes might take less space because
you only have to store each word once, no matter how many times it
appears in the text.  In some cases, saving space can also make your
program run faster, and in the extreme, your program might not run at
all if you run out of memory.  But for many applications, space is a
secondary consideration after run time.

One final thought: in this discussion, I have implied that
we should use one data structure for both analysis and generation.  But
since these are separate phases, it would also be possible to use one
structure for analysis and then convert to another structure for
generation.  This would be a net win if the time saved during
generation exceeded the time spent in conversion.


\section{Debugging}
\index{debugging}

When you are debugging a program, and especially if you are
working on a hard bug, there are four things to try:

\begin{description}

\item[reading:] Examine your code, read it back to yourself, and
check that it says what you meant to say.

\item[running:] Experiment by making changes and running different
versions.  Often if you display the right thing at the right place
in the program, the problem becomes obvious, but sometimes you have to
spend some time to build scaffolding.

\item[ruminating:] Take some time to think!  What kind of error
is it: syntax, runtime, semantic?  What information can you get from
the error messages, or from the output of the program?  What kind of
error could cause the problem you're seeing?  What did you change
last, before the problem appeared?

\item[retreating:] At some point, the best thing to do is back
off, undoing recent changes, until you get back to a program that
works and that you understand.  Then you can starting rebuilding.

\end{description}

Beginning programmers sometimes get stuck on one of these activities
and forget the others.  Each activity comes with its own failure
mode.

\index{typographical error}

For example, reading your code might help if the problem is a
typographical error, but not if the problem is a conceptual
misunderstanding.  If you don't understand what your program does, you
can read it 100 times and never see the error, because the error is in
your head.

\index{experimental debugging}

Running experiments can help, especially if you run small, simple
tests.  But if you run experiments without thinking or reading your
code, you might fall into a pattern I call ``random walk programming,''
which is the process of making random changes until the program
does the right thing.  Needless to say, random walk programming
can take a long time.

\index{random walk programming}
\index{development plan!random walk programming}

You have to take time to think.  Debugging is like an
experimental science.  You should have at least one hypothesis about
what the problem is.  If there are two or more possibilities, try to
think of a test that would eliminate one of them.

Taking a break helps with the thinking.  So does talking.
If you explain the problem to someone else (or even yourself), you
will sometimes find the answer before you finish asking the question.

But even the best debugging techniques will fail if there are too many
errors, or if the code you are trying to fix is too big and
complicated.  Sometimes the best option is to retreat, simplifying the
program until you get to something that works and that you
understand.

Beginning programmers are often reluctant to retreat because
they can't stand to delete a line of code (even if it's wrong).
If it makes you feel better, copy your program into another file
before you start stripping it down.  Then you can paste the pieces
back in a little bit at a time.

Finding a hard bug requires reading, running, ruminating, and
sometimes retreating.  If you get stuck on one of these activities,
try the others.


\section{Glossary}

\begin{description}

\item[deterministic:] Pertaining to a program that does the same
thing each time it runs, given the same inputs.
\index{deterministic}

\item[pseudorandom:] Pertaining to a sequence of numbers that appear
to be random, but are generated by a deterministic program.
\index{pseudorandom}

\item[default value:] The value given to an optional parameter if no
argument is provided.
\index{default value}

\item[override:] To replace a default value with an argument.
\index{override}

\item[benchmarking:] The process of choosing between data structures
by implementing alternatives and testing them on a sample of the
possible inputs.  
\index{benchmarking}

\end{description}


\section{Exercises}

\begin{ex}

\index{word frequency}
\index{frequency!word}
\index{Zipf's law}

The ``rank'' of a word is its position in a list of words
sorted by frequency: the most common word has rank 1, the
second most common has rank 2, etc.

Zipf's law describes a relationship between the ranks and frequencies
of words in natural languages\footnote{See
  \url{wikipedia.org/wiki/Zipf's_law}}.  Specifically, it
predicts that the frequency, $f$, of the word with rank $r$ is:

\[ f = c r^{-s} \]
%
where $s$ and $c$ are parameters that depend on the language and the
text.  If you take the logarithm of both sides of this equation, you
get:

\index{logarithm}

\[ \log f = \log c - s \log r \]
%
So if you plot $\log f$ versus $\log r$, you should get
a straight line with slope $-s$ and intercept $\log c$.

Write a program that reads a text from a file, counts
word frequencies, and prints one line
for each word, in descending order of frequency, with
$\log f$ and $\log r$.  Use the graphing program of your
choice to plot the results and check whether they form
a straight line.  Can you estimate the value of $s$?
\end{ex}


\appendix

\chapter{Debugging}
\index{debugging}

Different kinds of errors can occur
in a program, and it is useful to distinguish among them
in order to track them down more quickly:

\begin{itemize}

\item Syntax errors are produced by Python when it is translating the
  source code into byte code.  They usually indicate that there is
  something wrong with the syntax of the program.  Example: Omitting
  the colon at the end of a {\tt def} statement yields the somewhat
  redundant message {\tt SyntaxError: invalid syntax}.

\item Runtime errors are produced by the interpreter if something goes
  wrong while the program is running.  Most runtime error messages
  include information about where the error occurred and what
  functions were executing.  

\item Semantic errors are problems with a program that runs without
  producing error messages but doesn't do the right thing.  Example:
  An expression may not be evaluated in the order you expect, yielding
  an incorrect result.

\end{itemize}

\index{syntax error}
\index{runtime error}
\index{semantic error}
\index{error!compile-time}
\index{error!syntax}
\index{error!runtime}
\index{error!semantic}
\index{exception}

The first step in debugging is to figure out which kind of
error you are dealing with.  Although the following sections are
organized by error type, some techniques are
applicable in more than one situation.


\section{Syntax errors}

\index{error message}

Syntax errors are usually easy to fix once you figure out what they
are.  Unfortunately, the error messages are often not helpful.
The most common messages are {\tt SyntaxError: invalid syntax} and
{\tt SyntaxError: invalid token}, neither of which is very informative.

On the other hand, the message does tell you where in the program the
problem occurred.  Actually, it tells you where Python
noticed a problem, which is not necessarily where the error
is.  Sometimes the error is prior to the location of the error
message, often on the preceding line.

\index{incremental development}
\index{development plan!incremental}

If you are building the program incrementally, you should have
a good idea about where the error is.  It will be in the last
line you added.

If you are copying code from a book, start by comparing
your code to the book's code very carefully.  Check every character.
At the same time, remember that the book might be wrong, so
if you see something that looks like a syntax error, it might be.

Here are some ways to avoid the most common syntax errors:

\index{syntax}

\begin{enumerate}

\item Make sure you are not using a Python keyword for a variable name.

\index{keyword}

\item Check that you have a colon at the end of the header of every
compound statement, including {\tt for}, {\tt while},
{\tt if}, and {\tt def} statements.

\index{header}
\index{colon}

\item Make sure that any strings in the code have matching
quotation marks.

\index{quotation mark}

\item If you have multiline strings with triple quotes (single or double), make
sure you have terminated the string properly.  An unterminated string
may cause an {\tt invalid token} error at the end of your program,
or it may treat the following part of the program as a string until it
comes to the next string.  In the second case, it might not produce an error
message at all!

\index{multiline string}
\index{string!multiline}

\item An unclosed opening operator---\verb+(+, \verb+{+, or
  \verb+[+---makes Python continue with the next line as part of the
  current statement.  Generally, an error occurs almost immediately in
  the next line.

\item Check for the classic {\tt =} instead of {\tt ==} inside
a conditional.

\index{conditional}

\item Check the indentation to make sure it lines up the way it
is supposed to.  Python can handle space and tabs, but if you mix
them it can cause problems.  The best way to avoid this problem
is to use a text editor that knows about Python and generates
consistent indentation.

\index{indentation}
\index{whitespace}

\end{enumerate}

If nothing works, move on to the next section...


\subsection{I keep making changes and it makes no difference.}

If the interpreter says there is an error and you don't see it, that
might be because you and the interpreter are not looking at the same
code.  Check your programming environment to make sure that the
program you are editing is the one Python is trying to run.

If you are not sure, try putting an obvious and deliberate syntax
error at the beginning of the program.  Now run it again.  If the
interpreter doesn't find the new error, you are not running the
new code.

There are a few likely culprits:

\begin{itemize}

\item You edited the file and forgot to save the changes before
running it again.  Some programming environments do this
for you, but some don't.

\item You changed the name of the file, but you are still running
the old name.

\item Something in your development environment is configured
incorrectly.

\item If you are writing a module and using {\tt import},
make sure you don't give your module the same name as one
of the standard Python modules.

\index{module!reload}
\index{reload function}
\index{function!reload}

\item If you are using {\tt import} to read a module, remember
that you have to restart the interpreter or use {\tt reload}
to read a modified file.  If you import the module again, it
doesn't do anything.

\end{itemize}

If you get stuck and you can't figure out what is going on, one
approach is to start again with a new program like ``Hello, World!,''
and make sure you can get a known program to run.  Then gradually add
the pieces of the original program to the new one.


\section{Runtime errors}

Once your program is syntactically correct,
Python can compile it and at least start running it.  What could
possibly go wrong?


\subsection{My program does absolutely nothing.}

This problem is most common when your file consists of functions and
classes but does not actually invoke anything to start execution.
This may be intentional if you only plan to import this module to
supply classes and functions.

If it is not intentional, make sure that you
are invoking a function to start execution, or execute one from
the interactive prompt.  Also see the ``Flow of Execution'' section
below.


\subsection{My program hangs.}
\index{infinite loop}
\index{hanging}

If a program stops and seems to be doing nothing, it is ``hanging.''
Often that means that it is caught in an infinite loop.  
If there is a particular loop that you suspect is the
problem, add a {\tt print} statement immediately before the loop that says
``entering the loop'' and another immediately after that says
``exiting the loop.''

\index{infinite loop}
\index{loop!infinite}
\index{condition}
\index{loop!condition}

Run the program.  If you get the first message and not the second,
you've got an infinite loop.  
If you think you have an infinite loop and you think you know
what loop is causing the problem, add a {\tt print} statement 
inside the loop as the last statement in the loop
loop that prints the values of the variables in
the condition and the value of the condition.

For example:

\beforeverb
\begin{verbatim}
while x > 0 and y < 0 :
    # do something to x
    # do something to y

    print  "x: ", x
    print  "y: ", y
    print  "condition: ", (x > 0 and y < 0)
\end{verbatim}
\afterverb
%
Now when you run the program, you will see three lines of output
for each time through the loop.  The last time through the
loop, the condition should be {\tt false}.  If the loop keeps
going, you will be able to see the values of {\tt x} and {\tt y},
and you might figure out why they are not being updated correctly.

\subsubsection{Flow of Execution}
\index{flow of execution}

If you are not sure how the flow of execution is moving through
your program, add {\tt print} statements to the beginning of each
function with a message like ``entering function {\tt foo},'' where
{\tt foo} is the name of the function.

Now when you run the program, it will print a trace of each
function as it is invoked.


\subsection{When I run the program I get an exception.}
\index{exception}
\index{runtime error}

If something goes wrong during runtime, Python
prints a message that includes the name of the
exception, the line of the program where the problem occurred,
and a traceback.

\index{traceback}

The traceback identifies the function that is currently running,
and then the function that invoked it, and then the function that
invoked {\em that}, and so on.  In other words, it traces the
sequence of function invocations that got you to where you are.  It
also includes the line number in your file where each of these
calls occurs.

The first step is to examine the place in the program where
the error occurred and see if you can figure out what happened.
These are some of the most common runtime errors:

\begin{description}

\item[NameError:]  You are trying to use a variable that doesn't
exist in the current environment.
Remember that local variables are local.  You
cannot refer to them from outside the function where they are defined.

\index{NameError}
\index{TypeError}
\index{exception!NameError}
\index{exception!TypeError}

\item[TypeError:] There are several possible causes:

\begin{itemize}

\item  You are trying to use a value improperly.  Example: indexing
a string, list, or tuple with something other than an integer.

\index{index}

\item There is a mismatch between the items in a format string and
the items passed for conversion.  This can happen if either the number
of items does not match or an invalid conversion is called for.

\index{format operator}
\index{operator!format}

\item You are passing the wrong number of arguments to a function or method.
For methods, look at the method definition and
check that the first parameter is {\tt self}.  Then look at the
method invocation; make sure you are invoking the method on an
object with the right type and providing the other arguments
correctly.

\end{itemize}

\item[KeyError:]  You are trying to access an element of a dictionary
using a key that the dictionary does not contain.

\index{KeyError}
\index{exception!KeyError}
\index{dictionary}

\item[AttributeError:] You are trying to access an attribute or method
that does not exist.  Check the spelling!  You can use
{\tt dir} to list the attributes that do exist.

If an AttributeError indicates that an object has {\tt NoneType},
that means that it is {\tt None}.  One common cause is forgetting
to return a value from a function; if you get to the end of
a function without hitting a {\tt return} statement, it returns
{\tt None}.  Another common cause is using the result from
a list method, like {\tt sort}, that returns {\tt None}.

\index{AttributeError}
\index{exception!AttributeError}

\item[IndexError:] The index you are using
to access a list, string, or tuple is greater than
its length minus one.  Immediately before the site of the error,
add a {\tt print} statement to display
the value of the index and the length of the array.
Is the array the right size?  Is the index the right value?

\index{IndexError}
\index{exception!IndexError}

\end{description}

\index{debugger (pdb)}
\index{Python debugger (pdb)}
\index{pdb (Python debugger)}

The Python debugger ({\tt pdb}) is useful for tracking down
Exceptions because it allows you to examine the state of the
program immediately before the error.  You can read
about {\tt pdb} at \url{docs.python.org/lib/module-pdb.html}.


\subsection{I added so many {\tt print} statements I get inundated with
output.}

\index{print statement}
\index{statement!print}

One of the problems with using {\tt print} statements for debugging
is that you can end up buried in output.  There are two ways
to proceed: simplify the output or simplify the program.

To simplify the output, you can remove or comment out {\tt print}
statements that aren't helping, or combine them, or format
the output so it is easier to understand.

To simplify the program, there are several things you can do.  First,
scale down the problem the program is working on.  For example, if you
are searching a list, search a {\em small} list.  If the program takes
input from the user, give it the simplest input that causes the
problem.

\index{dead code}

Second, clean up the program.  Remove dead code and reorganize the
program to make it as easy to read as possible.  For example, if you
suspect that the problem is in a deeply nested part of the program,
try rewriting that part with simpler structure.  If you suspect a
large function, try splitting it into smaller functions and testing them
separately.

\index{testing!minimal test case}
\index{test case, minimal}

Often the process of finding the minimal test case leads you to the
bug.  If you find that a program works in one situation but not in
another, that gives you a clue about what is going on.

Similarly, rewriting a piece of code can help you find subtle
bugs.  If you make a change that you think shouldn't affect the
program, and it does, that can tip you off.


\section{Semantic errors}
\index{semantic error}
\index{error!semantic}

In some ways, semantic errors are the hardest to debug,
because the interpreter provides no information
about what is wrong.  Only you know what the program is supposed to
do.

The first step is to make a connection between the program
text and the behavior you are seeing.  You need a hypothesis
about what the program is actually doing.  One of the things
that makes that hard is that computers run so fast.

You will often wish that you could slow the program down to human
speed, and with some debuggers you can.  But the time it takes to
insert a few well-placed {\tt print} statements is often short compared to
setting up the debugger, inserting and removing breakpoints, and
``stepping'' the program to where the error is occurring.

\subsection{My program doesn't work.}

You should ask yourself these questions:

\begin{itemize}

\item Is there something the program was supposed to do but
which doesn't seem to be happening?  Find the section of the code
that performs that function and make sure it is executing when
you think it should.

\item Is something happening that shouldn't?  Find code in
your program that performs that function and see if it is
executing when it shouldn't.

\item Is a section of code producing an effect that is not
what you expected?  Make sure that you understand the code in
question, especially if it involves invocations to functions or methods in
other Python modules.  Read the documentation for the functions you invoke.
Try them out by writing simple test cases and checking the results.

\end{itemize}

In order to program, you need to have a mental model of how
programs work.  If you write a program that doesn't do what you expect,
very often the problem is not in the program; it's in your mental
model.

\index{model, mental}
\index{mental model}

The best way to correct your mental model is to break the program
into its components (usually the functions and methods) and test
each component independently.  Once you find the discrepancy
between your model and reality, you can solve the problem.

Of course, you should be building and testing components as you
develop the program.  If you encounter a problem,
there should be only a small amount of new code
that is not known to be correct.


\subsection{I've got a big hairy expression and it doesn't
do what I expect.}

\index{expression!big and hairy}
\index{big, hairy expression}

Writing complex expressions is fine as long as they are readable,
but they can be hard to debug.  It is often a good idea to
break a complex expression into a series of assignments to
temporary variables.

For example:

\beforeverb
\begin{verbatim}
self.hands[i].addCard(self.hands[self.findNeighbor(i)].popCard())
\end{verbatim}
\afterverb
%
This can be rewritten as:

\beforeverb
\begin{verbatim}
neighbor = self.findNeighbor(i)
pickedCard = self.hands[neighbor].popCard()
self.hands[i].addCard(pickedCard)
\end{verbatim}
\afterverb
%
The explicit version is easier to read because the variable
names provide additional documentation, and it is easier to debug
because you can check the types of the intermediate variables
and display their values.

\index{temporary variable}
\index{variable!temporary}
\index{order of operations}
\index{precedence}

Another problem that can occur with big expressions is
that the order of evaluation may not be what you expect.
For example, if you are translating the expression
$\frac{x}{2 \pi}$ into Python, you might write:

\beforeverb
\begin{verbatim}
y = x / 2 * math.pi
\end{verbatim}
\afterverb
%
That is not correct because multiplication and division have
the same precedence and are evaluated from left to right.
So this expression computes $x \pi / 2$.

A good way to debug expressions is to add parentheses to make
the order of evaluation explicit:

\beforeverb
\begin{verbatim}
 y = x / (2 * math.pi)
\end{verbatim}
\afterverb
%
Whenever you are not sure of the order of evaluation, use
parentheses.  Not only will the program be correct (in the sense
of doing what you intended), it will also be more readable for
other people who haven't memorized the rules of precedence.


\subsection{I've got a function or method that doesn't return what I
expect.}
\index{return statement}
\index{statement!return}

If you have a {\tt return} statement with a complex expression,
you don't have a chance to print the {\tt return} value before
returning.  Again, you can use a temporary variable.  For
example, instead of:

\beforeverb
\begin{verbatim}
return self.hands[i].removeMatches()
\end{verbatim}
\afterverb
%
you could write:

\beforeverb
\begin{verbatim}
count = self.hands[i].removeMatches()
return count
\end{verbatim}
\afterverb
%
Now you have the opportunity to display the value of
{\tt count} before returning.


\subsection{I'm really, really stuck and I need help.}

First, try getting away from the computer for a few minutes.
Computers emit waves that affect the brain, causing these
symptoms:

\begin{itemize}

\item Frustration and rage.

\index{frustration}
\index{rage}
\index{debugging!emotional response}
\index{emotional debugging}

\item Superstitious beliefs (``the computer hates me'') and
magical thinking (``the program only works when I wear my
hat backward'').

\index{debugging!superstition}
\index{superstitious debugging}

\item Random walk programming (the attempt to program by writing
every possible program and choosing the one that does the right
thing).

\index{random walk programming}
\index{development plan!random walk programming}

\end{itemize}

If you find yourself suffering from any of these symptoms, get
up and go for a walk.  When you are calm, think about the program.
What is it doing?  What are some possible causes of that
behavior?  When was the last time you had a working program,
and what did you do next?

Sometimes it just takes time to find a bug.  I often find bugs
when I am away from the computer and let my mind wander.  Some
of the best places to find bugs are trains, showers, and in bed,
just before you fall asleep.


\subsection{No, I really need help.}

It happens.  Even the best programmers occasionally get stuck.
Sometimes you work on a program so long that you can't see the
error.  A fresh pair of eyes is just the thing.

Before you bring someone else in, make sure you are prepared.
Your program should be as simple
as possible, and you should be working on the smallest input
that causes the error.  You should have {\tt print} statements in the
appropriate places (and the output they produce should be
comprehensible).  You should understand the problem well enough
to describe it concisely.

When you bring someone in to help, be sure to give
them the information they need:

\begin{itemize}

\item If there is an error message, what is it
and what part of the program does it indicate?

\item What was the last thing you did before this error occurred?
What were the last lines of code that you wrote, or what is
the new test case that fails?

\item What have you tried so far, and what have you learned?

\end{itemize}

When you find the bug, take a second to think about what you
could have done to find it faster.  Next time you see something
similar, you will be able to find the bug more quickly.

Remember, the goal is not just to make the program
work.  The goal is to learn how to make the program work.



\chapter{Contributor List}
\section*{Contributor List for ``Python for Informatics''}

TBD

\section*{Contributor List for ``Think Python''}

\index{contributors}

(Allen B. Downey)

More than 100 sharp-eyed and thoughtful readers have sent in
suggestions and corrections over the past few years.  Their
contributions, and enthusiasm for this project, have been a
huge help.

For the detail on the nature of each of the contributions from
these individuals, see the ``Think Python'' text.

Lloyd Hugh Allen,
Yvon Boulianne,
Fred Bremmer,
Jonah Cohen,
Michael Conlon,
Benoit Girard,
Courtney Gleason and Katherine Smith,
Lee Harr,
James Kaylin,
David Kershaw,
Eddie Lam,
Man-Yong Lee,
David Mayo,
Chris McAloon,
Matthew J. Moelter,
Simon Dicon Montford,
John Ouzts,
Kevin Parks,
David Pool,
Michael Schmitt,
Robin Shaw,
Paul Sleigh,
Craig T. Snydal,
Ian Thomas,
Keith Verheyden,
Peter Winstanley,
Chris Wrobel,
Moshe Zadka,
Christoph Zwerschke,
James Mayer,
Hayden McAfee,
Angel Arnal,
Tauhidul Hoque and Lex Berezhny,
Dr. Michele Alzetta,
Andy Mitchell,
Kalin Harvey,
Christopher P. Smith,
David Hutchins,
Gregor Lingl,
Julie Peters,
Florin Oprina,
D.~J.~Webre,
Ken,
Ivo Wever,
Curtis Yanko,
Ben Logan,
Jason Armstrong,
Louis Cordier,
Brian Cain,
Rob Black,
Jean-Philippe Rey at Ecole Centrale Paris,
Jason Mader at George Washington University made a number
Jan Gundtofte-Bruun,
Abel David and Alexis Dinno,
Charles Thayer,
Roger Sperberg,
Sam Bull,
Andrew Cheung,
C. Corey Capel,
Alessandra,
Wim Champagne,
Douglas Wright,
Jared Spindor,
Lin Peiheng,
Ray Hagtvedt,
Torsten H\"{u}bsch,
Inga Petuhhov,
Arne Babenhauserheide,
Mark E. Casida,
Scott Tyler,
Gordon Shephard,
Andrew Turner,
Adam Hobart,
Daryl Hammond and Sarah Zimmerman,
George Sass,
Brian Bingham,
Leah Engelbert-Fenton,
Joe Funke,
Chao-chao Chen,
Jeff Paine,
Lubos Pintes,
Gregg Lind and Abigail Heithoff,
Max Hailperin,
Chotipat Pornavalai,
Stanislaw Antol,
Eric Pashman,
Miguel Azevedo,
Jianhua Liu,
Nick King,
Martin Zuther,
Adam Zimmerman,
Ratnakar Tiwari,
Anurag Goel,
Kelli Kratzer,
Mark Griffiths,
Roydan Ongie,
Patryk Wolowiec,
Mark Chonofsky,
Russell Coleman,
Wei Huang,
Karen Barber,
Nam Nguyen,
St\'{e}phane Morin,
and
Paul Stoop.

% ENDCONTRIB

\normalsize

\printindex

\clearemptydoublepage
%\blankpage
%\blankpage
%\blankpage


\end{document}
